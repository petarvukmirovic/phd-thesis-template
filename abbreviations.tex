\usepackage{xspace}

\newcommand{\eq}{\approx}
\newcommand{\noteq}{\napprox}

\newcommand\cst[1]{\mathsf{#1}}
\newcommand\imp{\ensuremath{\rightarrow}}
\newcommand\lequiv{\ensuremath{\leftrightarrow}}

\newcommand{\lland}{\mathrel\land}
\newcommand{\llor}{\mathrel\lor}
\newcommand{\newehoh}{\ensuremath{\lambda}E}

\newcommand{\tuple}[2]{\overline{#1}_{#2}}
\newcommand{\tuplen}[1]{\overline{#1}_{n}}

\newcommand{\lam}[2]{\ensuremath{\lambda #1.\> #2}}
\newcommand{\lamx}[1]{\lam{x}{#1}}

\newcommand{\ifalse}{\pmb{\bot}}
\newcommand{\itrue}{\pmb{\top}}
\newcommand{\inot}{\pmb{\neg\,}}
\newcommand{\iand}{\pmb{\land}}
\newcommand{\ior}{\pmb{\lor}}
\newcommand{\iimplies}{\pmb{\rightarrow}}
\newcommand{\iequiv}{\pmb{\leftrightarrow}}
\newcommand{\iforall}{\pmb{\forall}}
\newcommand{\iexists}{\pmb{\exists}}
\newcommand{\ieq}{\pmb{\approx}}
\newcommand{\ineq}{\pmb{\not\approx}}
\newcommand{\equi}{\longleftrightarrow^*_{\alpha\beta\eta}}


\newcommand{\neglit}[1]{\ensuremath{#1 \noteq \itrue}}
\newcommand{\poslit}[1]{\ensuremath{#1 \eq \itrue}}

\newcommand{\substterm}[2]{\ensuremath{#1(#2)}}
\newcommand{\sigmaterm}[1]{\ensuremath{\substterm{\sigma}{#1}}}

\newcommand{\substcl}[2]{\substterm{#1}{#2}}
\newcommand{\sigmacl}[1]{\ensuremath{\substcl{\sigma}{#1}}}

\newcommand\unif{\mathrel{\smash{\stackrel{\lower.5ex\hbox{\ensuremath{\scriptscriptstyle ?}}}{=}}}}
\newcommand\MATCH{\mathrel{{\scriptstyle\lesssim}^{\scriptscriptstyle{?\!}}}}
%\newcommand\UNIF{=^?}



\newcommand\infname[1]{\ensuremath{\textsc{#1}}}
% Inference rule
\newcommand{\namedinference}[3]{\ensuremath{\prftree[r]{\relax{\infname{#1}}}{\strut#2}{\strut#3}}}
\newcommand{\inference}[2]{\ensuremath{\namedinference{}{\strut#1}{\strut#2}}}

% Simplification rule
\newcommand{\namedsimp}[3]{\ensuremath{\prftree[d][r]{\relax{\infname{#1}}}{\strut#2}{\strut#3}}}
\newcommand{\simp}[2]{\ensuremath{\namedinference{}{\strut#1}{\strut#2}}}
\newcommand{\ourmodel}{\ensuremath{\mathscr{J}}}
\newcommand{\NumberOK}[1]{#1}

\newcommand{\lfhol}[0]{$\lambda$fHOL}
\newcommand{\appvar}[0]{\ensuremath{\mathsf{@}}}

\newcommand{\ourpara}[1]{\paragraph{#1}}

\newcommand{\hooklongrightarrow}{\lhook\joinrel\longrightarrow}
\newcommand\Unifarrow{\Longrightarrow}
\newcommand\Matcharrow{\Longrightarrow}
\newcommand\Pdtarrow{\leadsto}
\newcommand\Matchiiarrow{\hooklongrightarrow}
\newcommand\Unifier{\mathcalx{U}}
\newcommand\Var{\mathcalx{V}\!\mathit{ar}} %% TYPESETTING: hack
\newcommand\Term{\mathcalx{T}\kern-.4ex\mathit{erm}} %% TYPESETTING: hack
\newcommand\Gfpf{\mathcalx{G}\mathit{fpf}}
\newcommand\GfpfRTL{\Gfpf\kern.3ex'}
\newcommand\Fp{\mathcalx{F}\kern-.3ex\mathit{p}} %% TYPESETTING: hack
\newcommand\FpRTL{\Fp'}

\newcommand{\cmark}{\text{\ding{51}}}
\newcommand{\xmark}{\text{\ding{55}}}

\renewcommand\AA{{\textsf{A}}}
\newcommand\BB{{\textsf{B}}}
\newcommand\NN{{\textsf{N}}}

\def\cpp{C\nobreak\raisebox{.1ex}{+}\nobreak\raisebox{.1ex}{+}}
\newcommand\appE{\ensuremath{\cst{@}{+}\text{E}}}

\newcommand\MyFunction[2]{\textbf{function} \textsc{#1}(#2) \textbf{is} \\}
\newcommand\MyProcedure[2]{\textbf{procedure} \textsc{#1}(#2) \textbf{is} \\}
\newcommand\MyIf[1]{\textbf{if} #1 \textbf{then} \\}
\newcommand\MyIfWoThen[1]{\textbf{if} #1 \\}
\newcommand\MyThen[1]{\phantom{\textbf{if}} \quad #1 \textbf{then} \\}
\newcommand\MyElse{\textbf{else} \\}
\newcommand\MyElsIf[1]{\textbf{else} \textbf{if} #1 \textbf{then} \\}
\newcommand\MyForTo[2]{\textbf{for} #1 \textbf{to} #2 \textbf{do} \\}
\newcommand\MyDo{\textbf{do} \\}
\newcommand\MyForever{\textbf{forever} \textbf{do} \\}
\newcommand\MyWhileOfDo[1]{\textbf{while} #1}
\newcommand\MyForDownto[2]{\textbf{for} #1 \textbf{downto} #2 \textbf{do} \\}
\newcommand\MyWhile[1]{\textbf{while} #1 \textbf{do} \\}
\newcommand\MyReturn{\textbf{return} }

\newcommand\q{\noindent\hbox{}\quad}
\newcommand\qq{\q\q}
\newcommand\qqq{\qq\q}
\newcommand\qqqq{\qqq\q}
\newcommand\qqqqq{\qqqq\q}
\newcommand\qqqqqq{\qqqqq\q}
\newcommand\qqqqqqq{\qqqqqq\q}
\newcommand\qqqqqqqq{\qqqqqqq\q}
\newcommand\qqqqqqqqq{\qqqqqqqq\q}

\newenvironment{quotex}
 {\list{}{\rightmargin0pt}%
  \item\relax}
 {\endlist}

 \newcommand\CRS{\confrep{\texttt{ConjectureRelativeSymbol}}{\texttt{CRS}}}

 \let\oldleft\left
\let\oldright\right 
\def\left{\mathopen{}\mathclose\bgroup\oldleft}
\def\right{\aftergroup\egroup\oldright}

\global\long\def\makeop#1{\operatorname{#1}}
\global\long\def\seq#1{\overline{#1} }
\newcommandx\param[1][usedefault, addprefix=\global, 1=\seq x]{\lambda#1.\, }
\global\long\def\set#1#2{\left\{  #1\,\middle|\,#2\right\}  }
\newcommandx\jp[3][usedefault, addprefix=\global, 1=3]{\textnormal{JP #2#1.#3}}
\newcommand\nf[2]{{#1}\downarrow_{{\ifthenelse{\equal{#2}{}}{\beta}{#2}}}}

\newcommand\newunifarrow{\longrightarrow}
\newcommand\jpunifarrow{\Longrightarrow}

\definecolor{light-gray}{gray}{0.875}
\newcommand{\selected}[1]{\smash{\setlength{\fboxsep}{.3ex}\colorbox{light-gray}{\ensuremath{\vphantom{('q}{#1}}}}}
\newcommand{\unifrulename}[1]{\textsf{#1}}


\newtheoremstyle{normal}
{}{}{\slshape}{}{\bfseries}{.}{.5em}{}

\theoremstyle{normal}
\newtheorem{theorem}{Theorem}{\bfseries}{\slshape}
\newtheorem{corollary}[theorem]{Corollary}{\bfseries}{\slshape}
\newtheorem{lemma}[theorem]{Lemma}{\bfseries}{\slshape}
% \newtheorem*{proof}{Proof}{\itshape}{\rmfamily}
\numberwithin{theorem}{chapter}

\newtheoremstyle{example}
{}{}{\rmfamily}{}{\bfseries}{.}{.5em}{}

\theoremstyle{example}
\newtheorem{exa}[theorem]{Example}{\bfseries}{\rmfamily}
\newtheorem{defi}[theorem]{Definition}{\bfseries}{\slshape}
\newcommand\ord{\makeop{ord}}
\newcommand{\ptarrow}[1]{\ensuremath{\;\Longrightarrow^{#1}\;}}
\newcommand{\ptarrowid}[0]{\ptarrow{\text{id}}}
\newcommand{\hotofo}[1]{\lfloor#1\rfloor}
\newcommand{\nametodb}[1]{\ensuremath{\langle #1 \rangle_\textsf{db}}}
\newcommand{\cstdb}[2]{\ensuremath{\textsf{db}_{#1}^{#2}}}
\newcommand{\cstdba}[1]{\ensuremath{\cstdb{#1}{\alpha}}}
\newcommand{\dbvar}[1]{\ensuremath{\textbf{\textit{#1}}}}

\newcommand{\Sigmaty}{\Sigma_\mathsf{ty}}
\newcommand{\VV}{\mathscr{V}}
\newcommand{\Vty}{\VV_\mathsf{ty}}

\newcommand\foralltynospace[1]{\Pi\,#1.}
\newcommand\forallty[1]{\foralltynospace{#1}\;}
\newcommand{\typeargs}[1]{{\langle#1\rangle}}
\newcommand{\typ}[1]{{\mathit{#1}}}

\newcommand{\lfsup}{$\lambda$fSup}
\newcommand{\lsup}{$\lambda$Sup}
\newcommand{\osup}{$o\lambda$Sup}
\newcommand{\instset}{\ensuremath{\mathit{Inst}}}
