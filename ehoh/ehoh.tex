\chapter{Extending a Brainiac Prover to Lambda-Free Higher-Order Logic}
\setheader{Extending a Brainiac Prover to Lambda-Free Higher-Order Logic}
\label{ch:ehoh}

\renewcommand{\confrep}[2]{#2}
\includeversion{rep}
\excludeversion{conf}

\blfootnote{In this work I desgined, implemented and evaluated all changes to term representation, algorithms and indexing data structures.}

\authors{Joint work with\\
Jasmin~Blanchette, Simon~Cruanes and Stephan~Schulz}


\begin{abstract}
Decades of work have gone into developing efficient proof calculi, data
structures, algorithms, and heuristics for first-order automatic theorem
proving. Higher-order provers lag behind in terms of efficiency. Instead of
developing a new higher-order prover from the ground up, we propose to start
with the state-of-the-art superposition prover E and gradually enrich it with
higher-order features. We explain how to extend the prover's data structures,
algorithms, and heuristics to $\lambda$-free higher-order logic, a formalism
that supports partial application and applied variables. Our extension
outperforms the traditional encoding and appears promising as a stepping stone
toward full higher-order logic.    
\end{abstract}

\newpage

\section{Introduction}
\label{sec:ehoh:introduction}

\begin{sloppypar}
Superposition provers such as E~\cite{scv-19-e23}, SPASS \cite{wdfksw-09-spass},
and Vampire \cite{lkav-13-vampire} are among the most successful first-order
reasoning systems. They serve as backends in various frameworks, including
software verifiers (e.g., Why3 \cite{fp-13-why3}),
%
% \pagebreak[2]%
automatic higher-order theorem provers (e.g., \hbox{Leo-III} \cite{sb-21-leo3},
Satallax \cite{cb-12-satallax}), and \begin{rep}one-click \end{rep}``hammers'' in proof assistants
(e.g., HOLyHammer in HOL Light \cite{ku-15-holyhammer}, Sledgehammer in
Isabelle \cite{pb-12-sh}).


Dec\-ades of research have gone
into refining calculi, devising efficient data structures and algorithms,
and developing heuristics to guide proof search\begin{rep}
  \cite{ss-17-arcade}\end{rep}.
This work has mostly focused on first-order logic with equality.
\end{sloppypar}

\looseness=-1
Research on higher-order automatic provers has resulted
in systems such as LEO \cite{cbmk-98-leo}, \textsc{Leo}-II
\cite{bspt-15-leo2}, and Leo-III \cite{sb-21-leo3},
based on resolution and paramodulation, and Satallax \cite{cb-12-satallax},
based on \confrep{tableaux}{tableaux and SAT solving}. They
feature a ``cooperative'' architecture, pioneered by LEO: They
are full-fledged higher-order provers that regularly invoke an
external first-order prover \confrep{}{with a low time limit as a terminal
procedure, }in an attempt to finish the proof quickly using only
first-order reasoning. However, the first-order backend will succeed
only
if all the necessary higher-order reasoning has been performed,
meaning that much of the first-order reasoning is carried out by the
slower higher-order prover. As a result, this architecture leads to
suboptimal performance on largely first-order problems,
such as those that often arise
in interactive verification \cite{ns-13-leo2sh}. For example,
at the 2017 installment of the CADE ATP System Competition (CASC)
\cite{gs-17-casc}, Leo-III, which uses E as a backend, proved
652 out of 2000 first-order problems in the Sledgehammer division, compared
with 1185 for E on its own and 1433 for Vampire.

To obtain better performance, we propose to start with a competitive
first-order prover and extend it to full higher-order logic one
feature at a time.  Our goal is a \emph{graceful} extension, so that
the system behaves as before on first-order problems, performs mostly
like a first-order prover on typical, mildly higher-order problems,
and scales up to arbitrary higher-order problems, in keeping with the
zero-overhead principle: \emph{What you don't use, you
  don't pay for.}

As a stepping stone toward full higher-order logic, we initially restrict our
focus to a higher-order logic without \hbox{$\lambda$-expressions}
(Sect.~\ref{sec:ehoh:logic}). Compared with first-order logic, its distinguishing
features are partial application and applied variables. It
is rich enough to express the recursive equations of higher-order
combinators, such as $\cst{map}$ on lists:
%
\begin{align*}
\cst{map} \> f \> \cst{nil} & \eq \cst{nil}
&
\cst{map} \> f \> (\cst{cons} \; x \; \mathit{xs}) & \eq
\cst{cons} \; (f \> x) \; (\cst{map} \> f \> \mathit{xs})
\end{align*}

Our vehicle is E\begin{rep} \cite{ss-02-brainiac,scv-19-e23}\end{rep},
a prover developed primarily by Schulz. It is written in C and offers
good performance, with more emphasis on ``brainiac''
heuristics than on raw speed. E regularly scores among the top
systems at CASC and is usually the strongest open-source prover in
the relevant divisions. It also serves as a backend for
competitive higher-order provers. We refer to our extended version of
E as Ehoh. It corresponds to a prerelease version of E~2.5 configured with the
option \verb|--enable-ho|.%
\footnote{\url{https://github.com/eprover/eprover/commit/80946ac}}
%We plan to include the new higher-order inferences in the E 2.5 release.

The main challenges we faced concerned the
representation of types and terms
(Sect.~\ref{sec:ehoh:types-and-terms}), the unification
\confrep{algorithm}{and matching algorithms}
(Sect.~\ref{sec:ehoh:unif-match}), and the indexing data structures
(Sect.~\ref{sec:ehoh:indexing}). We also adapted the
inference rules (Sect.~\ref{sec:ehoh:inferences})\confrep{ and}{,} the
heuristics (Sect.~\ref{sec:ehoh:heuristics})\confrep{}{, and the
  preprocessor (Sect.~\ref{sec:ehoh:preprocessing})}.

\looseness=-1
A central aspect of our work is a set of techniques we call
\emph{prefix optimization}. Taking a traditional look at higher-order terms, they contain twice as many proper
subterms as first-order terms; for example,
$\cst{f}\;(\cst{g}\;\cst{a})\;\cst{b}$ contains not only the ``argument'' subterms
$\cst{g}\;\cst{a}$, $\cst{a}$, $\cst{b}$ but also the ``prefix'' subterms
$\cst{f}$, $\cst{f}\;(\cst{g}\;\cst{a})$, $\cst{g}$.
\begin{rep}Many operations, including superposition and rewriting, require
traversing all subterms of a term.\end{rep}
Using the optimization, the prover traverses subterms recursively in a
first-order fashion, considering all the prefixes of a given subterm
together. %, at little additional cost.
%
Our experiments (Sect.~\ref{sec:ehoh:evaluation}) show that Ehoh is
almost as fast as E on first-order problems and can also prove
higher-order problems that do not require synthesizing
$\lambda$-terms. As next steps, we plan to add support for
$\lambda$-terms and higher-order unification.

\section{Logic}
\label{sec:ehoh:logic}

\looseness=-1
Our logic is a variant of the intensional $\lambda$-free Boolean-free higher-order
logic (\lfhol{}) described by Bentkamp et al.\ %, Blanchette, Cruanes, and Waldmann
\cite[Sect.~2]{bbcw-21-lfho}, which could also be called ``applicative
first-order logic.'' In the spirit of FOOL \cite{kotelnikov-16-fool}, we
%% kotelnikov-15-fool
extend the syntax of this logic by erasing the distinction between terms and
formulas, and its semantics by interpreting the Boolean type $o$ as a domain
of cardinality~2. Functional extensionality can be obtained by adding
suitable axioms \cite[Sect.~3.1]{bbcw-21-lfho}.

\looseness=-1

This logic differs from the higher-order logic described in
Sect.~\ref{sec:pre:hol} three ways. First, $\lambda$-abstraction is disallowed.
Second, logical connectives are not part of the set of symbols. Instead, there
is a special inductive case in the definition of terms that defines formulas.
Third, subterms are defined in a more traditional way. 

\looseness=-1
For completeness, we provide the definition of terms. Terms, ranged over by $s,t,u,v$, are either
\emph{variables} $x, y, z, \dots$, (\emph{function}) \emph{symbols} $\cst{a},
\cst{b}, \cst{c}, \cst{d},\allowbreak \cst{f},\allowbreak \cst{g}, \ldots{}$
(often called ``constants'' in the higher-order literature), binary applications
$s \; t$, or Boolean terms $\itrue$, $\ifalse$, $\inot s$, $s \mathbin{\iand}
t$, $s \mathbin{\ior} t$, $s \mathbin{\iimplies} t$, $s \mathbin{\iequiv} t$,
$\iforall x.\> s$, $\iexists x.\> s$, $s \mathrel{\ieq} t$. E and Ehoh clausify
the input as a preprocessing step, producing a clause set in which the only
proper Boolean subterms are variables, $\itrue$, and $\ifalse$. Note that we use
lowercase letters for free variables as bound variables do not appear in
clauses. A term's \emph{arity} is the number of extra arguments it can take. If
$\iota$ is a base type, $\cst{f}$ has type $\iota \to \iota \to \iota$, and
$\cst{a}$ has type $\iota$, then $\cst{f}$ is binary, $\cst{f}\;\cst{a}$ is
unary, and $\cst{f}\;\cst{a}\;\cst{a}$ is nullary. Subterms are defined in the
traditional higher-order way; for example, $s \; t$ has all subterms of $s$ and
$t$ as subterms, in addition to $s \; t$ itself; as a consequence
$\cst{f} \; \cst{a}$ is a subterm of $\cst{f} \; \cst{a} \;\cst{a}$.


In this chapter, substitutions $\sigma$ are partial
functions of finite domain from variables to terms, written $\{ x_1 \mapsto s_1,
\ldots, x_m \mapsto s_m \}$, where each $s_i$ has the same type as $x_i$. The
substitution $\sigma[x \mapsto s]$ maps $x$ to $s$ and otherwise coincides with
$\sigma$. Applying $\sigma$ to a variable beyond $\sigma$'s domain is the
identity. We deviated from the view of substitutions in Section \ref{sec:pre:hol}
as it made proofs in Sections \ref{sec:ehoh:unif-match} and \ref{sec:ehoh:indexing}
easier. It is easy to check that both views are equivalent.
%Applying a substitution to a term~$t$ applies it homomorphically to $t$'s
%variables.

A well-known technique to support \lfhol{} %in first-order reasoning systems
is to use the \emph{applicative encoding}:
Every $n$-ary symbol is mapped to a nullary symbol, and
application is represented by a distinguished binary symbol $\cst{@}.$ Thus,
the \lfhol{} term $\cst{f} \; (x\; \cst{a}) \; \cst{b}$ is
encoded as the first-order term $\cst{@}(\cst{@}(\cst{f}, \cst{@}(x,
\cst{a})), \cst{b}).$ However, this representation is not graceful, since it
also introduces $\cst{@}$'s for terms within \lfhol's first-order fragment. By
doubling the size and depth of terms, the encoding clutters data
structures and slows down term traversals.
%By changing the root symbol of terms,
%it impacts proof search heuristics in subtle ways.
In our empirical evaluation, we find that the applicative encoding can
decrease the success rate by up to \NumberOK{15\%} (Sect.~\ref{sec:ehoh:evaluation}).
For these and further reasons, it
is not ideal (Sect.~\ref{sec:ehoh:discussion-and-related-work}).



\section{Types and Terms}
\label{sec:ehoh:types-and-terms}

\section{Unification and Matching}
\label{sec:ehoh:unif-match}

\section{Indexing Data Structures}
\label{sec:ehoh:indexing}

\section{Inference Rules}
\label{sec:ehoh:inferences}

\section{Heuristics}
\label{sec:ehoh:heuristics}

\section{Preprocessing}
\label{sec:ehoh:preprocessing}

\section{Evaluation}
\label{sec:ehoh:evaluation}

\section{Discussion and Related Work}
\label{sec:ehoh:discussion-and-related-work}

\section{Conclusion}
\label{sec:ehoh:conclusion}
