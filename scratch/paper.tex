%%%%%%%%%%%%%%%%%%%%%%% file template.tex %%%%%%%%%%%%%%%%%%%%%%%%%
%
% This is a general template file for the LaTeX package SVJour3
% for Springer journals.          Springer Heidelberg 2010/09/16
%
% Copy it to a new file with a new name and use it as the basis
% for your article. Delete % signs as needed.
%
% This template includes a few options for different layouts and
% content for various journals. Please consult a previous issue of
% your journal as needed.
%
%%%%%%%%%%%%%%%%%%%%%%%%%%%%%%%%%%%%%%%%%%%%%%%%%%%%%%%%%%%%%%%%%%%
%
% First comes an example EPS file -- just ignore it and
% proceed on the \documentclass line
% your LaTeX will extract the file if required

%
\RequirePackage{fix-cm}
%
%\documentclass{svjour3}                     % onecolumn (standard format)
\documentclass[smallcondensed,draft]{svjour3}     % onecolumn (ditto)
%\documentclass[smallextended]{svjour3}       % onecolumn (second format)
%\documentclass[twocolumn]{svjour3}          % twocolumn
%
\smartqed  % flush right qed marks, e.g. at end of proof
%
\usepackage[utf8]{inputenc}
\usepackage[misc]{ifsym}
\usepackage[final]{graphicx}
\usepackage{array}
\usepackage{multirow}
\usepackage{cite}
\usepackage{tikz}
\usepackage{pgfplots}
\usepackage{prftree}
\usepackage{amsmath}
\usepackage{amssymb}
\usepackage{caption}
\usepackage{subcaption}
\usepackage{enumerate}
\usepackage{stmaryrd}
\usepackage{booktabs}
\usepackage{algorithmicx}
\usepackage{algorithm}% http://ctan.org/pkg/algorithms
\usepackage[noend]{algpseudocode}% http://ctan.org/pkg/algorithmicx
\usepackage[
   a4paper,
   pdftex,
   pdftitle={Making Higher-Order Superposition Work},
   pdfauthor={Petar Vukmirovi\'c, Alexander Bentkamp, Jasmin Blanchette, Simon Cruanes, Visa Nummelin, and Sophie Tourret},
   pdfkeywords={},
   pdfborder={0 0 0},
   draft=false,
   bookmarksnumbered,
   bookmarks,
   bookmarksdepth=2,
   bookmarksopenlevel=2,
   bookmarksopen]{hyperref}

\usepackage{mathptmx}      % use Times fonts if available on your TeX system

\DeclareSymbolFont{letters}{OML}{txmi}{m}{it} %% for Greek

% insert here the call for the packages your document requires
%\usepackage{latexsym}
% etc.
%
% please place your own definitions here and don't use \def but
% \newcommand{}{}
%
% Insert the name of "your journal" with
\journalname{Journal of Automated Reasoning}

\makeatletter

\def\vthinspace{\kern+0.083333em}
\newcommand\CORR{{(\vthinspace\Letter\vthinspace)}}

\def\orcid#1{{\href{http://orcid.org/#1}{\protect\raisebox{-1.25pt}{\protect\includegraphics{orcid.pdf}}}}}

% mathcal
\DeclareFontFamily{OT1}{pzc}{}
\DeclareFontShape{OT1}{pzc}{m}{it}{<-> s * [1.10] pzcmi7t}{}
\DeclareMathAlphabet{\mathcalx}{OT1}{pzc}{m}{it}
%\DeclareMathAlphabet{\mathcalx}{OMS}{zplm}{m}{n}

%%% HACK Springer's style to get bold subsection headings --JB
\def\subsection{\@startsection{subsection}{2}{\z@}%
    {-21dd plus-8pt minus-4pt}{10.5dd}
     {\normalsize\bfseries\boldmath}}

\spnewtheorem{definitionx}[theorem]{Definition}{\bfseries}{\rmfamily}
\spnewtheorem{examplex}[theorem]{Example}{\bfseries}{\rmfamily}


\begin{document}

\title{Making Higher-Order Superposition Work}
% \subtitle{Do you have a subtitle?\\ If so, write it here}

%\titlerunning{Short form of title}        % if too long for running head

\author{Petar~Vukmirović~\orcid{0000-0001-7049-6847}\and
Alexander~Bentkamp~\orcid{0000-0002-7158-3595}\and
Jasmin~Blanchette~\orcid{0000-0002-8367-0936}\and
Simon~Cruanes~\orcid{0000-0003-3969-5850}\and
Visa~Nummelin~\orcid{0000-0003-0078-790X} \and
Sophie~Tourret~\orcid{0000-0002-6070-796X}}

\authorrunning{Petar Vukmirović et al.} % if too long for running head

\institute{
  Petar~Vukmirovi\'c \CORR \and
  Alexander Bentkamp\and
  Jasmin Blanchette\and
  Visa Nummelin\at%
    Vrije Universiteit Amsterdam, Amsterdam, the Netherlands\\
    \email{\{p.vukmirovic,a.bentkamp,j.c.blanchette,visa.nummelin\}@vu.nl}
  \and
  Jasmin Blanchette\and Sophie Tourret \at%
    Universit\'e de Lorraine, CNRS, Inria, LORIA, Nancy, France
    \\
    \email{\{jasmin.blanchette,sophie.tourret\}@inria.fr}
  \and
  Jasmin Blanchette\and Sophie Tourret \at  Max-Planck-Institut f\"ur Informatik, Saarbr\"ucken,~Germany \\
  \email{\{jblanche,stourret\}@mpi-inf.mpg.de}
  \and
  Simon Cruanes \at Imandra, Austin, Texas, USA \\
  \email{simon@imandra.ai}
}

\date{Received: date / Accepted: date}
% The correct dates will be entered by the editor


\maketitle

\begin{abstract}
Superposition is among the most successful calculi for first-order logic. Its
extension to higher-order logic introduces new challenges such as infinitely
branching inference rules, new possibilities such as reasoning about
Booleans, and the need to curb the explosion of specific higher-order rules. We
describe techniques that address these issues and extensively evaluate their
implementation in the Zipperposition theorem prover. Largely thanks to their use,
Zipperposition won the higher-order division of the CASC-J10 competition.
\end{abstract}

\newcommand{\ifalse}{\pmb\bot}
\newcommand{\itrue}{\pmb\top}
\newcommand{\inot}{\pmb{\neg\,}}
\newcommand{\iand}{\pmb\land}
\newcommand{\ior}{\pmb\lor}
\newcommand{\iimplies}{\pmb\rightarrow}
\newcommand{\iequiv}{\pmb\leftrightarrow}
\newcommand{\iforall}{\pmb\forall}
\newcommand{\iexists}{\pmb\exists}
\newcommand{\ieq}{\pmb\approx}
\newcommand{\ineq}{\pmb{\not\approx}}

\newcommand{\comb}[1]{\ensuremath{\mathsf{#1}}}
\newcommand{\neglit}[1]{\ensuremath{#1 \eq \ifalse}}
\newcommand{\poslit}[1]{\ensuremath{#1 \eq \itrue}}
\newcommand{\skbci}{\textsf{SKBCI}}
\def\negvthinspace{\kern-0.083333em}


\newcommand{\cst}[1]{\ensuremath{\mathsf{#1}}}
\newcommand{\var}[1]{{\mathit{#1}}}
\newcommand{\typ}[1]{{\mathit{#1}}}
\newcommand\foralltynospace[1]{\mathsf{\Pi}#1.}
\newcommand\forallty[1]{\foralltynospace{#1}\;}
\newcommand{\typeargs}[1]{{\negvthinspace\langle#1\rangle\negvthinspace}}
\let\oldDelta=\Delta
\let\oldSigma=\Sigma
\renewcommand\Sigma{\mathrm{\oldSigma}}
\newcommand{\Sigmaty}{\Sigma_\mathsf{ty}}
\newcommand{\VV}{\mathscr{V}}
\newcommand{\Vty}{\VV_\mathsf{ty}}


% Equality predicate
\newcommand{\eq}{\approx}
\newcommand{\noteq}{\not\eq}
\newcommand{\lland}{\mathrel\land}
\newcommand{\llor}{\mathrel\lor}
\newcommand{\ccup}{\mathrel\cup}
\newcommand{\ccap}{\mathrel\cap}

\newcommand{\tuple}[2]{\overline{#1}_{#2}}
\newcommand{\tuplen}[1]{\overline{#1}_{n}}

\newcommand{\lam}[2]{\ensuremath{\lambda #1.\> #2}}
\newcommand{\lamx}[1]{\lam{x}{#1}}

% Names of inference rules
\newcommand{\infname}[1]{\textsc{#1}}

% Inference rule
\newcommand{\namedinference}[3]{\prftree[r]{\relax{\infname{#1}}}{\strut#2}{\strut#3}}
\newcommand{\inference}[2]{\namedinference{}{\strut#1}{\strut#2}}

% Simplification rule
\newcommand{\namedsimp}[3]{\prftree[d][r]{\relax{\infname{#1}}}{\strut#2}{\strut#3}}
\newcommand{\simp}[2]{\namedinference{}{\strut#1}{\strut#2}}
\newcommand{\lsupraw}{$\lambda$-super\-po\-si\-tion}
\newcommand{\lsup}{{\color{black}\lsupraw}}
\newcommand{\lsup}{{\color{black}Boolean-free \lsupraw}}
\newcommand{\osup}{{\color{black}Boolean \lsupraw}}

\newcommand\NumberOK[1]{#1}
\newcommand\NumberNOK[1]{\colorbox{red}{#1}}
\newcommand{\ourpara}[1]{\paragraph{\upshape\bfseries#1}}

\hyphenation{su-per-po-si-tion Zip-per-posi-tion tab-leau tab-leaux Bent-kamp ab-strac-tion
  Bool-ean}

\newcommand{\confrep}[2]{#2}

\section{Introduction}
\label{sec:ho-tech:intro}

%In recent decades,
Superposition-based first-order automatic theorem provers
have emerged as useful reasoning tools. They dominate at the annual CASC
\cite{gs-2016-casc} theorem prover competition, having always won the
first-order theorem division. They are also used as backends to proof assistants
\cite{ck-18-coqhammer,ku-15-holyhammer,pb-12-sh}, automatic
higher-order theorem provers \cite{sb-21-leo3}, and software verifiers
\cite{fp-13-why3}.

The superposition calculus has only recently been extended
to higher-order logic (more precisely, extensional simple type theory
\cite{henkin-1950-completeness}), resulting in
\emph{\lsup} \cite{bbtvw-21-sup-lam}, which we developed
together with Waldmann, as well as \emph{combinatory superposition}
\cite{br-20-full-sup-w-combs} by Bhayat and Reger. Although these two
calculi do not support an interpreted Boolean type,
they can be extended by ad hoc rules \cite{our-bool-paper} that support
most of the Boolean reasoning necessary in practice.

Both higher-order superposition calculi were designed to gracefully
extend first-order reasoning. As most steps in higher-order
proofs tend to be essentially first-order, extending the most successful first-order
calculus to higher-order logic seemed worth trying.
Our first attempt at testing this idea was in 2019:
Zipperposition~1.5, based on \lsup, finished third
in the higher-order theorem division of CASC-27 \cite{gs-19-casc27},
12~percentage points behind the winner, the tableau prover Satallax 3.4 \cite{cb-2013-satallax}.

Studying the competition results, we found that higher-order tableaux have
some advantages over higher-order superposition. To bridge the gap, we developed
techniques and heuristics that simulate tableaux in the
context of saturation. We implemented them in Zipperposition~2, which took part
in CASC-J10 \cite{gs-21-cascj10} in 2020. This time, our prover won the division, proving 84\% of
the problems, a whole 20~percentage points ahead of the runner-up, Satallax
3.4.

In this article, we describe the main techniques that explain this reversal
of fortunes. They cover most parts of a modern higher-order theorem prover, from
preprocessing to additional calculus rules to heuristics to backend
integration. We use a newer version of Zipperposition, based on a newer
calculus:\ Instead of \lsup{} augmented with ad hoc Boolean rules,
we work with \emph{\osup} \cite{bbtv-21-full-ho-sup}, a principled extension of superposition to full
higher-order logic, including an interpreted Boolean type.

Many higher-order problems extensively use symbol definitions to simplify
their representation. We describe several ways to exploit the definitions,
%of which the most successful is
such as turning them into rewrite rules (Sect.~\ref{sec:ho-tech:preprocessing}).
%Interesting patterns can be observed in various higher-order problem encodings.
%We show how we can exploit these to simplify problems (Sect.~\ref{sec:ho-tech:preprocessing}).
%
By working on formulas rather than clauses, tableau techniques take a more
holistic view of a higher-order problem.
Through its support for delayed clausification and, more generally,
calculus-level formula manipulation, \osup{} enables us to
simulate most successful tableau techniques in a saturating prover
(Sect.~\ref{sec:ho-tech:formulas}). This calculus also supports \emph{Boolean selection
functions}, a mechanism that allows us to choose on which Boolean subterms
to perform inferences first.
We implemented some Boolean selection functions and
evaluated them (Sect.~\ref{sec:ho-tech:bool-select}).

The main drawback of both \lsup{} variants compared with combinatory
superposition is that they rely on rules that enumerate possibly infinite sets
of unifiers. We describe a mechanism that interleaves infinitely
branching inferences with the standard saturation process
(Sect.~\ref{sec:ho-tech:infinite-branching}). The prover
retains the same behavior as
before on first-order problems, smoothly scaling with
increasing numbers of higher-order clauses.
%
We also propose some heuristics to curb the explosion induced by highly
prolific calculus rules (Sect.~\ref{sec:ho-tech:explosiveness}).

Using first-order backends to finish the proof is common practice in
higher-order reasoning. Since \lsup{} coincides with standard
superposition on first-order clauses, invoking backends may
seem redundant; yet Zipperposition is nowhere as efficient as E
\cite{scv-19-e23} or Vampire \cite{lkav-13-vampire}, so invoking a more
efficient backend does make sense. We describe how to achieve a balance
between allowing native higher-order reasoning and
delegating reasoning to a backend (Sect.~\ref{sec:ho-tech:backends}).
%
Finally, we compare Zipperposition~2 with other provers on all monomorphic
higher-order TPTP benchmarks \cite{gs-17-tptp} to perform a more extensive
evaluation than at CASC (Sect.~\ref{sec:ho-tech:comparison}). Our evaluation
corroborates the competition results.

This article is an extended version of a paper accepted at CADE-28
\cite{making-ho-work}. Compared with the conference paper, it
describes a new preprocessing technique, explores the effects of Boolean
selection functions, evaluates %some previously unevaluated
more techniques,
introduces new benchmark sets, and presents more examples.

\section{Background and Setting}
\label{sec:ho-tech:background}

We focus on monomorphic higher-order logic, without the axiom of infinity 
or the axiom of at least two individuals. However, the techniques can easily be
extended with polymorphism. Indeed, Zipperposition already supports some
of them polymorphically.

\ourpara{Higher-Order Logic}
We define terms $s, t, u, v$ inductively as free variables $F, X$, bound
variables $x, y, z, \dotsc$, constants $\cst{f}, \cst{g},\allowbreak
\cst{a},\allowbreak \cst{b}, \dotsc$, term applications $s \, t$, and
$\lambda$-abstractions $\lamx{s}$. The syntactic distinction between free and
bound variables yields \emph{loose bound variables} (e.g., $y$ in $\lamx{y \,
\cst{a}}$) \cite{tn-93-patterns}.
%We shorten iterated application $s \, t_1 \, \cdots \, t_n$ to $s \,
%\overline{t}_n$ and $\lambda$-abstraction $\lambda x_1. \, \cdots \, \lambda
%x_n. \> s$ to $\lam{\overline{x}_n}{s}$.
We let $s \, \overline{t}_n$ stand for $s \, t_1 \, \ldots \, t_n$ and
$\lam{\overline{x}_n}{s}$ for $\lambda x_1. \ldots \lambda x_n. \> s$. The $n$-fold application of
a unary term $s$ to a term $t$ is denoted by $s^n \, t$. Every
$\beta$-normal term can be written as $\lam{\overline{x}_m}{s \,
\overline{t}_n}$, where $s$ is not an application; we call $s$ the \emph{head}
of the term. If the type of a term $t$ is of the form $\tau_1 \to \cdots \to
\tau_n \to o$, where $o$ is the distinguished Boolean type and $n \ge 0$, we
call $t$ a \emph{predicate}. A term of type $o$ is called a \emph{formula}.

A literal $l$ is an equation $s \eq t$ or a disequation $s \not\eq t$. A clause is
a finite multiset of literals, interpreted and written disjunctively $l_1 \llor
\cdots \llor l_n$. Logical symbols that may occur within terms are written in
boldface: $\pmb{\neg}, \iand, \ior, \iimplies, \iequiv, \dots$. Quantified
formulas are expressed using (a type-indexed
family of) constants $\iforall$ and $\iexists$ as $\iforall \,
(\lambda x. \, t)$ and $\iexists \, (\lambda x. \, t)$, usually abbreviated to
$\iforall x.\,t$ and $\iexists x.\,t$. Following %the convention of
\osup{}, predicate literals are encoded as equations with $\itrue$ or
$\ifalse$: for example, $\cst{even}(x)$ becomes $\poslit{\cst{even}(x)}$, and
$\neg\,\cst{even}(x)$ becomes $\neglit{\cst{even}(x)}$.

\ourpara{Higher-Order Calculi}
\looseness=-1
The \osup{} calculus \cite{bbtv-21-full-ho-sup} is a refutationally
complete inference system and redundancy criterion for higher-order logic with
rank-1 polymorphism, Hilbert choice, and functional and Boolean extensionality.
The calculus relies on
\emph{complete sets of unifiers}
(\emph{CSUs}). The CSU for $s$ and $t$ with respect to a finite set of variables
$V$, denoted by $\mathrm{CSU}_V(s,t)$, is a set of unifiers of $s$~and~$t$ such
that for any unifier $\varrho$ of $s$~and~$t$, there exist substitutions $\sigma
\in \mathrm{CSU}_V(s,t)$ and $\theta$ such that $\varrho(X) = \sigma(\theta(X))$
for all variables $X \in V$. The set $V$ is used to distinguish
important variables from auxiliary variables (which may arise in intermediary
states of the unification procedure). We usually omit it.

Unlike \lsup, this calculus 
does not require axioms defining the logical symbols to cope with formulas.
Instead, it includes Boolean inference rules that mimic
superposition from such axioms into Boolean subterms,
while avoiding the explosion incurred by adding these axioms to the proof state. It
also includes rules that simulate Boolean inferences below applied variables.
Both sets of rules are disabled or replaced with incomplete, ad hoc rules
described by Vukmirović and Nummelin \cite{our-bool-paper} in most configurations
of the CASC portfolio.
A new feature of the calculus that we explore in detail is
the ability to select Boolean subterms
to restrict Boolean and superposition inferences.

In contrast to both \lsup{} variants, combinatory superposition can
avoid enumerating CSUs by
using a form of first-order unification.
Essentially, it enumerates higher-order terms
using rules that instantiate applied variables with partially applied
combinators from the complete combinator set $\{\cst{S}, \cst{K}, \cst{B},
\cst{C}, \cst{I}\}$. This calculus is the basis of Vampire~4.5
\cite{br-20-full-sup-w-combs}, which finished
closely behind Satallax 3.4 %~and~3.5
at CASC-J10.

\looseness=-1
A different, very successful calculus is Satallax's SAT-guided tableaux
\cite{backes-brown-2011}. Satallax was the leading higher-order prover of the
2010s. Its simple and elegant tableaux avoid deep superposition-style rewriting
inferences.
Nevertheless, our working hypothesis for the past six years has been
that superposition would likely provide a stronger basis for higher-order
reasoning.
Other competing higher-order calculi include SMT (implemented in CVC4
\cite{brotb-19-ho-smt, cbetal-11-cvc4}) and extensional paramodulation (implemented in Leo-III \cite{sb-21-leo3}).


\ourpara{Zipperposition}
Zipperposition \cite{sc-15-simon-phd,bbtvw-21-sup-lam} is a higher-order
theorem prover that implements both \lsup{} variants, combinatory
superposition, and other superposition-like calculi.
The prover was conceived as a testbed for rapidly
experimenting with extensions of first-order superposition, but over time it
has assimilated many of E's techniques and heuristics and become quite powerful.

Several of our techniques extend the \emph{given clause procedure}, the standard
saturation procedure pioneered by McCune and Wos \cite[Sect.~2.3]{mcw-1997-otter}. It partitions the proof
state into a set $P$ of \emph{passive} clauses and a set $A$ of \emph{active}
clauses. Initially, $P$ contains all input clauses, and $A$ is empty. At each
iteration, a \emph{given} clause is moved from $P$ to $A$ (i.e., it is
\emph{activated}), all inferences between it and clauses in $A$ are performed,
and the conclusions are added to $P$. Because Zipperposition fully simplifies
clauses only when they are activated, it implements a DISCOUNT-style loop
\cite{adf-1995-discount}.

\ourpara{Experimental Setup}
To assess our techniques, we carried out experiments with Zipperposition~2. We
used two sets of benchmarks:\ all 2851~monomorphic higher-order problems from the
TPTP library \cite{gs-17-tptp} version~7.4.0 (labeled \emph{TPTP})
and 1253 Sledgehammer-generated
monomorphic higher-order problems (labeled \emph{SH}).
Although some techniques support polymorphism, we
uniformly used monomorphic benchmarks.

We fixed a \emph{base} configuration
of Zipperposition parameters as a baseline for all comparisons. This is an
incomplete, pragmatic configuration of \osup{} using heuristics expected to perform
well on a wide range of problems.
%Note that
%we use a different baseline configuration than in our earlier paper \cite{making-ho-work}.
%%% That goes without saying. We use a different calculus! Now clarified in intro. --JB
In each experiment, we varied
the parameters associated with a specific technique to evaluate it. The
experiments were run on StarExec Miami \cite{sst-14-starexec} servers, equipped with
Intel Xeon E5-2620 v4 CPUs clocked at 2.10 GHz. Unless otherwise stated, we used a
CPU time limit of 15~s, roughly the time each configuration is given in the
CASC portfolio mode. The raw evaluation results are available online.%
\footnote{\url{http://doi.org/10.5281/zenodo.5007440}}


\section{Preprocessing Higher-Order Problems}
\label{sec:ho-tech:preprocessing}

The TPTP library contains thousands of higher-order problems. Despite their
diversity, they have a markedly different flavor from the TPTP first-order
problems. Notably, they extensively use the \verb|definition| role to identify
universally quantified equations (and equivalences) that define symbols.
%
Definitions $s \eq t$ (or $(s \iequiv t) \eq \itrue$) can be replaced by rewrite
rules $s \rightarrow t$,
using the orientation given in the input problem. If there are multiple
definitions for the same symbol, only the first one is replaced by a rewrite rule.
Then, whenever a clause is picked in the given clause procedure, it will be rewritten
using the collected rules.
Alternatively, we can rewrite
the input formulas as a preprocessing step. This ensures that the input
clauses will be fully simplified when the proving process starts and no
defined symbols will occur in clauses, which usually helps the heuristics.

Since the TPTP format enforces no constraints on
definitions, rewriting might diverge. To ensure
termination, we limit the number of applied rewrite steps. In
practice, most TPTP problems are well behaved: Only one
definition is given for each symbol, and the definitions are acyclic.

Turning the defining equations into rewrite rules, unfolding the definitions, and
$\beta$-reduc\-ing the result can eliminate all of a problem's higher-order features, making
it amenable to first-order methods. However, this can inflate the problem
beyond recognition and compromise the refutational completeness of
superposition.

\begin{examplex} 
  Removing higher-order features of a problem can have adverse effects.
  Consider the TPTP problem \texttt{NUM636\^{}3}, which defines the predicate $\cst{m}$
  as $\lambda x.\, \cst{s} \, x \ineq x$ and states its conjecture as $\iforall
  x.\, \cst{m} \, x $, where $\cst{s}$ is the standard Peano-style natural number
  successor constructor. When this definition is kept as is, the
  prover can superpose from either $\cst{m}$ or its definition into the
  (clausified) induction axiom, which is also given in the problem, and quickly prove
  the conjecture, without using any advanced inductive reasoning. In contrast,
  when the definition is
  unfolded and the problem is $\beta$-reduced, both $\cst{m}$ and the
  corresponding $\lambda$-abstraction disappear, forcing the prover to guess the
  correct instantiation for the induction axiom.
\end{examplex}

We describe two techniques to mitigate these issues. The first one is based on the observation that in practice,
the explosion associated with definition unfolding mostly
manifests itself on definitions of nonpredicate symbols. In some cases, it is
preferable to rely on superposition's term order and powerful simplification
engine to rewrite the proof state rather than to blindly rewrite definitions. On
the other hand, superposition's reasoning with equivalences is often inadequate
\cite{bbtv-21-full-ho-sup, gs-05-boolsup}. Thus, it makes sense to treat only
predicate definitions as rewrite rules.

The second technique aims at preserving completeness: We can try to force the term order that
parameterizes superposition to orient as many definitions as possible and rely on
demodulation to simplify the proof state. Usually, the Knuth--Bendix order (KBO)
\cite{db-1970-kbo} is used. It compares terms by first comparing their weights,
which is the sum of all the weights assigned to the symbols it contains. Given a
symbol weight assignment $\mathcal{W}$, we can update it so that it orients
acyclic definitions from left to right assuming that they are of the form $
\cst{f} \, \overline{X}_m \eq \lambda \overline{Y}_n. \, t$, where the only free
variables in $t$ are $\overline{X}_m$, no free variable is repeated or appears
applied in $t$, and $\cst{f}$ does not occur in $t$. Then we traverse the
symbols $\cst{f}$ that are defined by such equations following the dependency
relation, starting with a symbol $\cst{f}$ that does not depend on any other
defined symbol. For each $\cst{f}$, we set $\mathcal{W}(\cst{f})$ to $w + 1$,
where $w$ is the maximum weight of the right-hand sides of $\cst{f}$'s
definitions, computed using $\mathcal{W}$. By construction, for each equation
the left-hand side is heavier. Thus, the equations are orientable from left to
right.



\begin{examplex} 
  Many of the problems in the TPTP library's \verb|LCL| category encode modal logic
  in higher-order logic. More complex modal operators (such as
  implication and equivalence) are defined in terms of basic connectives (such as negation
  and disjunction). Some of the definitions present in the problems are
  $\cst{mnot} := \lambda p\, x. \, \inot \, p \, x$, $\cst{mor} := \lambda p\, q\, x.
  \, p \, x \ior q \, x$,  and $\cst{mimplies} = \lambda p\, q. \allowbreak\, \cst{mor} \,
  (\cst{mnot} \, p) \, q$. Assuming that the weight of $\lambda$, bound
  variables, and basic connectives is 2, we can orient equations using
  the above described approach as follows. Starting from symbols that do not
  depend on the other ones, we set $\mathcal{W}(\cst{mnot}) = 11$ and
  $\mathcal{W}(\cst{mor}) = 17$. Then, we use these values to set
  $\mathcal{W}(\cst{mimplies}) = 37$. Clearly, these weights
  enable us to orient all definitions from left to right.
\end{examplex}


% Many higher-order problems, especially those generated from proof assistants,
% contain hundreds of needless axioms.
% To filter out axioms that are unlikely to be useful in a proof attempt, many
% theorem provers rely on the SInE algorithm \cite{hv-2011-sine}. SInE starts with
% the set of symbols occurring in the conjecture and tries to find axioms that
% define the properties of these symbols. Then it looks for the definitions of
% newly found symbols until it reaches a fixpoint. Axioms annotated with
% \verb|definition| ease this search because they explicitly record the
% dependency between a symbol and its characterization. To exploit
% this information, we modified SInE to optionally include the definitions of
% symbols in the conjecture, regardless of whether they are filtered out or not.
% Similarly, we implemented mode of SInE which selects only conjecture and axioms
% annotated with \verb|definition|.

\ourpara{Evaluation and Discussion}

% The \textit{base} configuration treats all axioms annotated with
% \texttt{definition} as rewrite rules applied as preprocessing.
% In addition, we tested
We designed and evaluated the following strategies for handling
\texttt{definition} axioms:

\begin{enumerate}[\rm no-RW$+$KBO~]
  \item[\rm pre-RW~] rewrite all definitions as a preprocessing step;
  \item[\rm in-RW~] rewrite all definitions during the saturation, as an inprocessing step;
  \item[\rm $o$-RW~] rewrite only predicate definitions, during preprocessing;
  \item[\rm $o$-RW$+$KBO~] like $o$-RW but with adjusted KBO weights for the remaining
    definitions;
  \item[\rm no-RW~] no special treatment of definitions;
  \item[\rm no-RW$+$KBO~] like $no$-RW but adjusting KBO weights for all definitions.
\end{enumerate}

The results are given in Fig.~\ref{fig:rewrite}. In all the figures, each
cell gives the number of proved problems, and cells marked with $\star$
correspond to the base configuration. The highest number in a category is typeset in
\relax{bold}. SH benchmarks are not
included because they do not contain the \texttt{definition} role.

\newcommand{\unknownres}{\ensuremath{{\varnothing}}}
\newcommand{\colalign}{\phantom{0}}

\begin{figure}[t]
  \centering
  \def\arraystretch{1.1}%
  \relax{\begin{tabular}{@{}l@{\hskip 1em}c@{\hskip 1em}c@{\hskip 1em}c@{\hskip 1em}c@{\hskip 1em}c@{\hskip 1em}c@{}} \toprule
                & pre-RW                 & in-RW        & $o$-RW               & $o$-RW$+$KBO    & no-RW        & no-RW$ + $KBO  \\ \midrule
           TPTP & {\bf 1635}$^\star$     & 1619         & 1620                 & 1621            & 1298         & 1296
           \\ \bottomrule
         \end{tabular}}
       \captionof{figure}{Impact of the definition rewriting method}
       \label{fig:rewrite}
     \end{figure}



% {\tt Points to discuss:
% \begin{itemize}
%   \item On TPTP problems, there are only 2 problem proved by no-RW$+$KBO but not no-RW --- both after 11s: probably because of some strange effects
%   \item On TPTP problems, there are 4 problems proved by o-RW$+$KBO but not o-RW --- 2 of them under 1 second: suppressing superposition into rhs of definitions
%   \item 13 problems (some of them of rating 1) proved by oRW but not regular one
%   %  zip-jar-hot_defs_oRW(13) : NUM/NUM638^4.p,SEU/SEU800^1.p,NUM/NUM701^4.p,NUM/NUM699^4.p,SEU/SEU603^1.p,NUM/NUM647^4.p,SEU/SEU695^2.p,SEU/SEU630^2.p,NUM/NUM648^4.p,SEU/SEU540^1.p,SYO/SYO531^1.p,SEU/SEU594^1.p,SEU/SEU648^1.p
%   \item SH problems do not use definition tag, so this is just noise.
% \end{itemize}
% }

%%% @PETAR: I find this para very confusing. I don't know when you're talking
%%% about the old paper's numbers and the present one's. Lert's keep it simple
%%% and avoid mentions to the old paper. It's subsumed.
%%%
%%% Also, careful with words like "counterintuitively". What's counterintuitive
%%% to you might be intuitive to the reader. E.g. my experience is that hardly
%%% any well-meant change to a prover makes a big difference on performance,
%%% so I certainly wasn't surprised.
%
%In our earlier paper \cite{making-ho-work}, only an
%approximation of KBO weight adjustment was implemented: no checks for loops and
%no topological sorting was performed. For this article, we implemented the
%feature as described above. Counterintuitively, this did not substantially
%influence the performance of Zipperposition:

The four configurations in which definitions are treated as rewrite rules
perform much better than the other two. In contrast, adjusting KBO weights gives
no substantial improvement: Looking at raw data, we found only
\NumberOK{2}~problems proved by $o$-RW$+$KBO but not by $o$-RW in which the
feature was used in the proof. For no-RW and no-RW$+$KBO, the
\NumberOK{2}-problem difference may be just noise. Even though it proves fewer
problems, the configuration $o$-RW has some advantages over pre-RW: It proves
\NumberOK{16} problems that pre-RW does not, \NumberOK{3} of which have a TPTP
difficulty rating (the ratio of eligible %TPTP
provers that cannot prove the problem) of~1.

% \looseness=-1
Rewriting after clausification avoids getting stuck rewriting parts of the
proof state that might not contribute to the proof. In practice, we noticed that
rewriting can be so expensive that the prover can spend all
allotted CPU time in the preprocessing phase. The evaluation results confirm this
observation: There are \NumberOK{64} problems proved by in-RW but not by
pre-RW. Moreover, there are \NumberOK{41} problems that can be proved only
by in-RW but not by any other above described configuration. % from Fig.~\ref{fig:rewrite}.

\section{Reasoning about Formulas}
\label{sec:ho-tech:formulas}

Higher-order logic identifies formulas with terms of Boolean type. To prove a problem, we often
need to instantiate a variable with the right predicate.
Finding this predicate can be easier if the problem is not clausified.
Consider the conjecture $\iexists f. \, f \, \cst{p} \, \cst{q} \iequiv \cst{p}
\iand \cst{q}$. Expressed in this form, the formula is easy to prove by taking
$f := \lambda x \, y. \> x  \iand y$. By contrast, guessing the right
instantiation for the negated, clausified form $ \neglit{F \, \cst{p} \,
\cst{q}} \llor \neglit{\cst{p}} \llor \neglit{\cst{q}},
\poslit{F \, \cst{p} \, \cst{q}} \llor \poslit{\cst{p}}$, $\poslit{F \, \cst{p}
\, \cst{q}} \llor \poslit{\cst{q}}$ is more challenging.
One of the strengths of higher-order tableau provers is that they do not clausify the input
problem. This might partly explain Satallax's dominance in the THF division of CASC
competitions until CASC-J10.

% Traditionally, superposition provers clausify the input problem, simplify the
% initial clause set, and start the saturation loop. We propose to integrate
% clausification tightly in the saturation loop.
% %
% Instead of clausifying the input, we represent the input formula $\varphi$
% as a higher-order clause $\poslit{\varphi}$. Then we use Vukmirović and Nummelin's
% \emph{lazy clausification} rules \cite[Sect.~3.4]{our-bool-paper},
% which extend \lsup{}. These incrementally clausify top-level
% logical symbols; for example, a clause $C' \llor \neglit{(\cst{p} \iand \cst{q})}$ yields a
% new clause $C' \llor \neglit{\cst p} \llor \neglit{\cst q}$.
% Lazy clausification rules admit advanced
% forms of formula renaming and Skolem symbol reusing that are not available in
% standard clausification algorithms.
% \looseness=-1
The \osup{} calculus supports \emph{delayed clausification rules}
that insert problems into the proof state in their original,
nonclausified form, and clausify them gradually. 
Delayed clausification
allows the prover to analyze the syntactic structure of formulas during saturation,
whereas the more traditional approach of \emph{immediate clausification}
applies a standard clausification algorithm \cite{nw-01-small-cnf}
both as a preprocessing step and whenever predicate variables are instantiated.

An earlier evaluation of the \osup{} calculus \cite{bbtv-21-full-ho-sup} showed
that the \emph{outer} variant of delayed clausification substantially increases this calculus's performance.
The outer variant clausifies
top-level logical symbols, proceeding from the outside inwards;
for example, a clause $C \llor \neglit{(\cst{p} \iand \cst{q})}$ is transformed into $C \llor
\neglit{\cst p} \llor \neglit{\cst q}$. The calculus also supports \emph{inner} delayed
clausification, which uses only the core calculus rules to clausify problems.
Even though this is the laziest approach to clausification, the earlier
evaluation showed that this approach is inefficient. Thus, we focus only on the
outer rules.

\looseness=-1
Delayed clausification rules can be used as inference rules (which add conclusions
to the passive set) or as simplification rules (which delete premises and add
conclusions to the passive set).
%
Inferences give more flexibility, since all
intermediate clausification states will be stored in the proof state, at the
cost of producing many clauses. Simplifications produce fewer clauses,
but risk destroying informative syntactic structure.
Since clausifying equivalences can destroy a lot of syntactic structure
\cite{gs-05-boolsup}, we never apply simplifying rules
on them.

Delayed clausification can interfere with clause splitting techniques.
Zipperposition supports a lightweight variant of AVATAR \cite{av-2014-avatar},
an architecture that partitions the search space by
splitting clauses into variable-disjoint subclauses. This lightweight AVATAR is described
by Ebner et al.\ \cite[Sect.~7]{2021-ebt-unifying-splitting}. Combining it
with delayed clausification makes it possible to split a %higher-order
clause $(\varphi_1 \ior \cdots \ior \varphi_n) \eq \itrue$, where
the $\varphi_i$'s are arbitrarily complex formulas that share no free
variables with each other, into clauses $\varphi_i \eq \itrue$.
%
To finish the proof, it suffices to derive the empty clause under each assumption
$\varphi_i \eq \itrue$. Since the split is performed at the formula level, this
technique resembles tableaux, but it exploits the strengths of superposition,
such as its powerful redundancy criterion and simplification machinery, to
close the branches.

\newcommand{\instset}{\ensuremath{\mathit{Inst}}}
% When delayed clausification is used, the prover can make
% heuristic choices based on information obtained from the formulas on which
% clausification rules are applied.
% In particular, we look for $\lambda$-abstractions whose bodies are formulas
% in each active clause and store them in a set
% $\mathit{Inst}$. Optionally, $\mathit{Inst}$ can contain \emph{primitive
% instantiations} \cite{our-bool-paper}---that is, imitations (in the sense
% of higher-order unification) of logical symbols that approximate the shape of
% a formula that can instantiate a predicate variable. These instantiations resemble
% the ones tableau provers perform during the proof search.

Beyond splitting, interleaving clausification and saturation allows us to simulate another tableau-inspired
technique. Whenever dynamic clausification substitutes a fresh variable $X$ for
a predicate variable $x$ in a clause of the form $(\iforall x.\, \varphi) \eq
\itrue \llor C$, yielding $\varphi\{x \mapsto\nobreak X\}
\eq \itrue \llor C$, we can create additional clauses in which $x$ is replaced
with $t \in \instset$, where $\instset$ is a set of heuristically chosen terms.
This set contains $\lambda$-abstractions whose bodies are formulas and that
occur in activated clauses, and \emph{primitive instantiations}
\cite{our-bool-paper}---that is, imitations (in the sense of higher-order
unification) of logical symbols that approximate the shape of a predicate that
can instantiate a predicate variable.

Since a new term $t$ can be added to $\mathit{Inst}$ after a clause with a
quantified variable of $t$'s type has been activated, we 
remember the clauses $\varphi\{x \mapsto X\} \eq \itrue\allowbreak \llor C$ and instantiate
them when $\mathit{Inst}$ is extended.
Conveniently, these instantiated clauses are not recognized as subsumed by
Zipperposition, which uses an optimized, incomplete higher-order subsumption
algorithm.

Given a disequation $\cst{f}\,\tuple{s}{n} \not\eq \cst{f}\,\tuple{t}{n}$, the
\emph{abstraction} of $s_i$ is $\lambda x.\, u \ieq v$, where $u$ is obtained by
replacing all occurrences of $s_i$ in $\cst{f}\,\tuple{s}{n}$ with $x$ and $v$ is
obtained by replacing all occurrences of $s_i$ in
$\cst{f}\,\tuple{t}{n}$ with $x$. For \confrep{}{an equation }$\cst{f}\,\tuple{s}{n} \eq
\cst{f}\,\tuple{t}{n}$, the analogous abstraction is $\lambda x.\, \inot (u \ieq
v)$.
%
%
Adding abstractions of the literals occurring in the conjecture to $\mathit{Inst}$ can
provide useful instantiations for formulas such as induction principles of
datatypes. As the conjecture is negated\confrep{}{ in refutational theorem proving},
the equation's polarity is inverted in the
abstraction. 

\begin{examplex}
\label{ex:dat056-2}
The clausified conjecture of the problem \texttt{DAT056\^{}2}
\cite{ns-13-leo2sh} from the TPTP library is $\cst{ap} \, \cst{xs}
\, (\cst{ap} \, \cst{ys} \, \cst{zs}) \not\eq \cst{ap} \, (\cst{ap} \, \cst{xs} \,
\cst{ys}) \, \cst{zs}$, where $\cst{ap}$ is the \confrep{}{list }append operator defined
recursively on its first argument and $\cst{xs}$, $\cst{ys}$, and $\cst{zs}$ are
of list type. Abstracting $\cst{xs}$ from the disequation yields $t = \lambda \mathit{xs}.\, \cst{ap} \, \mathit{xs} \, (\cst{ap} \, \cst{ys} \, \cst{zs})
\allowbreak\ieq\allowbreak \cst{ap} \, (\cst{ap} \, \mathit{xs} \, \cst{ys}) \, \cst{zs}$, which is added
to $\mathit{Inst}$.
Included in the problem is the induction axiom
for the list datatype: $\iforall p. \, p \, \cst{nil} \iand (\iforall x \,
\mathit{xs}. \, p \, \mathit{xs} \iimplies\allowbreak p \, (\cst{cons} \, x \,
\mathit{xs})) \iimplies\allowbreak \iforall \mathit{xs}. \, p \, \mathit{xs}$, where
$\cst{nil}$ and $\cst{cons}$ have the usual meanings.
Instantiating $p$ with $t$ and
using the $\cst{ap}$ definition, we can prove
$\iforall \mathit{xs}. \, \cst{ap} \, \mathit{xs} \, (\cst{ap} \, \cst{ys} \, \cst{zs})
\ieq \cst{ap} \, (\cst{ap} \, \mathit{xs} \, \cst{ys}) \, \cst{zs}$,
from which we easily derive a contradiction.
\end{examplex}


\ourpara{Evaluation and Discussion}

\begin{figure}
\centering
  \def\arraystretch{1.1}%
  \begin{tabular}{@{}l@{\hskip 1.5em}l@{\hskip 0.5em}@{\hskip 1em}c@{\hskip 1em}c@{}} \toprule
    & & $+$\relax{LA} & $-$\relax{LA} \\ \midrule
    TPTP &
%    \parbox[t]{2mm}{\multirow{3}{*}{\rotatebox[origin=c]{90}{TPTP}}}&
      \relax{IC}  & 1616  & 1635$^\star$    \\
    & \relax{DCI} & 1507  & 1532\phantom{$^\star$}    \\
    & \relax{DCS} & 1668  & {\bf 1703}\phantom{$^\star$} \\ \midrule
    SH &
%    \parbox[t]{2mm}{\multirow{3}{*}{\rotatebox[origin=c]{90}{SH}}} &
      \relax{IC}  & \colalign425 & \colalign452$^\star$    \\
    & \relax{DCI} & \colalign362 & \colalign385\phantom{$^\star$}    \\
    & \relax{DCS} & \colalign441 & \colalign\textbf{457}\phantom{$^\star$} \\ \bottomrule
  \end{tabular}
  \caption{Impact of clausification and~lightweight AVATAR}
  \label{fig:avatar-clause}
\end{figure}

The base configuration (\emph{base}) uses immediate clausification (\relax{IC}) and 
disables lightweight AVATAR ($-$\relax{LA}). To test
the merits of delayed clausification, we vary \emph{base}'s parameters along two axes: We
choose immediate clausification (\relax{IC}), delayed clausification as inference
(\relax{DCI}), or delayed clausification as simplification (\relax{DCS}), and we
either enable ($+$\relax{LA}) or disable ($-$\relax{LA}) lightweight AVATAR.
Neither of the configurations uses instantiation with terms from $\instset$.

Figure~\ref{fig:avatar-clause} shows that using delayed clausification as
simplification greatly increases the success rate, regardless of whether
lightweight AVATAR is used. Using delayed clausification as inference has the
opposite effect on both problem sets, presumably due to the large number of
clauses it creates. By manually inspecting the  proofs found by the \relax{DCS}
configuration, we noticed that a main reason for its success is that it does
not simplify away equivalences.
%
Overall, the lightweight AVATAR harms performance, but the sets of
problems proved with and without it are vastly different. For example,
the \relax{IC}$+$\relax{LA} configuration proves \NumberOK{38} problems not
proved by \relax{IC}$-$\relax{LA} (i.e., \emph{base}) on TPTP benchmarks and
\NumberOK{14} such problems on SH benchmarks.

\looseness=-1
The Boolean instantiation technique presented above requires delayed
clausification.
We assessed it in the best configuration from
%To assess it, we enabled it in the best configuration from
Fig.~\ref{fig:avatar-clause}, \relax{DCS}$-$\relax{LA}. With this change ($+$\relax{BI}),
Zipperposition proves \NumberOK{1700} TPTP problems and \NumberOK{456} SH
problems.
On TPTP, even though $+$\relax{BI} solves \NumberOK{3} problems less
than \relax{DCS}$-$LA, it is very useful: \NumberOK{41}
problems can be proved with $+$\relax{BI} but not with \relax{DCS}$-$\relax{LA}. 
Conversely, \NumberOK{44} problems are solved with \relax{DCS}$-$\relax{LA},
but not with $+$\relax{BI}, which suggests that Boolean instantiation can be 
explosive.
%
One of the problems Boolean instantiation helps solve is \texttt{NUM636\^{}2} (a
re-encoding of \texttt{NUM636\^{}3}).
It conjectures that $\iforall x.\, \cst{s} \, x \ineq x$, where $x$ ranges over
Peano-style numbers specified by $\cst{z}$ and $\cst{s}$. The given axioms are
the induction principle $\forall p.\, p \, \cst{z} \,\iand\, \forall x. \, (p \,
x \iimplies p \, (\cst{s} \, x)) \iimplies \forall x. \, p\, x$, injectivity
$\forall x y. \, \cst{s}\,x \ieq \cst{s}\,y \iimplies x \ieq y$, and
distinctness $\forall x. \, \cst{s}\,x \ineq \cst{z}$. The conjecture is easily
proved if Boolean instantiation is enabled: Even though the conjecture literal
cannot be abstracted, instantiating $p$ with the term $\lambda x.\, \cst{s}
\, x \ineq x$ used in the encoding of the (nonclausified) conjecture leads to a proof in just
\NumberOK{22} given clause loop iterations. Zipperposition also finds a
proof using the \relax{DCI}$-$\relax{LA} configuration, but this requires
\NumberOK{294} iterations.

%Raw data reveals that Boolean instantiation
The $+$\relax{BI} configuration proves \NumberOK{18} TPTP problems no other
configuration from Fig.~\ref{fig:avatar-clause} can prove. Among these is
\texttt{DAT056\^{}2} (Example~\ref{ex:dat056-2}). In contrast, on SH benchmarks, only
\NumberOK{6} problems are proved using $+$\relax{BI} and not
\relax{DCS}$-$\relax{LA}. For all these problems, Boolean instantiation does not
appear in the proof, suggesting that this result is due to the randomness in the
evaluation environment. The fact that \relax{BI} has no effect on SH benchmarks
is to be expected because Sledgehammer does
not include lemmas whose name contains the substring \texttt{.induct} and that
contain predicate variables. Therefore, \relax{BI} applies to fewer clauses.

\section{Exploring Boolean Selection Functions}
\label{sec:ho-tech:bool-select}

Superposition calculi are parameterized by a literal selection function and a
term order that help prune considerable swaths of the search space without
jeopardizing completeness. The core inferences apply only to a clause's
\emph{eligible} literals, defined as either the clause's selected
literals or, if none are selected, the clause's literals that are maximal with
respect to the term order. To further restrict which terms can be targeted by
an inference, the \osup{} calculus introduces \emph{Boolean selection
functions}. 

A Boolean selection function chooses \emph{green subterms} of Boolean type
(different than $\top$ or $\bot$ and not occurring at either side of a positive
literal) in a clause and gives rise to a notion of eligibility that considers
the formula structure. Green subterms correspond to the first-order skeleton of
a higher-order term; that is, they do not occur in positions under applied
variables, quantifiers, or $\lambda$-abstractions.

%In particular, green
%subterms are subterms arising inside first-order-like contexts, and for first
%other problems, all subterms are green.


\begin{definitionx}[Green subterms and green positions]
  \,Green subterms and green positions are defined inductively as
  follows: $t$ is a green subterm of $t$ at green position $\varepsilon$; if $t$ is a green subterm of $u_i$ at green position $p$
  and $\cst{f}$ is a constant different from $\iforall$ and $\iexists$, then
  $t$ is a green subterm of $\cst{f} \, \tuplen{u}$ at green position $i.p$,
  assuming $i \leq n$.
\end{definitionx}

\begin{examplex} The green subterms of the term $F \, \cst{a} \, \iand \, \cst{p} \,
(\iforall \, (\lambda x. \, \cst{q} \, x)) \, \cst{b}$ are the term itself, $F \, \cst{a}$, $\cst{p} \, (\iforall \, (\lambda x. \, \cst{q} \, x))
\, \cst{b}$, $\iforall \, (\lambda x. \, \cst{q} \, x)$, and $\cst{b}$.
\end{examplex} 
%
Green positions are lifted to clauses as follows: If $p$ is the green position
of a subterm in $s$, and $s$ occurs in a literal $l \in \{s \eq t{,}\; s \not\eq
t\}$ of $C$, the green position of the same subterm in the clause is denoted by $l.s.p$.
\osup{} mandates additional restrictions on the Boolean selection function:
$\itrue$, $\ifalse$ and variable-headed terms cannot be selected; for literals
$s \eq t$, neither $s$ not $t$ cannot be selected; if a term $s$ contains a
variable $X$ as a green subterm, and $X\> \tuplen{u}$, with $n \ge 1$, is a
maximal term of the clause, $s$ cannot be selected.

\begin{definitionx}[Eligibility]
  \,Given a substitution $\sigma$ and term order $\succ$, we say a literal $l$
  is (strictly) eligible with respect to $\sigma$ in $C$ if it is selected in
  $C$ or there are no selected literals and no selected Boolean subterms in
  $C$ and $l\sigma$ is (strictly) maximal in $C\sigma$ with respect to the
  term order.
%
  The eligible subterms of a clause $C$ with respect to a substitution
  $\sigma$ are inductively defined as follows:
  Any subterm selected by the Boolean selection function is eligible.
  %Any selected subterm is eligible.
  For a strictly eligible literal $s \eq t$ with $t\sigma \not\succ
  s\sigma$, $s$ is eligible. For an eligible literal $s \not\eq t$ with
  $t\sigma \not\succ s\sigma$, $s$ is eligible. If a subterm $t$ is eligible
  and the head of $t$ is not $\ieq$ or $\ineq$, all direct green subterms of
  $t$ are eligible. If a subterm $t$ is eligible and $t$ is of the form $u
  \ieq v$ or $u \ineq v$, then $u$ is eligible if $v\sigma \not\succ u\sigma$
  and $v$ is eligible if $u\sigma \not\succ v\sigma$.
\end{definitionx}

The above definitions of green subterms and eligibility were originally introduced
with Boolean $\lambda$-superposition \cite{bbtv-21-full-ho-sup}.
The Boolean selection function plays a similar role as the literal
selection function in standard superposition.
Literal selection functions eliminate some of the nondeterminism present in the superposition
calculus by focusing on selected parts of the search space. Boolean selection functions achieve
the same goal, but in a different context: They eliminate nondeterminism that is not
present in standard superposition, namely, the choice of subformula on which 
the Boolean calculus rules are to be applied. As with literal selection functions,
selecting few (and smaller) subterms can give rise to fewer possible inferences
and reduce clause proliferation.

%
This notion of eligibility opens up possibilities for reasoning with
formulas that are hard to simulate with the existing superposition machinery.
For example, given a formula $\varphi \iimplies \psi$, selecting the antecedent
simulates forward reasoning, whereas selecting the consequent simulates backward
reasoning. The new eligibility also makes it possible to restrict the proof
search to a small, promising part of a formula. Note that
literal selection can override Boolean selection: Selecting a literal 
might make some of its green subterms eligible, regardless
of Boolean selection.

In our previous work \cite{nbtv-2021-foboolsup}, we left this area of new
possibilities largely unexplored. We designed simple functions that selected
smallest, largest, innermost, or outermost terms, but they did not impact
performance much. Here, we propose alternatives.
%
Intuitively, a well-performing literal selection
function might succeed at taming the combinatorial explosion if the
selected literal can take part in few inferences
\cite{hrsv-16-selsel}. However, Boolean selection functions
introduce another factor to consider:\ the context in which the selected
subterm occurs. This suggests the following definition:

\begin{definitionx}[Contextualized Boolean selection function]
  \label{def:context-bool-sel}
  \,Let $\mathit{ctx}(C)$ be a function that maps a clause $C$ to a set of green positions
  $p$ such that $C|_p$ is a selectable Boolean subterm, and let
  $\vartriangleright$ be a partial order on pairs of terms and green positions.
  The \emph{context
  Boolean selection function} $\mathit{Sel\/}_{\mathit{ctx}}^{\,\vartriangleright}(C)$
  selects all terms $t$ such that $t=C|_p$, $p \in \mathit{ctx}(C)$, and
  $(t,p)$ is maximal with respect to $\vartriangleright$.
\end{definitionx}

In the above definition, the function $\mathit{ctx}$ lets
us choose the context in which the Boolean subterm appears. Then, among
the terms in the chosen context, we choose the ones that are maximal with
respect to $\vartriangleright$.

Ganzinger and Stuber considered Boolean subterm selection for their extension of
first-order superposition with interpreted Boolean type \cite{gs-05-boolsup}. Unlike our calculus, their calculus requires
selection of subterms occurring in negative green positions, defined below.

\begin{definitionx}[Polarity of green positions]
 \,Negative and positive green positions in a clause $C = l_1 \llor \cdots \llor
 l_n$ are defined inductively as follows: For each
 $1 \leq i \leq n$, the green position $l_i.s$ is positive if $l_i = \poslit{s}$ and negative if $l_i
 = \neglit{s}$. If $p$ is positive (negative) and $C|_p = s \,
 \tuplen{t}$ where $s$ is either $\iand$ or $\ior$, then each
 $p.i, 1 \leq i \leq n$, is positive (negative). If $p$ is positive and $C|_p =
 \inot \, s$, then $p.1$ is negative; if $p$ is negative and $C|_p = \inot \, s$,
 then $p.1$ is positive. If $p$ is positive and $C|_p = s \iimplies t$, then
 $p.1$ is negative and $p.2$ is positive; if $p$ is negative and $C|_p = s
 \iimplies t$, then $p.1$ is positive and $p.2$ is negative.
\end{definitionx}
Note that the polarity of $p$ is undefined whenever $C|_p$ is not a green Boolean subterm
or it occurs under a (dis)equivalence or an uninterpreted symbol.
To assess how the function $\mathit{ctx}$ affects performance, we use the following
selection functions that consider green positions of selectable Boolean terms:
%
\begin{enumerate}[\rm\texttt{Backward}~]
  \item[\rm\texttt{Any}~] select all green positions;
  \item[\rm\texttt{Pos}~] select all positive green positions;
  \item[\rm\texttt{Neg}~] select all negative green positions;
  \item[\rm\texttt{Forward}~] select all green positions $p = q.1$ such that $C|_q = s \iimplies t$;
  \item[\rm\texttt{Backward}~] select all green positions $p = q.2$ such that $C|_q = s \iimplies t$;
  \item[\rm\texttt{Deep}~] select all green positions of maximal length;
  \item[\rm\texttt{Shallow}~] select all green positions of minimal length.
\end{enumerate}

\looseness=-1
We also introduce three partial orders for selecting subterms from a given
context. For all three orders, if exactly one of the subterms has a logical head,
then the subterm with the nonlogical head is larger, because logical symbols
are more explosive. Otherwise, the orders use the following criteria:
%
\begin{enumerate}[$\vartriangleright_\text{ground}$~]
  \item[$\vartriangleright_\text{ground}$~] If exactly one of the subterms is ground, make the
  ground subterm larger; otherwise, if exactly one of the subterms is of the form $s \ieq
  t$, make this subterm larger.

\smallskip

  \item[$\vartriangleright_\text{depth}$~] If one of the subterms has larger subterm depth
  (longest valid green subterm position), make this subterm larger; otherwise, if one of the
  subterms has less distinct variables, make this subterm larger.

\smallskip

  \item[$\vartriangleright_\text{def}$~] If exactly one of the subterms is of the form
  $\cst{p}\,\tuplen{X}$ where $\tuplen{X}$ is a tuple of free variables, make
  the other subterm larger; otherwise, if exactly one of the subterms is of the form $X \,
  \tuplen{s}$, make the other subterm larger.
\end{enumerate}

In case of a tie, the subterm with the smaller syntactic weight is made larger,
and if both subterms have the same weight, the term that occurs in a position further
to the left (i.e., that has a lexicographically smaller position) is made larger.

These orders follow the design principle enunciated by Hoder et
al.~\cite{hrsv-16-selsel} that ground or deep terms and terms with repeated
variables are ``less unifiable'' with the similar observation for higher-order
logic that reasoning about interpreted symbols or applied variables is usually
explosive.

\begin{examplex}
  Selecting the right Boolean subterm can help avoid elaborating
  high\-er-order inferences. Consider the unsatisfiable clause set consisting of $\cst{p} \, (\lambda y.\, X
  \, (\lambda x.\, x) \, \cst{a}) \iimplies\allowbreak \inot(\cst{p} \, (\lambda y. \allowbreak\, X
  \, y \, \cst{a}))$, $\cst{p} \, (\lambda y.\, \cst{a})$, and
  $\cst{p} \, (\lambda y. \, y^{100} \, \cst{b})$.
  Note that $\cst{p} \, (\lambda y.\, X
  \, (\lambda x.\, x) \, \cst{a})$ and $\cst{p} \,
  (\lambda y.\, \cst{a})$ have infinitely many unifiers of the form $\{ X \mapsto \lambda fx.
  \, f^i\, (x) \}, i \geq 0$, whereas $\cst{p} \, (\lambda y. \, X
  \, y \, \cst{a})$ and $\cst{p} \, (\lambda y. y^{100} \, \cst{b})$ have only one unifier. 
  If \texttt{Forward} context selection is enabled,
  $\cst{p} \, (\lambda y.\, X \, (\lambda x.\, x) \, \cst{a})$ is made the target of superposition inference, 
  forcing computation of at least 100 unifiers (under the assumption that
  unifiers are returned in order of increasing $i$) before we get to refute
  $\inot(\cst{p} \, (\lambda y. y^{100} \, \cst{b}))$.
  In contrast, \texttt{Backward} context selection allows us to
  superpose from $\cst{p} \, (\lambda y. \, y^{100} \, \cst{b})$ into $\cst{p} \, (\lambda y. \, X
  \, y \, \cst{a})$, avoiding this explosion. 
\end{examplex}

\ourpara{Evaluation and Discussion}

\begin{figure}
\centering
\def\arraystretch{1.1}%
 \begin{tabular}{@{}l@{\hskip 1.5em}l@{\hskip 1em}c@{\hskip 1em}c@{\hskip 1em}c@{\hskip 1em}c@{\hskip 1em}c@{\hskip 1em}c@{\hskip 1em}c@{}} \toprule
  &                                    & \texttt{Any} & \texttt{Pos} & \texttt{Neg}        & \texttt{Forward} & \texttt{Backward} & \texttt{Deep}     & \texttt{Shallow}  \\ \midrule
  TPTP &
    $\vartriangleright_\text{ground}$  & 1538         & 1550         & 1547                & 1534             & {\bf 1554}        & 1539              & 1538     \\
  & $\vartriangleright_\text{depth}$   & 1542         & 1550         & 1528                & 1542             & 1550              & 1547              & 1535     \\
  & $\vartriangleright_\text{def}$     & 1543         & 1551         & 1540                & 1540             & 1551              & 1545              & 1537  \\ \midrule

%  \parbox[t]{2mm}{\multirow{3}{*}{\rotatebox[origin=c]{90}{SH}}} &
  SH &
    $\vartriangleright_\text{ground}$  & \colalign386 & \colalign379  & \colalign386       & \colalign386     & \colalign379      & \colalign{\bf387} & \colalign{\bf387}     \\
  & $\vartriangleright_\text{depth}$   & \colalign377 & \colalign376  & \colalign384       & \colalign378     & \colalign376      & \colalign379      & \colalign376      \\
  & $\vartriangleright_\text{def}$     & \colalign379 & \colalign374  & \colalign{\bf387}  & \colalign379     & \colalign380      & \colalign377      & \colalign381  \\ \bottomrule
 \end{tabular}
 \caption{Impact of the Boolean selection function}
 \label{fig:bool-sel}
\end{figure}

When the input problem is clausified using immediate clausification, almost
all Boolean structure is lost. In this case, we expect Boolean selection to
have a modest effect. To better assess this feature, in this evaluation we use
\relax{DCI}$-$\relax{LA} from Sect.~\ref{sec:ho-tech:formulas} as the baseline
configuration. To avoid interference of literal and Boolean selection, we
additionally forbid the literal selection function from selecting a literal if
it contains a selectable Boolean subterm.

The results of evaluating 21 concrete selection functions obtained
by instantiating the contextualized Boolean selection function are shown in
Fig.~\ref{fig:bool-sel}. Rows denote the used partial order $\vartriangleright$,
while columns denote the function $\mathit{ctx}$.

\looseness=-1
On TPTP benchmarks, Boolean selection helps tame the explosion caused by
dynamic clausification used as inference: \NumberOK{All but one} selection functions
outperform the \relax{DCI}$-$\relax{LA} baseline of \NumberOK{1532} proved problems. Coming back to the problem
\texttt{NUM636\^{}2} from Sect.~\ref{sec:ho-tech:formulas}, using Boolean selection
can reduce the number of given clause loop iterations from
\NumberOK{294} to \NumberOK{71}.

The results suggest that selection of term context has more impact
than the partial term order. % The number of proved problems fluctuates more when
%changing the function $\mathit{ctx}$ than when changing the partial order.
Also, the best results are obtained when a context more specific than \texttt{Any}
is chosen. Remarkably, functions using \texttt{Pos} context
perform better than the ones using \texttt{Neg} context
on TPTP but the opposite is observed on SH.


Using different Boolean selection functions yields vastly different sets of
proved problems on TPTP benchmarks: In total, there are \NumberOK{103} problems
proved by some configuration from Fig.~\ref{fig:bool-sel} but not by
\relax{DCI}$-$\relax{LA}. However, there are only \NumberOK{16} such SH problems.
It would seem that the advanced formula reasoning facilitated by the Boolean
selection formulas is usually not required by Sledgehammer problems.

% {\tt
%   \begin{itemize}
%      \item To evaluate bool selection functions we used DCI from above, but
%            disabled literal selection function from selecting a literal if it
%            has a selectable Boolean subterm.
%      \item For TPTP it seems that selection is a good idea: all configurations
%      are better than naked DCI (only 6 for SH).
%      \item Overall, it seems that position has bigger influence that the
%      selection function.
%      \item On SH there are 14 problems provable by some of the Boolean
%            selection functions but not naked CSI. Sel3\_deep proves 12 problems
%            not proved by DCI. Interestingly, the best function Sel3\_neg proves
%            only 7 problems not proved by DCI.
%      \item On TPTP much better: 110 problems not proved by DCI but proved by
%      some of the selection functions. sel3\_pos proves 74 problems not proved
%      by DCI, sel1 backward 73.
%      \item TODO: Add any context and maybe $\vartriangleright_4$ which uses weight only
%      \item What we also observed is that it on TPTP problem it is reasonable to have any selection
%            function and that already stops the explosion. On SH quality is more important.
%      \item
%   \end{itemize}
% }

\section{Enumerating Infinitely Branching Inferences}
\label{sec:ho-tech:infinite-branching}

\looseness=-1
As an optimization and to simplify the code, Leo-III
\cite{as-18-phd} and Vampire 4.4 \cite{br-19-restricted-unif} (which uses
\confrep{}{\emph{restricted combinatory unification}, }a predecessor of combinatory
superposition) compute only a finite subset of the possible conclusions for
inferences that require enumerating a CSU. Not only is this a source of
incompleteness, but choosing the cardinality of the computed subset is a
difficult heuristic choice. Small sets can result in missing the unifier
necessary for the proof, whereas large sets make the prover spend too long in
the unification procedure, generate useless clauses, and possibly get sidetracked into the wrong parts of the search space.

We propose a modification to the given clause procedure to seamlessly
interleave unifier computation and proof state exploration. Given a complete
unification procedure, which may yield infinite streams of unifiers, our
modification fairly enumerates all conclusions of inferences relying on
elements of a CSU. Under some reasonable assumptions, it behaves exactly like
the standard given clause procedure on purely first-order problems.
We also describe heuristics that help achieve a similar
performance as when using incomplete, terminating unification procedures
without sacrificing completeness.

Given that we cannot decide whether there exists a next CSU
element in a stream of unifiers, the request for the next conclusion might not
terminate,
effectively bringing the theorem prover to a halt. Our modified given clause procedure
expects the unification procedure to return a lazily computed stream
\cite[Sect.~4.2]{co-1999-funds}, where each element is either $\emptyset$ or
a singleton set containing a unifier. To avoid getting stuck waiting for a unifier
that may not exist, the unification procedure should return
$\emptyset$ after it performs some number of operations without finding a unifier.

The complete unification procedure by Vukmirović et al.\
\cite{our-unif-paper} returns such a stream. Other procedures such
as Huet's \cite{gh-75-unification} and Jensen and
Pietrzykowski's \cite{jp-76-unif} can easily be adapted to meet this
requirement. Based on the stream of unifiers interspersed with $\emptyset$, we
can construct a stream of inferences similarly interspersed with $\emptyset$.
Any finite prefixes of this stream can be computed in finite time.

To support such streams in the given clause procedure, we extend it to
represent the proof state not only by the active ($A$) and passive ($P$) clause
sets, but also by a priority queue $Q$ containing the inference streams,
similar to the ``to do'' set $T$ present in the abstract Zipperposition loop of 
Waldmann et al.\  \cite[Sect.~4]{wtrb-20-sat-framework}.
Each stream is associated with a weight, and $Q$ is sorted in order of
increasing weight\confrep{}{, a low weight corresponding to a high priority}.
When they introduced \lsup, Bentkamp et al.\
\cite{bbtvw-21-sup-lam} described an older version of this extension. Here
we present a newer version in more detail, including heuristics to postpone
unpromising streams. The pseudocode of the modified procedure is as follows:
\vspace{2.5\jot}
%
\newcommand\ParamMode{\ensuremath{K_{\mathrm{fair}}}}
\newcommand\ParamMaxStreams{\ensuremath{K_{\mathrm{best}}}}
\newcommand\ParamRetry{\ensuremath{K_{\mathrm{retry}}}}
\newcommand{\assign}[2]{\State \ensuremath{\mathit{#1} \gets #2}}
\newcommand{\assignSameLine}[2]{\ensuremath{\mathit{#1} \gets #2}}
\algrenewcommand\algorithmicindent{1em}
\begin{algorithmic}[0]
    \Function{ExtractClause}{$Q$, $\mathit{stream}$}
      \assign{maybe\_clause}{\text{pop and compute the first element of } \mathit{stream} }
      \If{$\mathit{stream}$ is not empty}
        \State add $\mathit{stream}$ to $Q$ with an increased weight
      \EndIf
      \State \Return $\mathit{maybe\_clause}$
    \EndFunction

    \vspace{2\jot}

    \Function{HeuristicProbe}{$Q$}
      \assign{i}{0}
      \assign{collected\_clauses}{\emptyset}
      \While{$i < \ParamMaxStreams$ and $Q \not= \emptyset$}
        \assign{j}{0}
        \assign{maybe\_clause}{\emptyset}
        \While{$j < \ParamRetry$ and $Q \not= \emptyset$ and $\mathit{maybe\_clause} = \emptyset$}
          \assign{stream}{\text{pop the lowest-weight stream in } Q}
          \assign{maybe\_clause}{\textsc{ExtractClause}(Q, \mathit{stream})}
          \assign{j}{j+1}
        \EndWhile
        \assign{collected\_clauses}{\mathit{collected\_clauses} \mathrel\cup \mathit{maybe\_clause}}
        \assign{i}{i+1}
      \EndWhile
      \State \Return $\mathit{collected\_clauses}$
    \EndFunction

    \vspace{2\jot}

  \Function{FairProbe}{$Q$, $\mathit{num\_oldest}$}
    \assign{collected\_clauses}{\emptyset}
    \assign{oldest\_streams}{\text{pop } \mathit{num\_oldest} \text{ oldest streams from } Q}
    \For{$\mathit{stream}$ in $\mathit{oldest\_streams}$}
      \assign{collected\_clauses}{collected\_clauses \mathrel\cup \textsc{ExtractClause}(Q, \mathit{stream})}
    \EndFor
    \State \Return $\mathit{collected\_clauses}$
  \EndFunction

  \vspace{2\jot}

  \Function{ForceProbe}{$Q$}
    \assign{collected\_clauses}{\emptyset}
    \While {$Q \not= \emptyset$ and $\mathit{collected\_clauses} = \emptyset$}
      \assign{collected\_clauses}{\textsc{FairProbe}(Q, |Q|)}
    \EndWhile

    \If{$Q = \mathit{collected\_clauses} = \emptyset$}
      \assign{status}{\textsf{Satisfiable}}
      % \Return (\textsf{Satisfiable}, $\mathit{collected\_clauses}$)
    \Else{}
      \assign{status}{\textsf{Unknown}}
      % \Return (\textsf{Unknown}, $\mathit{collected\_clauses}$)
    \EndIf

    \State \Return $(\mathit{status}, \mathit{collected\_clauses})$
  \EndFunction

  \vspace{2\jot}

  \Function{GivenClause}{$P$, $A$, $Q$}

  \assign{i}{0}
  \assign{status}{\textsf{Unknown}}
  \While{$\mathit{status} = \textsf{Unknown}$}
    \If{$P = \emptyset$}
      \assign{(status, \,forced\_clauses)}{\textsc{ForceProbe}(Q)}
      \assign{P}{P \mathrel\cup \mathit{forced\_clauses}}
    \Else
      \assign{given}{\text{pop a chosen clause from }P \text{ and simplify it}}
      \If{$\mathit{given}$ is the empty clause}
        \assign{status}{\textsf{Unsatisfiable}}
      \Else
        \assign{A}{A \mathrel\cup \{\mathit{given}\}}
        \For{$\mathit{stream}$ in streams of inferences between $\mathit{given}$ and $\mathit{other} \in A$}
          \If{$\mathit{stream}$ is not empty} \assign{P}{P \mathrel\cup \textsc{ExtractClause}($Q$, \mathit{stream})}
          \EndIf
        \EndFor
        \assign{i}{i+1}
        \If{$i \; \mathrm{mod} \; \ParamMode = 0$} \assign{P}{P \mathrel\cup \textsc{FairProbe}(Q, i/\ParamMode )}
        \Else{} \assign{P}{P \mathrel\cup \textsc{HeuristicProbe}(Q)}
        \EndIf
      \EndIf
    \EndIf
  \EndWhile

  \State \Return $\mathit{status}$

  \EndFunction
\end{algorithmic}
% \caption{Our modified given clause procedure}
% \label{fig:gc-pseudocode}
%
% \end{figure}
\vspace{2.5\jot}

Initially, all input clauses are put into $P$, and $A$ and $Q$ are empty. Unlike
in the standard given clause procedure, inference results are represented as clause
streams. The first element is inserted into $P$, and the rest of the stream is
stored in $Q$ with some positive integer weight computed from the inference rule.

% The functions \textsc{Probe} and \textsc{ForceProbe} extract some conclusions
% from the inference streams and store them in $P$. \textsc{Probe} heuristically
% chooses some clauses, whereas \textsc{ForceProbe} is used when $P$ is empty to
% find one clause if previous probes failed to produce any clauses, as a fallback.
% By choosing clauses from several streams at the same time,
% we give more freedom to the prover's existing clause selection heuristics,
% which select a clause from $P$.

% \looseness=-1
% \textsc{Probe} has two modes of operation, controlled by a parameter \ParamMode~(by
% default, $\ParamMode = 70$). In every $\ParamMode$th invocation, \textsc{Probe} extracts
% conclusions from an increasing number of oldest streams. This amounts to
% dovetailing, which achieves fairness, analogously to the
% pick--given ratio \cite{ss-02-brainiac,mcw-1997-otter} in the
% given clause procedure. For the other
% invocations, \textsc{Probe} selects the $\ParamMaxStreams$ highest-priority streams (by
% default, $\ParamMaxStreams = 10$) and extracts elements from, or \emph{probes}, each stream,
% discarding $\emptyset$ values, until a clause is found or until
% $\ParamRetry$ (by default, $\ParamRetry=20$) $\emptyset$ values have been discarded.
% Setting $\ParamRetry>1$ ensures that promising streams are given a fair chance to
% produce a clause, even if they rely on a complicated unifier. Each
% probed stream~$S$ is put back in $Q$ with a new priority.%

\looseness=-1
To eventually consider inference conclusions from streams in $Q$ as given
clauses, we extract elements from, or \emph{probe}, streams and move any obtained
clauses to $P$. Analogously to the traditional pick--given ratio
\cite{ss-02-brainiac,mcw-1997-otter}, we use a parameter
\ParamMode{} (by default, $\ParamMode = 70$) to ensure fairness: Every \ParamMode{}th iteration,
\textsc{FairProbe} probes an increasing number of oldest streams, which achieves
dovetailing. In all other iterations, \textsc{HeuristicProbe} attempts to
extract up to \ParamMaxStreams{}~clauses from the most promising streams (by default,
$\ParamMaxStreams = 7$).
In each attempt, the most promising stream in $Q$ is chosen. If its first
element is $\emptyset$, the rest of the stream is inserted into $Q$ and a new stream is
chosen. This is repeated until either \ParamRetry{} occurrences of $\emptyset$ have  been
met (by default, $\ParamRetry = 20$) or the stream yields a singleton. Setting $\ParamRetry > 0$ increases
the chance that \textsc{HeuristicProbe} will return $\ParamMaxStreams$ clauses, as desired. Finally, if $P$ becomes empty, \textsc{ForceProbe}
searches relentlessly for a clause in $Q$, as a fallback.

\looseness=-1
The function \textsc{ExtractClause} extracts an element from a nonempty stream
not in $Q$ and inserts the remaining
stream into $Q$ with an increased weight, calculated as follows.
Let $n$ be the number of times the stream was chosen for
probing. If probing results in $\emptyset$, the stream's weight is increased by
$\max\,\{2{,}\; n-16\}$. If probing results in a clause $C$ whose penalty is
$p$, the stream's weight is increased by $p \cdot \max\,\{1{,}\; n-64\}$. The
penalty of a clause is a number assigned by Zipperposition based on
features such as the depth of its derivation and the rules used in it.
The constants $16$ and $64$ increase the chance that newer clause-producing streams are picked,
which is desirable because their first clauses are expected to be useful.

\looseness=-1
All three probing functions are invoked by
\textsc{GivenClause}, which contains the saturation loop. It differs
from the standard given clause procedure in three ways:
First, the proof state includes $Q$ in addition to $P$ and $A$. Second,
new inferences involving the given clause are added to $Q$ instead of being
performed immediately. Third, inferences in $Q$ are periodically performed
lazily to fill $P$.

\newcommand\infstream[1]{[#1]}

\begin{examplex} 
  \begin{sloppypar}
  Consider the unsatisfiable two-clause problem $\{ X \, (\cst{f} \,
  \cst{a}) \not\eq \cst{f} \, (X \, \cst{a}) \lor \cst{p} \, (X \, \cst{a}),\allowbreak
  \neg \cst{p} \, (\cst{f}^{100} \, \cst{a})  \}$ and a selection function which
  selects negative literals.
  Let $P \mid A \mid Q$ denote
  the state of the given clause loop (i.e., the contents of the passive and active set
  and of the stream queue), and let $\infstream{ a_1, a_2, \ldots }$
  denote an infinite stream of elements.
  \end{sloppypar}

  The given clause loop begins in the state $X \, (\cst{f} \, \cst{a}) \not\eq
  \cst{f} \, (X \, \cst{a}) \lor \cst{p} \, (X \, \cst{a}), \neg \cst{p} \,
  (\cst{f}^{100} \, \cst{a}) \mid \emptyset \mid \emptyset$. If the clause $\neg
  \cst{p} \, (\cst{f}^{100} \, \cst{a})$ is chosen for processing, since $Q$ is empty
  and no inferences with the chosen clause are possible, the state becomes $X \,
  (\cst{f} \, \cst{a}) \not\eq \cst{f} \, (X \, \cst{a}) \lor \cst{p} \, (X \,
  \cst{a}) \mid \neg \cst{p} \, (\cst{f}^{100} \, \cst{a}) \mid \emptyset$. When
  the clause $X \, (\cst{f} \, \cst{a}) \not\eq \cst{f} \, (X \, \cst{a}) \lor
  \cst{p} \, (X \, \cst{a})$ is chosen, a new stream which enumerates results of
  equality resolution (on its first literal) is created. There are infinitely many
  conclusions of this inference, since there are infinitely many unifiers for the
  first literal of the form $\{ X \mapsto \lambda x. \, \cst{f}^i \, x \}$, for
  $i \geq 0$. Thus, the stream is $\infstream{ \{\cst{p} \, \cst{a}\}, \{\cst{p} \, (\cst{f}\,\cst{a})\}, \ldots }$,
  possibly with $\emptyset$s interspersed. With the standard given clause procedure,
  there would have been no way to represent this infinitary result.
  %The creation of the
  %inference result stream is the most striking difference between our
  %and standard given clause procedure.

  When the stream is created, its first element is popped and put into $P$. Then, based on the
  parameters that control inference stream probing, some number of clauses from
  the stream are computed and moved to $P$. After two iterations, the state might be
  $ \cst{p} \,
  \cst{a}, \cst{p} \, (\cst{f} \, \cst{a}), \cst{p} \, (\cst{f} \, (\cst{f} \, \cst{a}))   \mid X \,
  (\cst{f} \, \cst{a}) \not\eq \cst{f} \, (X \, \cst{a}) \lor \cst{p} \, (X \,
  \cst{a}), \neg \cst{p} \, (\cst{f}^{100} \, \cst{a}) \mid \infstream{ \{\cst{p} \, (\cst{f}^3\,\cst{a})\}, \ldots }$. 
  
  In the next iterations, some clause of the form $\cst{p} \, (\cst{f}^{i} \,
  \cst{a})$, where $i < 100$, is chosen, but no inferences with it can be
  performed. Then, the stream created in the second iteration is probed, and its
  results fill the set $P$. Ultimately, the clause $\cst{p} \, (\cst{f}^{100} \,
  \cst{a})$ is chosen, at which point $\bot$ is quickly derived.
\end{examplex}

\textsc{GivenClause} eagerly stores the first element of a new inference stream
in $P$ to imitate the standard given clause procedure. If the underlying
unification procedure behaves like the standard first-order unification
algorithm on higher-order logic's first-order fragment, our given clause
procedure coincides with the standard one. The unification procedure by
Vukmirović et al.\ terminates on the first-order and other fragments
\cite{tn-93-patterns}. To avoid computing
complicated unifiers eagerly, it immediately returns $\emptyset$ for a problem that does not
belong to one of the fragments that admit efficient unifier computation.


The design of our given clause procedure was guided by folklore knowledge about
higher-order theorem proving. First, in our experience most steps in
long higher-order proofs involve first-order literals. The unification
procedure and inference scheduling ensure that first-order inference
conclusions are put in the proof state as early as possible. Second, some
inference rules are expected to be largely useless. We initialize the stream
penalty differently for each rule, allowing old streams of more useful
inferences to be queried before newly added, but potentially less useful
streams. Finally, if we use a unification procedure that has aggressive
redundancy elimination, we will often find the necessary unifier within the
first few unifiers returned. Similarly, if a stream keeps returning
$\emptyset$, it is likely that it is blocked in a nonterminating computation
and should be ignored. Our heuristics to increase the stream penalties take
both observations into account.

\ourpara{Evaluation and Discussion}

When the unification procedure of Vukmirović et al.\ was implemented  in
Zipperposition, it was observed that this is the only competing
higher-order prover that proves all Church numeral problems from the TPTP,
never spending more than 5~s on a problem \cite{our-unif-paper}. On these hard
unification problems, the stream system allows the prover to explore the proof
state lazily.

Consider the TPTP problem \verb|NUM800^1|, which requires finding
a function $F$ such that $F \, \cst{c_1} \, \cst{c_2}
\ieq \cst{c_2} \iand F \, \cst{c_2} \, \cst{c_3} \ieq \cst{c_6}$, where
$\cst{c}_n$ abbreviates the Church numeral for~$n$, $\lambda s\, z. \>
s^n \, z$. To prove
\confrep{it}{the problem}, it suffices to take $F$ to be the multiplication operator
$\lambda x \, y \, s \, z. \> x \, (y \, s) \, z$.
However, this unifier is only one out of many available for each occurrence of
$F$.

% TODO: If I add a new benchmark set, make sure this sentence is fixed
In an independent evaluation setup on a set of 2606 TPTP version 7.2.0
problems almost identical to the one we use, Vukmirović et al.\
\cite[Sect.~7]{our-unif-paper}
compared a complete, nonterminating variant of the unification procedure and a
pragmatic, terminating variant. The
pragmatic variant was used directly---all the inference conclusions were put
immediately in $P$, bypassing $Q$. The complete variant, which relies on
possibly infinite streams and is much more prolific, proved only 15  problems
less than the most competitive pragmatic variant. Furthermore, it proved 19
problems not proved by the pragmatic variant.
%
This shows that our given clause procedure, with its heuristics, allows the
prover to defer exploring less promising branches of the unification and uses
the full power of a complete higher-order unifier search to solve unification
problems that cannot be proved by a restricted procedure.

The parameters \ParamMode{}, \ParamRetry{}, and \ParamMaxStreams{} can greatly
influence the behavior of the given clause procedure, even when the same
unification procedure is used. Figure~\ref{fig:streams}
presents the effects of these parameters on TPTP and SH.
Selecting a low number of best clauses seems to
perform well on both benchmark sets. However, on SH benchmarks, which require
overwhelmingly first-order unifiers, visiting older streams should be delayed
a lot.

As with Boolean selection functions, changing these three parameters causes
a substantial difference in the set of proved problems. For example, the
configuration that performs the worst on TPTP benchmarks proves \NumberOK{12}
problems that the configuration performing the best on TPTP cannot prove; moreover, there
are \NumberOK{29} TPTP problems that are proved by some set of parameters
other than $\ParamMode=\ParamMaxStreams=16, \ParamRetry=2$. On SH, these
effects are much weaker; most reasonable combinations
of parameters perform similarly.

Among the competing higher-order provers, only Satallax uses infinitely
branching calculus rules. It maintains a queue of ``commands'' that contain
instructions on how to create a successor state in the tableau. One
command describes infinite enumeration of all closed terms of a given function
type. Each execution of this command makes progress in the enumeration. Unlike
evaluation of streams representing elements of CSU, each command execution
is guaranteed to make progress in enumerating the next closed functional
term, so there is no need to ever return $\emptyset$.



\begin{figure}
\centering
\begin{subfigure}[b]{1\textwidth}
  \centering
  \begin{tabular}{@{}l@{\kern.5em}l@{\qquad}c@{\kern.75em}c@{\kern.75em}c@{}l@{}c@{\kern.75em}c@{\kern.75em}c@{}l@{}c@{\kern.75em}c@{\kern.75em}c@{}}\toprule
  &&&&&&& \ParamMode \\[.5\jot]
  & & & 2 & & \hbox{\qquad} & & 16 & & \hbox{\qquad} & & 128 & \\[.25\jot]
  \cline{3-5}\cline{7-9}\cline{11-13}
  \\[-1.5\jot]
  &&& \ParamRetry &&&& \ParamRetry &&&& \ParamRetry \\[.5\jot]
  %                                     2                        16                     256
  &                         & 2    & 16   & 128  & & 2         & 16   & 128  & & 2    & 16   & 128 \\\midrule
  & $2$                     & 1643 & 1645 & 1645 & & 1661      & 1661 & 1658 & & 1669 & 1664 & 1664 \\[0.5\jot]
  $\ParamMaxStreams$ & $16$ & 1647 & 1646 & 1609 & & {\bf1670} & 1654 & 1602 & & 1665 & 1659 & 1597 \\[0.5\jot]
  & $128$                   & 1646 & 1644 & 1583 & & 1661      & 1656 & 1577 & & 1665 & 1658 & 1576 \\ \bottomrule
  \end{tabular}
  \caption{TPTP benchmarks}
  \label{fig:streams-tptp}
\end{subfigure}
\par\bigskip
\begin{subfigure}[b]{1\textwidth}
  \centering
  \begin{tabular}{@{}l@{\kern.5em}l@{\qquad}c@{\kern.75em}c@{\kern.75em}c@{}l@{}c@{\kern.75em}c@{\kern.75em}c@{}l@{}c@{\kern.75em}c@{\kern.75em}c@{}}\toprule
  &&&&&&& \ParamMode \\[.5\jot]
  & & & 2 & & \hbox{\qquad} & & 16 & & \hbox{\qquad} & & 128 & \\[.25\jot]
  \cline{3-5}\cline{7-9}\cline{11-13}
  \\[-1.5\jot]
  &&& \ParamRetry &&&& \ParamRetry &&&& \ParamRetry \\[.5\jot]
  %                                          2                                               16                                      128
  &                         & 2            & 16            & 128          & & 2            & 16            & 128          & & 2                  & 16            & 128 \\\midrule
  & $2$                     & \colalign460 & \colalign455  & \colalign454 & & \colalign465 & \colalign463  & \colalign458 & & \colalign466       & \colalign461  & \colalign461 \\[0.5\jot]
  $\ParamMaxStreams$ & $16$ & \colalign458 & \colalign453  & \colalign445 & & \colalign464 & \colalign459  & \colalign441 & & \colalign{\bf468}  & \colalign459  & \colalign442 \\[0.5\jot]
  & $128$                   & \colalign456 & \colalign452  & \colalign430 & & \colalign465 & \colalign458  & \colalign428 & & \colalign{\bf468}  & \colalign459  & \colalign425 \\ \bottomrule
  \end{tabular}
  \caption{SH benchmarks}
  \label{fig:streams-sh}
\end{subfigure}
\caption{Impact of the stream enumeration parameter}
\label{fig:streams}
\end{figure}

\section{Controlling Prolific Rules}
\label{sec:ho-tech:explosiveness}

To support higher-order features
such as function extensionality and quantification over functions,
many refutationally complete calculi employ highly prolific rules.
For example, \lsup{} includes a
\infname{FluidSup} rule \cite{bbtvw-21-sup-lam} that very often applies to two
clauses if one of them contains a term of the form $F \, \overline{s}_n$,
where $n > 0$.
We describe three mechanisms to keep rules like these under control.

% \begin{enumerate}
%   \item Penalize the streams of expensive inferences.
%   \item Defer selecting the resulting clauses for processing.
%   \item Restrict the applicability of the rules.
% \end{enumerate}

% Complete provers that implement \lsup{} %, such as Zipperposition,
% have two main heuristic choices: which inference stream to query for clauses
% and which clauses  from the passive set to choose for processing. The first two mechanisms
% control these two choice points. The third mechanism
% sacrifices completeness by applying the inference rules only to a subset of chosen clauses.

First, \emph{we limit applicability of the prolific rules}. In practice, it
often suffices to apply prolific higher-order rules only to initial or shallow
clauses---clauses with a shallow derivation depth. Thus, we added an option to
forbid the application of a rule if the derivation depth of any premise exceeds a
limit.

\newcommand\ParamPenaltyIncrease{\ensuremath{K_\mathrm{incr}}}
Second, \emph{we penalize the streams of expensive inferences}. The weight of
each stream is given an initial value based on characteristics of the inference
premises such as their derivation depth. For prolific rules such as
\infname{FluidSup}, we increment this value by a parameter \ParamPenaltyIncrease. Weights for
less prolific variants of this rule, such as \infname{DupSup} \cite{bbtvw-21-sup-lam}, are increased by a
fraction of $\ParamPenaltyIncrease$ (e.g., $\lfloor \ParamPenaltyIncrease/3 \rfloor$).

Third, \emph{we defer the selection of prolific clauses}. To select the given
clause, most saturating provers evaluate clauses according to some criteria and
choose the clause with the lowest evaluation. To make this choice efficient,
passive clauses are organized into a priority queue ordered by their
evaluations. Like E, Zipper\-position maintains multiple
queues, ordered by different evaluations, that are visited in a round-robin
fashion. It also uses E's two-layer evaluation functions, a variant of which
has recently been implemented in Vampire \cite{gs-20-clausesel}.
%
The two layers are \emph{clause priority} and \emph{clause weight}. Clauses
with higher priority are preferred, and the weight is used for tie-breaking.
Intuitively, the first layer crudely separates clauses into priority classes,
whereas the second one uses heuristic weights to prefer clauses within a
priority class. To control the selection of prolific clauses, we introduce new
clause priority functions that take into account features specific to
higher-order clauses.

The first new priority function, \verb|PreferHOSteps| (\verb|PHOS|), assigns a
higher priority if rules specific to higher-order superposition calculi
were used in the clause derivation. Since most of the other clause priority
functions tend to defer higher-order clauses, having a clause queue that prefers
them might be useful to find some proof more
efficiently. A simpler function, which prefers clauses containing
$\lambda$-abstractions, is \verb|PreferLambda| (\verb|PL|).

\verb|PreferHOSteps| separates clauses created using first-
and higher-order inference rules crudely. However, within higher-order
inference rules there are the ones which make clauses simpler and are thus more
preferable. An example of such a rule is
%
\[\namedinference{\infname{ArgCong}}{C \lor s \eq t}{C \lor s \, \tuplen{X}
\eq t \, \tuplen{X}}\]
%
where $s$ is of the type $\alpha_1 \rightarrow \cdots \rightarrow \alpha_k
\rightarrow \beta$, $\beta$ is a base type, $n \leq k$, free variables
$\tuplen{X}$ are fresh, and literal $s \eq t$ is strictly eligible. When $n =
k$, in most cases, the resulting clause has a first-order literal $s \, \tuplen{X}
\eq t \, \tuplen{X}$ in place of the literal $s \eq t$ of functional type, which usually makes the clause more useful. To
prefer clauses that are only mildly higher-order, we designed the function
\verb|PreferEasyHO| (\verb|PEHO|). It prefers clauses that are the result of
\infname{ArgCong}, have equations between terms of functional type or
between higher-order patterns, or have literals containing logical symbols, in that
order of priority.

A higher-order inference that applies a complicated substitution to a clause is
usually followed by a $\beta\eta$-normalization step. If
$\beta\eta$-normalization greatly reduces the size of a clause, it is likely
that this substitution simplifies the clause (e.g., by removing a variable's
arguments). The new priority function \verb|ByNormalizationFactor| (\verb|BNF|)
is designed to exploit this observation. It prefers clauses that were produced
by $\beta\eta$-normalization, and among those it prefers the ones with larger
size reductions.

Another new priority function is \verb|PreferShallowAppVars| (\verb|PSAV|). This
prefers clauses with lower depths of the deepest occurrence of an applied
variable---that is, $C[X \, \cst{a}]$ is preferred over $C[\cst{f}\,(X \,
\cst{a})]$. The intuition is that applying a substitution to an applied
variable often reduces the variable to a term with a constant head,
yielding a less explosive clause, and the gain is greater for variables closer
to the top level.  Among the functions that rely
on properties of applied variables, we implemented \verb|PreferDeepAppVars|
(\verb|PDAV|), which returns the priority opposite of \verb|PSAV|, and
\verb|ByAppVarNum| (\verb|BAVN|), which prefers clauses with fewer occurrences
of applied variables.



\ourpara{Evaluation and Discussion}

\begin{figure}[t]
  \centering
  % \def\arraystretch{1.1}%
  \relax{\begin{tabular}{@{}l@{\hskip 2em}c@{\hskip 1em}c@{\hskip 1em}c@{\hskip 1em}c@{\hskip 1em}c@{\hskip 1em}c@{\hskip 1em}c@{\hskip 1em}c@{}} \toprule
            & \texttt{CP}                & \texttt{BAVN}  & \texttt{PL}  & \texttt{PSAV}       & \texttt{PHOS} & \texttt{PEHO}  & \texttt{BNF} & \texttt{PDAV}      \\ \midrule
   TPTP     & {1635}$^\star$             & {\bf 1640}     & 1604         & {1635}              & 1609          & 1617           & 1575         & 1533 \\[0.5\jot]
   SH       & {\bf \colalign452}$^\star$ & \colalign451   & \colalign417 & {\colalign450}      & \colalign439  & \colalign407   & \colalign411 & \colalign302 \\ \bottomrule
  \end{tabular}}
  \caption{Impact of the priority function}
  \label{fig:priorities}
\end{figure}
\begin{figure}[t]
  \centering
  \def\arraystretch{1.1}%
  \relax{\begin{tabular}{@{}l@{\hskip 2em}c@{\hskip 1em}c@{\hskip 1em}c@{\hskip 1em}c@{\hskip 1em}c@{\hskip 1em}c@{}} \toprule
              & $\infty$                     & $16$          & $8$           & $4$             & $2$             & $1$      \\ \midrule
   TPTP       & {\bf 1635}$^\star$           & 1625          & 1632          & 1629            & 1628            & 1618 \\[0.5\jot]
   SH         & {\bf \colalign452}$^\star$   & \colalign438  & \colalign435  & \colalign439    & \colalign435    & \colalign440 \\ \bottomrule

  \end{tabular}}
  \caption{Impact of the \infname{FluidSup} weight increment \ParamPenaltyIncrease}
  \label{fig:penalties}
\end{figure}

In the base configuration (\emph{base}), Zipperposition visits several clause
queues. The configuration uses queues that prefer the clauses that stem from
the conjecture, the ones that have at least one positive literal, the ones that
have been moved from active to passive set, and so on. One of the queues uses the constant
priority function \texttt{ConstPrio} (\texttt{CP}), meaning that it assigns the same
priority to every clause. As this queue is the most often visited one in \emph{base},
changing its priority function should affect the result noticeably.
To evaluate the new priority functions, we replaced
\texttt{CP} with one of the new functions in this queue,
%the queue ordered by \texttt{CP} with the queue ordered by one of the new
%functions,
leaving the clause weight intact. The results are shown in Fig.~\ref{fig:priorities}.

Even though constant priority function achieves remarkable performance, the new priority functions are
useful additions to the prover's repertoire: \NumberOK{37} additional
TPTP problems and \NumberOK{17} additional SH problems can be proved when some
nonconstant priority is used. The generally average-performing \verb|PEHO|
function can prove \NumberOK{9} problems not proved with any other priority function on TPTP
(and \NumberOK{1} on SH). Globally, \NumberOK{24} TPTP problems and \NumberOK{6} SH problems
can be proved exclusively using one particular priority function.


% It
% shows that the expensive priority functions \texttt{PHOS} and \texttt{BNF},
% which require inspecting the proof of clauses, hardly help. Simple functions
% such as \texttt{PL} are more effective: Compared with \textit{base}, \texttt{PL}
% loses \NumberNOK{one} problem overall but proves \NumberNOK{22} new problems.

% {\tt
%  \begin{itemize}

%   \item Another result which is hard to analyze without looking into differences.
%   \item On TPTP 32 solutions by using different priorities. On SH 20. Interestingly,
%   generally bad performing PEHO finds the largest number of solutions not found by CP : 15 (TPTP).
%   On SH the largest number of unique solutions is by PHOS and PEHO.
%   \item Another illustration of orthogonality: 25 problems can be proved by only one prio fun (TPTP), 9 on SH.

%  \end{itemize}
% }

Although it is necessary for refutational completeness, the \infname{FluidSup}
rule is disabled in \emph{base} because it is so explosive and so seldom useful.
To test whether increasing inference stream weights makes a difference on the
success rate, we tried enabling \infname{FluidSup} with different weight
increments~$\ParamPenaltyIncrease$ for \infname{FluidSup} inference queues. The
results are shown in Fig.~\ref{fig:penalties}. As expected, using a low
increment with \infname{FluidSup} is detrimental on TPTP. On this
benchmark set, \NumberOK{16} additional problems can be proved when
\infname{FluidSup} is enabled. The penalty mostly affects only proving time: All
but \NumberOK{2} of these problems were proved by using at least three different
values of $\ParamPenaltyIncrease$. On SH problems, the best result is obtained
when the rule is disabled as well. Unexpectedly, the next best result is obtained when
$\ParamPenaltyIncrease=1$.
% However, as the column for
% $\ParamPenaltyIncrease=16$ shows, nor should we use too high an increment, since that
% delays useful \infname{FluidSup} inferences. Interestingly, even
% though the configuration with $\ParamPenaltyIncrease=1$ proves the least problems overall, it proves
% \NumberNOK{7} problems not proved by \emph{base}, which is more than any other
% configuration we tried.

% {\tt
% \begin{itemize}
%   \item Not a single solution found using fluidsup on SH
%   \item 11 solutions found on TPTP, almost all (9) with at least 3 configurations.
%         Configurations usually only influence the time necessary to prove.
%   \item Even though it proves the least problems p1 is necessary in portfolio: with p=1 it takes 0.13 s seconds to prove NUM636\^{}2
%   and 5.95s with p16 (provable by all ps but not base).
%   \end{itemize}
% }

%%% @PETAR: Too little, not informative. What's a "shallow clause"? Not defined yet. --JB
%In informal experiments, we also tested limiting the applicability
%of some rules to shallow clauses, but this had no clear impact.

\section{Controlling the Use of Backends}
\label{sec:ho-tech:backends}

\newcommand{\ParamNumClauses}{\ensuremath{K_\mathrm{size}}}
\newcommand{\ParamTime}{\ensuremath{K_\mathrm{time}}}

Cooperation with efficient off-the-shelf first-order theorem provers is an
essential feature of higher-order theorem provers such as Leo-III
\cite[Sect.~4.4]{as-18-phd} and Satallax \cite{cb-2013-satallax}.
% but also HOLyHammer \cite{ku-15-holyhammer} and Sledgehammer \cite{bn-10-sh}.
Those provers invoke first-order backends repeatedly
during a proof attempt and spend a substantial amount of time in backend
collaboration. Since \lsup{} generalizes a highly efficient
first-order calculus, we expect that future efficient \lsup{}
implementations will not benefit much from backends.
Nevertheless, experimental provers such
as Zipperposition can still gain a lot. We present some
techniques for controlling the use of backends.

In his thesis \cite[Sect.~6.1]{as-18-phd}, Steen extensively studies
the effects of using different first-order backends on the performance of
Leo-III. His results suggest that adding only one backend already substantially
improves the performance. To reduce the effort required for integrating multiple backends, we chose Ehoh \cite{ehoh-section} as our single
backend. Ehoh is an extension of the highly optimized superposition prover E 2.5
%%% PETAR: Check 2.5. Later we have 2.6 but that's for Ehoh++.
with support for higher-order features such as partial
application, applied variables, and interpreted Booleans.
%formulas occurring as arguments of function symbols.
On the one hand, Ehoh provides the efficiency of E while easing the translation from full
higher-order logic---the only missing syntactic feature is
$\lambda$-abstraction. On the other hand, Ehoh's higher-order reasoning
capabilities are limited. Its unification algorithm is essentially first-order,
and it cannot synthesize $\lambda$-abstractions.
% or instantiate predicate variables.

In a departure from Leo-III and other cooperative provers,
\confrep{}{instead of regularly invoking the backend, }%
we invoke \confrep{the backend}{it} at most once during a run of Zipperposition.
This is because most competitive higher-order provers, including Zipperposition, use a portfolio
mode in which many configurations are run for a short time, and we want to
leave enough time for native higher-order reasoning. Moreover, multiple
backend invocations tend to be wasteful, because currently each invocation
starts with no knowledge of the previous ones.

\looseness=-1
Only a carefully chosen subset of the available clauses are translated and sent
to Ehoh. Let $I$ be the set of \confrep{input clauses}{clauses representing the
input problem}. Given a proof state, let \confrep{$M = P \cup A$}{$M$ denote the
union of the current active and passive sets}, and let $M_\mathrm{ho}$ denote
the subset of $M$ that contains only clauses that were derived using at least
one \lsup{} rule not present in regular superposition. We order the clauses in
$M_\mathrm{ho}$ by increasing derivation depth, using syntactic weight to break
ties. Then we choose all clauses in $I$ and the first \ParamNumClauses~clauses from
$M_\mathrm{ho}$ for use with the backend reasoner.
%We include all the clauses in $I$ since they constitute the backbone of the
%initial problem. -- Tautology. --JB
We leave out clauses in $M \setminus (I \cup M_\mathrm{ho})$ because Ehoh can rederive them.
We also expect large clauses with deep derivations to be less useful.

The remaining step is the translation of $\lambda$-abstractions. We implemented two
translation methods:\ $\lambda$-lifting \cite{tj-1985-lambdalift} and
$\cst{SKBCI}$ combinators \cite{da-1979-combtrans}. For
$\cst{SKBCI}$, we omit the combinator definition axioms, because they are
very explosive \cite{br-20-full-sup-w-combs}. A third mode simply omits clauses
containing $\lambda$-abstractions.

\ourpara{Evaluation and Discussion}
  %
  % \begin{enumerate}
  %   \item Different points when E is called, for different time periods
  %   \item Number of clauses from $M_\mathrm{ho}$ that is translated
  %   \item Different kinds of liftings
  % \end{enumerate}

  % We evaluated the following parameters of backend collaboration: the moment of
  % backend invocation, size of $M_\mathrm{ho}$, and $\lambda$-abstraction
  % translation method.
  In Zipperposition, we can adjust the CPU time allotted to Ehoh, Ehoh's own
  parameters, the point when Ehoh is invoked, the number \ParamNumClauses~of selected clauses from
  $M_\mathrm{ho}$, and the $\lambda$ translation method. We fix the time
  limit to 3~s, use Ehoh in \emph{autoschedule} mode, and focus on the last three
  parameters. In \emph{base}, collaboration with Ehoh is
  disabled (labeled $-$Ehoh).
  %Parameters are evaluated by enabling the collaboration with E and
  %varying parameter values.  -- Uninformative. --JB

  \begin{figure}[t]
    \noindent\hbox{}\hfill
    \begin{minipage}[t]{.46\linewidth}
      \centering
      \def\arraystretch{1.1}%
      \relax{\begin{tabular}{@{}l@{\hskip 0.75em}c@{\hskip 0.75em}c@{\hskip 0.75em}c@{\hskip 0.75em}c@{\hskip 0.75em}c@{}} \toprule
            & $-$Ehoh                   & $0.1$              & $0.25$          & $0.5$               & $0.75$      \\ \midrule
      TPTP & 1635$^\star$               & {\bf 1981}         & 1980            & 1979                & 1972 \\[0.5\jot]
      SH   & \colalign452$^\star$       & \colalign606       & {\bf \colalign608}    & \colalign600        & \colalign592 \\ \bottomrule
      \end{tabular}}
      \caption{Impact of the backend invocation point $\ParamTime$}
      \label{fig:invocation}
    \end{minipage}\hfill\hfill
    \begin{minipage}[t]{.46\textwidth}
      \centering
      \def\arraystretch{1.1}%
      \relax{\begin{tabular}{@{}l@{\hskip 0.75em}c@{\hskip 0.75em}c@{\hskip 0.75em}c@{\hskip 0.75em}c@{}} \toprule
           & $-$Ehoh              & lifting              & \textsf{SKBCI}    & omitted \\ \midrule
      TPTP & 1635$^\star$         & {\bf 1980}           & 1877              & 1866 \\[0.5\jot]
      SH   & \colalign452$^\star$ & \colalign{\bf 608}   & \colalign577      & \colalign566 \\ \bottomrule
      \end{tabular}}
      \caption{Impact of the method used to translate $\lambda$-abstractions}
      \label{fig:translation}
    \end{minipage}%
    \hfill\hbox{}
  \end{figure}

  \begin{figure}[t]
    \centering
    \def\arraystretch{1.1}%
    \relax{\begin{tabular}{@{}l@{\hskip 1em}c@{\hskip 1em}c@{\hskip 1em}c@{\hskip 1em}c@{\hskip 1em}c@{\hskip 1em}c@{\hskip 1em}c@{}}\toprule
          & $-$Ehoh              & $16$               & $32$               & $64$           & $128$             & $256$         & $512$     \\ \midrule
     TPTP & 1635$^\star$         & {\bf 1985}         & 1980               & 1978           & 1968              & 1968          & 1919 \\[0.5\jot]
     SH   & \colalign452$^\star$ & \colalign606       & {\bf \colalign608} & \colalign600   & \colalign598      & \colalign596  & \colalign589 \\ \bottomrule
    \end{tabular}}
    \caption{Impact of the number of selected clauses~$\ParamNumClauses$}
    \label{fig:m-ho-size}
  \end{figure}

  \looseness=-1
  Ehoh is invoked after $ \ParamTime \cdot t $ CPU seconds, where $0 \le \ParamTime < 1$ and $t$
  is the total CPU time allotted to Zipperposition. Figure~\ref{fig:invocation}
  shows the effect of varying $\ParamTime$ when $\ParamNumClauses = 32$
  and $\lambda$-lifting is used. The evaluation confirms that using a highly
  optimized backend such as Ehoh greatly improves the performance of a
  less optimized prover such as Zipperposition.
  The figure indicates that it is preferable to invoke the backend early. We
  have indeed observed that if the backend is invoked late, small clauses with
  deep derivations tend to be present by then. These clauses might have been
  used to delete important shallow clauses already. But due to their derivation
  depth, they will not be translated. In such situations, it is better to
  invoke the backend before the important clauses are deleted.

  Figure~\ref{fig:translation} quantifies the effects of the three
  $\lambda$-abstraction translation methods. We fixed $\ParamTime = 0.25$ and
  $\ParamNumClauses=32$. The clear winner is $\lambda$-lifting.
  $\cst{SKBCI}$ combinators perform slightly better than omitting clauses
  containing $\lambda$-abstractions.

  \looseness=-1
  Figure~\ref{fig:m-ho-size} shows the effect of $\ParamNumClauses$ on
  performance, with $\ParamTime = 0.25$ and $\lambda$-lifting. Including a
  small number of higher-order clauses with the lowest weight performs better
  than including a large number of such clauses.

\section{Comparison with Other Provers}
\label{sec:ho-tech:comparison}

%The raw evaluation data for the previous experiments show that
Different choices of
parameters lead to noticeably different sets of proved problems. In an attempt
to use Zipperposition~2 to its full potential, we have created a portfolio mode
that runs up to 50 configurations in parallel during the allotted time. The portfolio
was designed to solve as many problems as possible from the TPTP benchmark set. To
provide some context, we compare Zipperposition~2 with the latest versions of
all higher-order provers that competed at CASC-J10:\ CVC4 1.9
\cite{cbetal-11-cvc4}, Leo-III 1.5.6 \cite{sb-21-leo3}, Satallax 3.5
\cite{cb-2013-satallax}, and Vampire 4.5.1 \cite{br-20-full-sup-w-combs}.
The provers were run using the same parameters as in CASC, but with updated
executables.
Note that
Vampire's higher-order schedule is optimized for running on a single core.
We also include E~2.7 (more precisely, Ehoh, its higher-order configuration), the first version of this prover to syntactically support
full higher-order logic, including $\lambda$-abstractions.
Semantically, E~2.7 is arguably the weakest among the listed provers:
It simply performs $o$-RW rewriting described in Sect.~\ref{sec:ho-tech:preprocessing} followed by
$\lambda$-lifting before it applies \relax{$\lambda$-free superposition} \cite{bbcw-21-lfho}
(a precursor of all three higher-order superposition calculi) on the preprocessed problem.

We use the same benchmark sets as elsewhere in this article. To imitate the
CASC-J10 setup, we use a 120~s wall-clock limit and a 960~s CPU limit.
We even carried out our evaluation on the 8-core CPU nodes that were used for
CASC-J10. We also ran Zipperposition in uncooperative mode, in which its
collaboration with a backend is disabled. Figure~\ref{fig:other-provers}
summarizes the results.

\begin{figure}[t]
  \centering
  \def\arraystretch{1.1}%
  \relax{\begin{tabular}{@{}l@{\hskip 2.5em}c@{\hskip 2em}c@{}} \toprule
                  & TPTP    & SH  \\ \midrule
   CVC4           & 1816    & 587   \\
   E              & 1980    & 676   \\
   Leo-III        & 2122    & 616  \\
   Satallax       & 2175    & 588  \\
   Vampire        & 2072    & 660   \\
   Zipperposition-uncoop & 2311 & 652 \\
   Zipperposition & {\bf 2412}    & {\bf 715}  \\ \bottomrule
  \end{tabular}}
  \caption{Comparison of competing higher-order theorem provers}
  \label{fig:other-provers}
\end{figure}

% Among the cooperative provers, Zipperposition is the one that depends the least
% on its backend, and its \emph{uncooperative} mode is only \NumberNOK{one}~problem
% behind Satallax's \emph{cooperative} mode. This confirms our hypothesis that
% \osup{} is a suitable basis for automatic higher-order reasoning.
% \confrep{This also}{The increase in performance due to the addition of an efficient backend}
% suggests that the implementation of this calculus in a modern first-order
% superposition prover such as E or Vampire would achieve markedly better results.
% Moreover, we believe that there are still techniques inspired by tableaux, SAT
% solving, and SMT solving that could be adapted and integrated in saturation
% provers.

The evaluation results corroborate the CASC results. They also
show that Zipperposition outperforms all other provers on SH benchmarks. This
confirms our hypothesis that \lsup{} is a suitable basis for automatic
higher-order reasoning. Further confirmation is provided by the success rate of
Zipperposition's uncooperative version: Even without backend,
Zipperposition is substantially better than all other provers on TPTP
benchmarks, and it matches the performance of the top contenders on SH.
On the other hand, the increase in performance due to the addition
of an efficient backend suggests that the implementation of this calculus in a
modern first-order superposition prover such as E or Vampire would achieve
even better results.

We believe that there are still techniques inspired by
tableaux, SAT solving, and SMT solving that could be adapted and integrated in
saturation provers. In particular, there are still \NumberOK{25} TPTP problems
and \NumberOK{17} SH problems that can be proved by other provers but not by
Zipperposition.

\section{Discussion and Conclusion}
\label{sec:ho-tech:discussion}

%  In this paper, we proposed solutions to issues that arise when implementing a
%  higher-order prover based on \lsup. Some of these also apply
%  to combinatory superposition. We also discussed many choices that can be made
%  during proof search and evaluated some of those choices.

\begin{sloppypar}
Back in 1994, Kohlhase \cite[Sect.~1.3]{mk-94-hores} was optimistic about the
future of higher-order automated reasoning:
%
\begin{quote}
  %The author believes that
  The obstacles to proof search intrinsic
  to higher-order logic may well be compensated by the greater expressive power
  of higher-order logic and by the existence of shorter proofs. Thus
  higher-order automated theorem proving will be practically as feasible as
  first-order theorem proving is now as soon as the technological backlog is made up.
\end{quote}
%
For higher-order superposition, the backlog consisted of designing calculus
extensions, heuristics, and algorithms that mitigate its weaknesses. In
this article, we presented such enhancements, justified their design, and
evaluated them. We explained how each weak point in the
higher-order proving pipeline could be improved, from preprocessing to reasoning
about formulas, to delaying unpromising or explosive inferences, to invoking a
backend. Our evaluation indicates that higher-order superposition is now the
state of the art in higher-order reasoning.
\end{sloppypar}

\looseness=-1
Higher-order extensions of first-order superposition have been
considered %from a theoretical perspective    -- The point is made later in the same para. --JB
by Bentkamp et al.\
\cite{bbtvw-21-sup-lam, bbcw-21-lfho} and Bhayat and Reger
\cite{br-19-restricted-unif, br-20-full-sup-w-combs}. They introduced proof calculi,
proved them refutationally complete, and suggested optional rules, but
they hardly discussed the practical aspects of higher-order superposition. Extensions
of SMT are discussed by Barbosa et al. \cite{brotb-19-ho-smt}.
Bachmair and Ganzinger \cite{bg-1992-nonclausal}, Manna and Waldinger
\cite{mw-1979-nonclausal}, and Murray \cite{nm-1982-nonclausal} have studied
nonclausal resolution calculi.

In contrast, there is a vast literature on practical aspects of first-order
reasoning using superposition and related calculi.
The literature evaluates various procedures and techniques
\cite{hv-09-unifalgs,rsv-15-playing-with-avatar}, literal and term order selection
functions \cite{hrsv-16-selsel}, and clause evaluation functions
\cite{sm-2016-clausesel, gs-20-clausesel}, among others. Our work joins the
select club of papers devoted to practical aspects of higher-order
reasoning
\cite{sb-15-beta,wskb-16-effective-norm,fb-2016-internal-guidance-satallax,benzmueller-et-al-05-can-ho-fo-coop}.

%In contrast, higher-order calculi implemented in competitive provers are usually discussed from a theoretical viewpoint.

%\section{Conclusion and Future Work}
%\label{sec:ho-tech:conclusion}

% We filled in this gap: we describe how
% We described some ways in which the explosion incurred by
% \lsup{} can be kept under control. We showed how native
% support for a Boolean type can dramatically improve a higher-order prover's
% performance. For many choices that can be made during the higher-order proof
% search, we described heuristics and evaluated them. We also designed a
% saturation loop that enumerates infinite sets of higher-order inference
% conclusions, while maintaining the same behavior as the standard given clause
% procedure on first-order clauses.

As a next step, we plan to implement the described techniques in E.
%%% @PETAR: In E 2.6, basically. --JB
% Ehoh \cite{ehoh-section}, the $\lambda$-free higher-order extension of E.
We
expect the resulting prover to be substantially more efficient than
Zipperposition. Moreover, we want to investigate the proofs found by provers
such as CVC4 and Satallax but missed by Zipperposition. Finding the reason
behind why Zipperposition fails to prove specific problems will likely result in useful new techniques.


\def\ackname{Acknowledgment} % American English
\begin{acknowledgements}
We are grateful to the maintainers of StarExec for letting us use their service.
%Mathias Fleury helped us setup the ORCID icons.
Ahmed Bhayat and Giles Reger guided us
through details of Vampire 4.5. Ahmed Bhayat, Michael F\"arber,
Mathias Fleury, Predrag Jani\v ci\'c,
Mark Summerfield, and the anonymous reviewers suggested content, textual, and
typesetting improvements. We thank them~all.

Vukmirovi\'c, Bentkamp, and Blanchette's research has received funding from
the European Research Council (ERC) under the European Union's Horizon 2020
research and innovation program (grant agreement No.\ 713999, Matry\-osh\-ka).
%
Blanchette and Nummelin's research has
received funding from the Netherlands Organization for Scientific Research (NWO)
under the Vidi program (project No.\ 016.Vidi.189.037, Lean Forward).
\end{acknowledgements}


% Authors must disclose all relationships or interests that
% could have direct or potential influence or impart bias on
% the work:
%
% \section*{Conflict of interest}
%
% The authors declare that they have no conflict of interest.


% BibTeX users please use one of
\bibliographystyle{spmpsci}

\bibliography{bib}

\end{document}
% end of file template.tex
