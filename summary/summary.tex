\chapter*{Summary}
\addcontentsline{toc}{chapter}{Summary}
\setheader{Summary}

\looseness=-1
In the last decades, proof assistants were immeasurably useful in formally
proving validity of hard mathematical theories and correctness of
safety-critical software. Using these tools one can formally describe a problem
and produce machine-checkable proofs for statements about it. To make the
checking phase efficient and trustworthy, proof steps need to be of fine
granularity. As a consequence, seemingly simple statements must be proven in
minute detail, making the use of proof assistants very tedious.

In an attempt to automate significant parts of this proving process, assistants
can invoke automatic theorem provers to finish the proof. However, most
assistants are based on higher-order logic, while most automatic theorem provers
are based on first-order logic. This means that the two systems must communicate
through translations, which obfuscate the problem and likely make it more
difficult than it originally was.

In this thesis, we try to bridge this translation gap. We start from E, one of the best
first-order proof assistants backends, based on the superposition calculus, and
gradually extend it to higher-order logic. As this approach is large in scope,
we split it in three parts.

The first part extends E to support a fragment of higher-order logic devoid of
$\lambda$-abstra\-ction, yielding a prover called Ehoh. Despite this extension is small in scope, 
Ehoh shows stronger performance on proof assistant
benchmarks than E.

The second part concerns adding support for $\lambda$-abstraction. The most
important challenge we faced during this extension is that of higher-order
unification. At the time we started, state-of-the-art procedure
for full higher-order unification was developed in 1970s. We designed a new
procedure inspired by this one that enumerates fewer redundant unifiers, has many
modern optimizations built-in, and can easily be customized to trade its
completeness for performance. It is implemented in a
prototype prover called Zipperposition, a less efficient, but more easily extendible
prover compared to E. Our evaluation shows that our
procedure substantially improves on the state of the art.

The last part concerns adding support for
first-class Boolean terms. For this part we took inspiration from traditional
automatic higher-order provers, based on tableaux. We fitted those techniques in
superposition context and implemented them in Zipperposition. They
made Zipperposition the best prover in the higher-order division of the annual CASC
theorem prover competition for two consecutive years.

After testing out our ideas in Zipperposition, we decided to go back to Ehoh and
extend it to full higher-order logic, obtaining a prover called \ehohii{}. We
chose the most successful approaches implemented in Zipperposition that can
easily be ported to \ehohii{}. The results were once again positive: Our
evaluation shows that \ehohii{} is the best prover on proof assistant
benchmarks, and second only to Zipperposition on a standard benchmark set.

We took our idea of porting techniques from weaker to stronger logics one step
back: We explored SAT simplification techniques that can be implemented in the
superposition context. We discovered that while some of them can be used during proving,
most of them work best as preprocessors.


\chapter*{Samenvatting}
\addcontentsline{toc}{chapter}{Samenvatting}
\setheader{Samenvatting}

{\selectlanguage{dutch}


}



