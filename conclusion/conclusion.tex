\chapter{Conclusion}
\setheader{Conclusion}
\label{ch:conclusion}
 
As we bring this thesis to an end, I would like to use these last pages to
reflect not only on the work I have done in the last four years, but also on the
many bumps on the road that I did not get to describe in the preceding chapters.
When it comes to reflections, the  performance art piece Rhythm 10, performed by
Marina Abramovi\'c,  one of the greatest Yugoslav artists, comes to mind. The
photograph on the cover of this thesis was taken when it was first performed, in
Edinburgh, in 1973. Its instructions read:

\begin{quote}
    \underline{Preparation} \\[1\jot]
    I place a sheet of white paper on the floor.\\
    I lay 20 knives of different sizes and shapes on the floor.\\
    I place 2 tape recorders with microphones on the floor.
    
    \underline{Performance} \\[1\jot]
    I turn on the first tape recorder.\\
    I take the first knife and stab in between the fingers of my left hand as
    fast as possible.\\
    Every time I cut myself, I change to a different knife. \\
    When I've used all of the knives (all the rhythms), I rewind the tape
    recorder. \\ 
    I listen to the recording of the first part of the performance. \\ I
    concentrate. \\
    I repeat the first part of the performance. \\
    I take the knives in the same order, follow to the same order, follow to the
    same rhythm and cut myself in the same places. \\
    In this performance, the mistakes of time past and time present are
    synchronized.\\
    I rewind the second tape recorder and listen to the double rhythm of the
    knives.\\
    I leave.
\end{quote}

Just like this performance, the work that I have done in the Matryoshka project
was cyclic: It consists of three phases (three stops as we described them in
Chapter \ref{ch:intro}), each of which has the same parts: theory (calculus and
algorithms), engineering (implementation) and optimization (heuristics). While
the calculi were mostly developed by Alexander Bentkamp
\cite{bbcw-21-lfho,bbtvw-21-sup-lam,bbtv-21-full-ho-sup}, the task of developing
algorithms for indexing or unification, as well as the tasks of implementing and
optimizing the implementation were mainly mine.

Structuring the project cyclically turned out to be a great idea. When I first
started doing the research which will later form the basis of Chapter
\ref{ch:ehoh}, I was overwhelmed by the scope of the project and felt uncertain
if E could ever fully support $\lambda$-free higher-order logic. Many of the
extensions described in the  later chapters were even larger in scope. However,
as each extension relied on the previous one, I quickly developed a feeling of
where the possible pitfalls are and how to avoid them. This gave me an invaluable
source of encouragement, without which I could hardly arrive at the end of the
project.

The work in Chapter~\ref{ch:ehoh} completed the first cycle. Soon after the
original paper introducing Ehoh \cite{vbss-19-ehoh1} was published, Ehoh found
its use as the backend of Sledgehammer and Satallax, showing that even when limited
to a fragment of higher-order logic, efficient higher-order reasoning is in
demand.

The goal of the next cycle was to support $\lambda$-abstraction. When we first
implemented \lsup{}, calculus that achieves this goal, we quickly realized that
one of its main flaws was that it relies on enumerating full unifiers rather
than preunifiers, like some older resolution-based higher-order calculi do. This
is why we focused on developing a full unification procedure that removes the
redundancy present in other procedures and that implements all the advancements
and optimizations described in the unification literature after the introduction
of the most influential full unification procedure, Jensen-Pietrzykowski's
procedure. We also made sure that this procedure can easily be customized to
trade bits of its completeness for substantially improved performance. This
procedure, described in Chapter~\ref{ch:unif}, transformed the \lfsup{} calculus
into a competitive higher-order calculus.

Even though \lfsup{} became competitive, its implementation in Zipperposition
still still lagged behind the competition. To investigate why this is the case,
we manually analyzed hundreds of benchmarks on which the competition proves, but
on which Zipperposition fails. This analysis forms the basis of work that is
described in Chapters \ref{ch:bools} and \ref{ch:ho-techniques}. The main
conclusion was that the competition, most notably tableaux-based Satallax,
reasons better with formulas. Tableaux inferences are more intuitive than the
ones that superposition or resolution provers perform and are more in line with
how a mathematician structures a proof. Even though fitting these kind of
inferences into the superposition context was challenging, it proved very
successful. After they were implemented in Zipperposition, it took its first
victory at higher-order division of CASC.

Our extension of \lfsup{} was guided by performance, rather than completeness
with respect to full higher-order logic. After the first victory, we were left
wondering what were the kind of problems that our implementation can not
solve, and started looking for complete full higher-order calculus. At the end,
we designed \osup{} which gave us the precise answer in the form of problems that
cannot be solved and the rules which are necessary to solve them. 
