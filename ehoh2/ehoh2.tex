\chapter{Extending a Brainiac Prover to Higher-Order Logic}
\setheader{Extending a Brainiac Prover to Higher-Order Logic}
\label{ch:ehoh2}

% \renewcommand{chapter}[0]{chapter}
\newcommand\ehohii{$\lambda$E}


\blfootnote{In this work I designed, implemented and evaluated all changes to
term representation, algorithms and indexing data structures. Jasmin Blanchette
did the daily supervision. Stephan Schulz provided
the necessary E expertise.}

\begin{abstract}%
    The automatic discharge of tedious subgoals is high on the wishlist of many
  users of proof assistants. Some proof assistants discharge such goals
  by translating them to first-order logic and invoking an efficient prover on
  them, but much is lost in translation. As an alternative,
  we propose to extend first-order provers with native support for
  higher-order features. Building on our extension of E to $\lambda$-free
  higher-order logic, we now extend E to full higher-order logic.
  The resulting prover is the \NumberOK{strongest} one on benchmarks coming from a
  proof assistant, and the second best on TPTP benchmarks.
    %It also incurs no overhead on first-order problems.  -- sounds like a detail --JB
\end{abstract}

\newpage

\section{Introduction}
\label{sec:ehoh2:introduction}

We began this thesis by introducing Ehoh, a rather conservative extension of
state-of-the-art first-order prover to a fragment of higher-order logic. This
extension gave us a flavor of the difficulties that we might encounter on the
way to full higher-order logic. In chapters that precede this one,
we discussed many ways in which those difficulties can be overcome.
In this chapter, we fulfill a promise we gave in the beginning of the thesis: We present the extension of
Ehoh to full higher-order logic (Sect.~\ref{sec:ehoh2:logic}) using incomplete variants
of $\lambda$-superposition. We call this prover \ehohii.

%
The $\lambda$-superposition calculi were
previously implemented in Zipperposition, and
extensive experiments with various heuristic choices have been performed
(Chapter \ref{ch:ho-techniques}). In \ehohii{}'s implementation, we used
these experiences to choose a set of effective rules that could easily be
retrofitted into an originally first-order prover. Another principle that guided 
the design of \ehohii{} was \emph{gracefulness}: we made sure that our changes
do not impact the strong first-order performance of E and $\lambda$-free higher-order performance of Ehoh. 

% We
% also used the experience of fine-tuning the calculi
% \cite{section-making-ho-work} in Zipperposition to carefully choose which
% calculus extensions and heuristics to use in E.

One of the main challenges we faced was retrofitting $\lambda$-terms in Ehoh's
term representation (Sect.~\ref{sec:ehoh2:terms}). Furthermore, Ehoh's main inference
engine was designed with the assumption that it will be used with
inferences that compute a most general unifier. We
implementd a higher-order unification procedure (Chapter \ref{ch:unif})
that can return multiple unifiers (Sect.~\ref{sec:ehoh2:unif-match-index}) and
integrated it in the inference engine. Finally, we extended and adapted the
superposition rule, resulting in an incomplete, pragmatic variant of
$\lambda$-superposition (Sect.~\ref{sec:ehoh2:calculus}).

We \NumberOK{evaluated} \ehohii{} on a selection of proof assistants benchmarks
as well as all higher-order theorems in the TPTP library \cite{gs-17-tptp}
(Sect.~\ref{sec:ehoh2:eval}). We found
that \ehohii{} clearly outperforms Ehoh on all benchmarks. It outperformed all other higher-order provers on
proof assistant benchmarks; on TPTP benchmarks it ended up second only 
to the cooperative version of Zipperposition, which employs Ehoh as a
backend. An arguably fairer comparison without the backend puts \ehohii{} in the
first place for both benchmark suites.
We also compared the performance of \ehohii{} with E on first-order
%%% @PETAR: I made this slightly sentence shorter and more forceful. Please
%%% check if you agree. --JB
problems and found that little overhead has been introduced by the
extension to higher-order logic.

