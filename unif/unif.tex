\chapter{Efficient Full Higher-Order Unification}
\setheader{Efficient Full Higher-Order Unification}
\label{ch:unif}

\blfootnote{In this work I designed the main algorithm, solid oracle and
fingerprint indexing. I also implemented and evaluated the algorithms. Visa
Nummelin proved the completeness of the main algorithm with Alexander Bentkamp's help.
The completeness and termination of solid oracle was proven by me with Alexander
Bentkamp's help.}

\authors{Joint work with\\
Alexander Betnkamp and Visa Nummelin}

\begin{abstract}
  We developed a procedure to enumerate complete sets of higher-order unifiers based on work
  by Jensen and Pietrzykowski. Our procedure removes many redundant unifiers by
  carefully restricting the search space and tightly integrating decision
  procedures for fragments that admit a finite complete set of unifiers. We
  identify a new such fragment and describe a procedure for computing its unifiers.
  Our unification procedure, together with new higher-order term indexing data structures, 
  is implemented in the Zipperposition theorem prover.
  Experimental evaluation shows a clear advantage over Jensen and Pietrzykowski's
  procedure.    
\end{abstract}
    
\newpage

\section{Introduction}
\label{sec:unif:intro}

Unification is concerned with finding a substitution that makes two terms
equal, for some notion of syntactic equality. Since the invention of Robinson's first-order unification
algorithm \cite{ar-65-resolution}, it has become an indispensable
tool in theorem proving, logic
programming, natural language processing, programming language compilation and other areas of computer science.

Many of these applications are based on higher-order formalisms and require
higher-order unification. Due to its undecidability and explosiveness,
the higher-order unification problem is considered one of the main
obstacles on the road to efficient higher-order tools.

One of the reasons for higher-order unification's explosiveness lies in
\emph{flex-flex pairs}, which consist of two variable-headed terms,
e.g., $F \, X \unif G \, \cst{a}$. Even this seemingly simple
problem has infinitely many incomparable unifiers.
%
One of the first methods designed to combat this explosion is Huet's
preunification~\cite{gh-75-unification}. Huet noticed that some logical calculi
would remain complete if flex-flex pairs are not eagerly solved but postponed as
constraints. If only flex-flex constraints remain, we know that a unifier must
exist and we do not need to solve them.
%
Huet's preunification has been used in many reasoning tools including Isabelle
\cite{tn-02-isabelle}, Leo-III \cite{sb-21-leo3}, and Satallax
\cite{cb-12-satallax}. However, recent developments in higher-order theorem
proving~\cite{bbtvw-21-sup-lam,br-19-restricted-unif} require full unification---i.e., enumeration of unifiers even for
flex-flex pairs, which is the focus of this chapter.

Jensen and Pietrzykowski's (JP) procedure \cite{jp-76-unif} is the best known
procedure for this purpose (Section \ref{sec:unif:background}). Given two terms to
unify, it first identifies a position where the terms disagree.
Then, in parallel branches of the search tree, it applies suitable substitutions, involving a
variable either at the position of disagreement or above, and repeats this process on
the resulting terms until they are equal or trivially nonunifiable.

Building on the JP procedure, we designed a new procedure (Section
\ref{sec:unif:the-unification-procedure}) with the same completeness guarantees (Section \ref{sec:unif:proof-of-completeness}).
The new procedure addresses many of the issues that are
detrimental to the performance of the JP procedure.
%
First, the JP procedure does not terminate in many cases of obvious
nonunifiability, e.g., for $X \unif \cst{f} \, X$, where $X$ is a non-functional
variable and $\cst{f}$ is a function constant. This example also shows that
the JP procedure does not generalize Robinson's first-order procedure gracefully. To address
this issue, our procedure detects whether a unification problem belongs to a
fragment for which unification is decidable and finite complete sets of unifiers (CSUs)
exist.
% Another approach for treating explosiveness of higher-order unification is to
% restrict our attention to some fragment $L$ of $\lambda$-terms that has
% decidable, unary unification problem. We say that the unification problem is
% unary if any two terms belonging to $L$ are either not unifiable or have a most
% general unifier (MGU). 
We call algorithms that enumerate elements of the CSU for such fragments
\emph{oracles}. Noteworthy fragments with oracles are first-order terms,
patterns \cite{tn-93-patterns}, functions-as-constructors
\cite{tl-16-facunif}, and a new fragment 
we present in Section~\ref{sec:unif:solid-oracle}.
%
The unification procedures of Isabelle and Leo-III check whether the unification
problem belongs to a decidable fragment, but we take this idea a step further by
checking this more efficiently and for every subproblem arising during
unification.

Second, the JP procedure computes many redundant unifiers. Consider the
example $F \, (G \, \cst{a}) \unif F \, \cst{b}$, where it produces, in addition
to the desired unifiers $\{F \mapsto \lambda x. \, H\}$ and $\{G \mapsto \lambda
x. \, \cst{b}\}$, the redundant unifier $\{F \mapsto \lambda x. \, H,\; G \mapsto
\lambda x. \, x\}$. 
The design of our procedure avoids computing many redundant unifiers, including
this one. Additionally, as oracles usually return a small CSU, 
their integration reduces the number of redundant unifiers.

\newpage
Third, the JP procedure applies more explosive rules than Huet's preunification procedure to
flex-rigid pairs. To gracefully generalize Huet's procedure, 
we show that his rules for flex-rigid pairs suffice 
to enumerate CSUs
if combined with appropriate rules for flex-flex pairs.

Fourth, the JP procedure repeatedly traverses the parts of the unification
problem that have already been unified. Consider the problem $\cst{f}^{100} \,
(G \, \cst{a}) \unif \cst{f}^{100} \, (H \, \cst{b})$, where the exponents
denote repeated application. It is easy to see that this problem can be reduced
to $G \, \cst{a} \unif H \, \cst{b}$. However, the JP procedure will wastefully
retraverse the common context $ \cst{f}^{100} [\;] $ after applying each new
substitution. Since the JP procedure must apply substitutions to the variables occurring in the
common context above the position of disagreement, it cannot be easily adapted to
eagerly decompose unification pairs. By contrast, our procedure is designed to decompose
the pairs eagerly, never traversing a common context twice.

Last, the JP procedure does not allow to apply substitutions and
$\beta$-reduce lazily.
The rules of simpler procedures (e.g., first-order
\cite{hv-09-unifalgs} and pattern unification \cite{tn-93-patterns}) depend only on
the heads of the unification pair.
Thus, to
determine the next step, implementations of these procedures need to
substitute and
$\beta$-reduce only until the heads of the current unification
pair are not mapped by the substitution and are not $\lambda$-abstractions. 
Since the JP procedure is not based on the decomposition of unification pairs, 
it is unfit for
optimizations of this kind.
We designed our procedure to allow for this optimization.

% In Section \ref{sec:unif:indexing} we discuss a higher-order extension of fingerprint indexing,
% a technique to filter out terms that are not unifiable with a given
% query term from a set of terms. 
To more efficiently find terms (in a large term set) that are unifiable with a given query term,
we developed a higher-order extension of fingerprint indexing \cite{ss-12-fp-indexing}
(Section \ref{sec:unif:indexing}).
We implemented our procedure, several oracles,
and the fingerprint index in
the Zipperposition prover (Section \ref{sec:unif:implementation}). Since a
straightforward implementation of the JP procedure already existed in
Zipperposition, we used it as a baseline to evaluate the performance of our
procedure (Section \ref{sec:unif:evaluation}). The results show substantial
performance improvements.

% This invited article is an extended version of our FSCD-2020 paper \cite{vbn-21-unif}.
% Most notable extension is the Section \ref{sec:unif:proof-of-completeness}, which gives
% the detailed proof of completeness of our new procedure. In addition, 
% we give proofs for all the unproved statements from the paper, expand the examples 
% and provide more detailed explanations.


\section{Background}
\label{sec:unif:background}

Our setting is the simply typed $\lambda$-calculus. Unless mentioned otherwise,
we use the same notation as laid out in Chapter~\ref{ch:pre}. Additional notions
are introduced as follows.

\emph{Parameters} and \emph{body} for any term $\param[\seq x]s$ are defined to
be $\seq x$ and $s$ respectively, where $s$ is not a $\lambda$-abstraction. The
\emph{size} of a term is inductively defined as $\text{size}(F) = 1$;
$\text{size}(x) = 1$; $\text{size}(\cst{f}) = 1$; $\text{size}(s\,t) =
\text{size}(s) + \text{size}(t)$; $\text{size}(\lambda x.\, s) = \text{size}(s)
+ 1$. A term is in \emph{head normal form} ({\em hnf}) if it is of the form
$\lambda \overline{x}.\,a\,\overline{t}$, where $a$ is a free variable, bound
variable, or a constant. In this case, $a$ is called the \emph{head} of the
term. Note that this relaxes the condition that the term needs to be in
$\beta$-normal form to determine its head. A term is called flex or rigid if its
head is flex or rigid, respectively. By $\nf{s}{\textsf{h}}$ we denote the term
obtained from a term $s$ by repeated $\beta$-reduction of the leftmost outermost
redex until it is in hnf. Unless stated otherwise, we view terms syntactically,
as opposed to $\alpha\beta\eta$-equivalence classes. The common context
$\mathcal{C}(s,t)$ of two $\eta$-long $\beta$-reduced terms $s$~and~$t$ of the
same type is defined inductively as follows, assuming that $a \not= b$:
$\mathcal{C}(\lambda x.\, s, \lambda y.\, t) = \lambda x.\,
\mathcal{C}(s,\{y\mapsto x\}t)$; $\mathcal{C}(a\,\overline{s}_m,
b\,\overline{t}_n) = \square$; $\mathcal{C}(a\,\overline{s}_m,
a\,\overline{t}_m) = a\,\mathcal{C}(s_1,t_1)\,\ldots\,\mathcal{C}(s_m,t_m)$.
Unless otherwise stated, we take unification constraint $s \unif t$ to be an
unordered pair of two terms of the same type. To ease notation, we do not write
parentheses around application of substitutions to terms (or other objects containing terms);
in other words we shorten $\sigma(\theta(\varrho(s)))$ to $\sigma\theta\varrho \, s$

\ourpara{Remark}
We use the definition of a CSU from Sect.~\ref{sec:pre:unif} because 
JP's definition of a CSU, which we have adopted in our earlier work \cite{vbn-21-unif}, is flawed.
JP's definition does not employ the notion of auxiliary variables,
but instead requires $\rho X = \theta\sigma X$ for all variables mapped by $\rho$.
This is problematic because nothing prevents $\rho$ from mapping the auxiliary variables.
For example, $\sigma = \{F \mapsto \lambda x y.\> G\>y\}$ is supposed
to be an MGU for $F\>\cst{a}\>\cst{c} \unif F\>\cst{b}\>\cst{c}$.
But for the unifier $\rho = \{F \mapsto \lambda x y.\> y,\> G \mapsto \lambda x.\> \cst{d}\}$,
without the notion of auxiliary variables,
there exists no appropriate substitution $\theta$
because $\rho G = \theta\sigma G$ requires $\theta G = \lambda x.\> \cst{d}$
and $\rho F = \theta\sigma F$ requires $\theta G = \lambda x.\> x$.



\section{The Unification Procedure}
\label{sec:unif:the-unification-procedure}

\section{Proof of Completeness}
\label{sec:unif:proof-of-completeness}

\section{A New Decidable Fragment}
\label{sec:unif:solid-oracle}

\section{An Extension of Fingerprint Indexing}
\label{sec:unif:indexing}

\section{Implementation}
\label{sec:unif:implementation}

\section{Evaluation}
\label{sec:unif:evaluation}

\section{Discussion and Related Work}

\section{Conclusion}

