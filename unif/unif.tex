\chapter{Efficient Full Higher-Order Unification}
\setheader{Efficient Full Higher-Order Unification}
\label{ch:unif}

\blfootnote{In this work I designed the main algorithm, solid oracle and
fingerprint indexing. I also implemented and evaluated the algorithms. Visa
Nummelin proved the completeness of the main algorithm with Alexander Bentkamp's help.
The completeness and termination of solid oracle was proven by me with Alexander
Bentkamp's help.}

\authors{Joint work with\\
Alexander Betnkamp and Visa Nummelin}

\begin{abstract}
  We developed a procedure to enumerate complete sets of higher-order unifiers based on work
  by Jensen and Pietrzykowski. Our procedure removes many redundant unifiers by
  carefully restricting the search space and tightly integrating decision
  procedures for fragments that admit a finite complete set of unifiers. We
  identify a new such fragment and describe a procedure for computing its unifiers.
  Our unification procedure, together with new higher-order term indexing data structures, 
  is implemented in the Zipperposition theorem prover.
  Experimental evaluation shows a clear advantage over Jensen and Pietrzykowski's
  procedure.    
\end{abstract}
    
\newpage

\section{Introduction}
\label{sec:unif:intro}

Unification is concerned with finding a substitution that makes two terms
equal, for some notion of syntactic equality. Since the invention of Robinson's first-order unification
algorithm \cite{ar-65-resolution}, it has become an indispensable
tool in theorem proving, logic
programming, natural language processing, programming language compilation and other areas of computer science.

Many of these applications are based on higher-order formalisms and require
higher-order unification. Due to its undecidability and explosiveness,
the higher-order unification problem is considered one of the main
obstacles on the road to efficient higher-order tools.

One of the reasons for higher-order unification's explosiveness lies in
\emph{flex-flex pairs}, which consist of two variable-headed terms,
e.g., $F \, X \unif G \, \cst{a}$. Even this seemingly simple
problem has infinitely many incomparable unifiers.
%
One of the first methods designed to combat this explosion is Huet's
preunification~\cite{gh-75-unification}. Huet noticed that some logical calculi
would remain complete if flex-flex pairs are not eagerly solved but postponed as
constraints. If only flex-flex constraints remain, we know that a unifier must
exist and we do not need to solve them.
%
Huet's preunification has been used in many reasoning tools including Isabelle
\cite{tn-02-isabelle}, Leo-III \cite{sb-21-leo3}, and Satallax
\cite{cb-12-satallax}. However, recent developments in higher-order theorem
proving~\cite{bbtvw-21-sup-lam,br-19-restricted-unif} require full unification---i.e., enumeration of unifiers even for
flex-flex pairs, which is the focus of this chapter.

Jensen and Pietrzykowski's (JP) procedure \cite{jp-76-unif} is the best known
procedure for this purpose (Section \ref{sec:unif:background}). Given two terms to
unify, it first identifies a position where the terms disagree.
Then, in parallel branches of the search tree, it applies suitable substitutions, involving a
variable either at the position of disagreement or above, and repeats this process on
the resulting terms until they are equal or trivially nonunifiable.

Building on the JP procedure, we designed a new procedure (Section
\ref{sec:unif:the-unification-procedure}) with the same completeness guarantees (Section \ref{sec:unif:proof-of-completeness}).
The new procedure addresses many of the issues that are
detrimental to the performance of the JP procedure.
%
First, the JP procedure does not terminate in many cases of obvious
nonunifiability, e.g., for $X \unif \cst{f} \, X$, where $X$ is a non-functional
variable and $\cst{f}$ is a function constant. This example also shows that
the JP procedure does not generalize Robinson's first-order procedure gracefully. To address
this issue, our procedure detects whether a unification problem belongs to a
fragment for which unification is decidable and finite complete sets of unifiers (CSUs)
exist.
% Another approach for treating explosiveness of higher-order unification is to
% restrict our attention to some fragment $L$ of $\lambda$-terms that has
% decidable, unary unification problem. We say that the unification problem is
% unary if any two terms belonging to $L$ are either not unifiable or have a most
% general unifier (MGU). 
We call algorithms that enumerate elements of the CSU for such fragments
\emph{oracles}. Noteworthy fragments with oracles are first-order terms,
patterns \cite{tn-93-patterns}, functions-as-constructors
\cite{tl-16-facunif}, and a new fragment 
we present in Section~\ref{sec:unif:solid-oracle}.
%
The unification procedures of Isabelle and Leo-III check whether the unification
problem belongs to a decidable fragment, but we take this idea a step further by
checking this more efficiently and for every subproblem arising during
unification.

Second, the JP procedure computes many redundant unifiers. Consider the
example $F \, (G \, \cst{a}) \unif F \, \cst{b}$, where it produces, in addition
to the desired unifiers $\{F \mapsto \lambda x. \, H\}$ and $\{G \mapsto \lambda
x. \, \cst{b}\}$, the redundant unifier $\{F \mapsto \lambda x. \, H,\; G \mapsto
\lambda x. \, x\}$. 
The design of our procedure avoids computing many redundant unifiers, including
this one. Additionally, as oracles usually return a small CSU, 
their integration reduces the number of redundant unifiers.

\newpage
Third, the JP procedure applies more explosive rules than Huet's preunification procedure to
flex-rigid pairs. To gracefully generalize Huet's procedure, 
we show that his rules for flex-rigid pairs suffice 
to enumerate CSUs
if combined with appropriate rules for flex-flex pairs.

Fourth, the JP procedure repeatedly traverses the parts of the unification
problem that have already been unified. Consider the problem $\cst{f}^{100} \,
(G \, \cst{a}) \unif \cst{f}^{100} \, (H \, \cst{b})$, where the exponents
denote repeated application. It is easy to see that this problem can be reduced
to $G \, \cst{a} \unif H \, \cst{b}$. However, the JP procedure will wastefully
retraverse the common context $ \cst{f}^{100} [\;] $ after applying each new
substitution. Since the JP procedure must apply substitutions to the variables occurring in the
common context above the position of disagreement, it cannot be easily adapted to
eagerly decompose unification pairs. By contrast, our procedure is designed to decompose
the pairs eagerly, never traversing a common context twice.

Last, the JP procedure does not allow to apply substitutions and
$\beta$-reduce lazily.
The rules of simpler procedures (e.g., first-order
\cite{hv-09-unifalgs} and pattern unification \cite{tn-93-patterns}) depend only on
the heads of the unification pair.
Thus, to
determine the next step, implementations of these procedures need to
substitute and
$\beta$-reduce only until the heads of the current unification
pair are not mapped by the substitution and are not $\lambda$-abstractions. 
Since the JP procedure is not based on the decomposition of unification pairs, 
it is unfit for
optimizations of this kind.
We designed our procedure to allow for this optimization.

% In Section \ref{sec:unif:indexing} we discuss a higher-order extension of fingerprint indexing,
% a technique to filter out terms that are not unifiable with a given
% query term from a set of terms. 
To more efficiently find terms (in a large term set) that are unifiable with a given query term,
we developed a higher-order extension of fingerprint indexing \cite{ss-12-fp-indexing}
(Section \ref{sec:unif:indexing}).
We implemented our procedure, several oracles,
and the fingerprint index in
the Zipperposition prover (Section \ref{sec:unif:implementation}). Since a
straightforward implementation of the JP procedure already existed in
Zipperposition, we used it as a baseline to evaluate the performance of our
procedure (Section \ref{sec:unif:evaluation}). The results show substantial
performance improvements.

% This invited article is an extended version of our FSCD-2020 paper \cite{vbn-21-unif}.
% Most notable extension is the Section \ref{sec:unif:proof-of-completeness}, which gives
% the detailed proof of completeness of our new procedure. In addition, 
% we give proofs for all the unproved statements from the paper, expand the examples 
% and provide more detailed explanations.


\section{Background}
\label{sec:unif:background}

Our setting is the simply typed $\lambda$-calculus. Unless mentioned otherwise,
we use the same notation as laid out in Chapter~\ref{ch:pre}. Additional notions
are introduced as follows.

\emph{Parameters} and \emph{body} for any term $\param[\seq x]s$ are defined to
be $\seq x$ and $s$ respectively, where $s$ is not a $\lambda$-abstraction. The
\emph{size} of a term is inductively defined as $\text{size}(F) = 1$;
$\text{size}(x) = 1$; $\text{size}(\cst{f}) = 1$; $\text{size}(s\,t) =
\text{size}(s) + \text{size}(t)$; $\text{size}(\lambda x.\, s) = \text{size}(s)
+ 1$. A term is in \emph{head normal form} ({\em hnf}) if it is of the form
$\lambda \overline{x}.\,a\,\overline{t}$, where $a$ is a free variable, bound
variable, or a constant. In this case, $a$ is called the \emph{head} of the
term. Note that this relaxes the condition that the term needs to be in
$\beta$-normal form to determine its head. A term is called flex or rigid if its
head is flex or rigid, respectively. By $\nf{s}{\textsf{h}}$ we denote the term
obtained from a term $s$ by repeated $\beta$-reduction of the leftmost outermost
redex until it is in hnf. Unless stated otherwise, we view terms syntactically,
as opposed to $\alpha\beta\eta$-equivalence classes. The common context
$\mathcal{C}(s,t)$ of two $\eta$-long $\beta$-reduced terms $s$~and~$t$ of the
same type is defined inductively as follows, assuming that $a \not= b$:
$\mathcal{C}(\lambda x.\, s, \lambda y.\, t) = \lambda x.\,
\mathcal{C}(s,\{y\mapsto x\}t)$; $\mathcal{C}(a\,\overline{s}_m,
b\,\overline{t}_n) = \square$; $\mathcal{C}(a\,\overline{s}_m,
a\,\overline{t}_m) = a\,\mathcal{C}(s_1,t_1)\,\ldots\,\mathcal{C}(s_m,t_m)$.
Unless otherwise stated, we take unification constraint $s \unif t$ to be an
unordered pair of two terms of the same type. To ease notation, we do not write
parentheses around application of substitutions to terms (or other objects containing terms);
in other words we shorten $\sigma(\theta(\varrho(s)))$ to $\sigma\theta\varrho \, s$.

\ourpara{Remark}
We use the definition of a CSU from Sect.~\ref{sec:pre:unif} because 
JP's definition of a CSU, which we have adopted in our earlier work \cite{vbn-21-unif}, is flawed.
JP's definition does not employ the notion of auxiliary variables,
but instead requires $\rho X = \theta\sigma X$ for all variables mapped by $\rho$.
This is problematic because nothing prevents $\rho$ from mapping the auxiliary variables.
For example, $\sigma = \{F \mapsto \lambda x y.\> G\>y\}$ is supposed
to be an MGU for $F\>\cst{a}\>\cst{c} \unif F\>\cst{b}\>\cst{c}$.
But for the unifier $\rho = \{F \mapsto \lambda x y.\> y,\> G \mapsto \lambda x.\> \cst{d}\}$,
without the notion of auxiliary variables,
there exists no appropriate substitution $\theta$
because $\rho G = \theta\sigma G$ requires $\theta G = \lambda x.\> \cst{d}$
and $\rho F = \theta\sigma F$ requires $\theta G = \lambda x.\> x$.



\section{The Unification Procedure}
\label{sec:unif:the-unification-procedure}

To unify two terms $s$ and $t$, our procedure builds a tree as follows. The
nodes of the tree have the form $(E,\sigma)$, where $E$ is a multiset of
unification constraints $\{(s_1 \unif t_1), \ldots, (s_n~\unif~t_n)\}$ and
$\sigma$ is the substitution constructed up to that point.  The root node is $(\{s~\unif~t\}, \makeop{id})$, where $\makeop{id}$ is the identity
substitution. The tree is then constructed applying the transitions listed
below. The leaves of the tree are either failure nodes  $\bot$ or substitutions
$\sigma$. Ignoring failure nodes, the set of all substitutions in the leaves
forms a complete set of unifiers for $s$ and $t$. More generally, our procedure can be used to
unify a multiset $E$ of constraints by making the root of the unification tree
$(E, \makeop{id})$.

The procedure requires an infinite supply of fresh free variables. These fresh
variables must be disjoint from the variables occurring in the initial multiset
$E$. Whenever a transition $(E,\sigma) \newunifarrow (E',\sigma')$ is made, all
fresh variables used in $\sigma'$ are removed from the supply and cannot be used
again as fresh variables.

\looseness=-1
The transitions are parametrized by a mapping~$\mathcal{P}$ that assigns a set
of substitutions to a unification pair; this mapping abstracts the
concept of unification rules present in other unification procedures. Moreover,
the transitions are parametrized by a selection function~$S$ mapping a multiset
$E$ of unification constraints to one of those constraints $S(E) \in E$, the
\emph{selected} constraint in $E$. The transitions, defined as follows, are only
applied if the $\selected{\text{grayed}}$ constraint is selected.
\begin{description}[labelwidth=\widthof{\rm\textsf{Normalize$_{\alpha\eta}$}}]
  \item[\rm\unifrulename{Succeed}]
      $(\varnothing, \sigma) \newunifarrow \sigma$ 
  % \smallskip
  \item[\rm\unifrulename{Normalize$_{\alpha\eta}$}]
      $(\{\selected{\lambda \overline{x}_m. \, s \unif \lambda \overline{y}_n. \, t}\}\uplus E, \sigma) 
      \newunifarrow 
      (\{\lambda \overline{x}_m. \, s \unif \lambda \overline{x}_m. \, t'\, x_{n+1} \ldots x_m\}\uplus E, \sigma)$\\
      % where $n \geq m$ and $t'=\{y_1\mapsto x_1,\ldots,y_m\mapsto x_m\}t$
      where $m \geq n$, $\overline{x}_m \not= \overline{y}_n$, and $t'=\{y_1\mapsto x_1,\ldots,y_n\mapsto x_n\}t$
  % \smallskip
  \item[\rm\unifrulename{Normalize$_\beta$}]
      $(\{\selected{\lambda \overline{x}. \, s \unif \lambda \overline{x}. \, t}\}\uplus E, \sigma) 
      \newunifarrow 
      (\{\lambda \overline{x}. \, \nf{s}{\textsf{h}} \unif \lambda \overline{x}. \, \nf{t}{\textsf{h}}\}\uplus E, \sigma)$\\
      where $s$ or $t$ is not in hnf
      % \smallskip
  \item[\rm\unifrulename{Dereference}]
      $(\{\selected{\lambda \overline{x}. \, F \, \overline{s} \unif \lambda \overline{x}. \, t}\}\uplus E, \sigma)
      \newunifarrow 
      (\{\lambda \overline{x}. \, (\sigma F) \,\overline{s} \unif \lambda \overline{x}. \, t\}\uplus E, \sigma)$\\
      where none of the previous transitions apply and $F$ is mapped by $\sigma$
  % \smallskip
      \item[\rm\unifrulename{Fail}]
      $(\{\selected{\lambda \overline{x}. \, a \, \overline{s}_m \unif \lambda \overline{x}. \, b \, \overline{t}_n}\}
      \uplus E, \sigma) \newunifarrow \bot$\\
      where none of the previous transitions apply, and
      $a$ and $b$ are different rigid heads
  % \smallskip
  \item[\rm\unifrulename{Delete}]  $(\{\selected{s \unif s}\}\uplus E, \sigma) \newunifarrow (E, \sigma)$\\
      where none of the previous transitions apply
  % \smallskip 
  \item[\rm\unifrulename{OracleSucc}]
  $(\{\selected{s \unif t}\}\uplus E, \sigma) \newunifarrow (E, \varrho \sigma)$\\
      where none of the previous transitions apply, 
      some oracle found a finite CSU $U$ for $\sigma s \unif \sigma t $ using fresh auxiliary variables,
      and $\varrho \in U$; if multiple oracles found a CSU, only one of them is considered
  % \smallskip
  \item[\rm\unifrulename{OracleFail}]
  $(\{\selected{s \unif t}\}\uplus E, \sigma) \newunifarrow \bot$\\
  where none of the previous transitions apply, 
  and some oracle determined $\sigma s \unif \sigma t$ has no solutions
  % \smallskip
  \item[\rm\unifrulename{Decompose}]
      $(\{\selected{\lambda \overline{x}. \, a \, \overline{s}_m \unif \lambda \overline{x}. \, a \, \overline{t}_m}\}
      \uplus E, \sigma) 
      \newunifarrow 
      ( \{s_1 \unif t_1, \ldots ,s_m \unif t_m\} \uplus E,
      \sigma)$\\
      where none of the transitions \unifrulename{Succeed} to \unifrulename{OracleFail} apply
  % \smallskip
  % \item [\rm\textsf{Eliminate}] 
  %     $(\{\selected{\lambda \overline{x}_n. \, F \, \overline{s}_m \unif \lambda \overline{x}_n. \, F \, \overline{t}_m}\}
  %     \uplus E, \sigma)
  %     \newunifarrow
  %     (\{s_{j_{1}}\unif t_{j_{1}},\ldots,s_{j_{k}}\unif t_{j_{k}}\} \uplus E,\varrho \sigma)$\\
  %     where none of the transitions \textsf{Succeed} to \textsf{OracleFail} apply,
  %     $ 1 \leq j_1 < \cdots < j_k \leq n$, and
  %     % $(j_{i})_{i=1}^{k}$ is a subsequence
  %     % of $\left(i\right)_{i=1}^{m}$, and
  %     $\varrho = \{ F \mapsto \param G\, x_{j_{1}} \ldots x_{j_{k}} \}$ with $G$ a fresh variable
  % \smallskip
  \item[\rm\textsf{Bind}]
      $(\{\selected{s \unif t}\} 
      \uplus E, \sigma) 
      \newunifarrow 
      (\{s \unif t\}
      \uplus E, \rho \sigma)$\\
      where none of the transitions \textsf{Succeed} to \textsf{OracleFail} apply,
      and $\rho\in\mathcal{P}(s \unif t)$.
\end{description}


The transitions are designed so that only \unifrulename{OracleSucc},
\unifrulename{Decompose}, and \unifrulename{Bind} can introduce parallel branches in the
constructed tree. \unifrulename{OracleSucc} can introduce branches
using different unifiers of the CSU, \unifrulename{Bind} can introduce branches 
using different substitutions in $\mathcal{P}$, and \unifrulename{Decompose}
can be applied in parallel with \unifrulename{Bind}. 

The form of the rules \unifrulename{OracleSucc} and \unifrulename{Bind} is similar: both
extend the current substitution. However,
they are designed following different principles. \unifrulename{OracleSucc} solves
the selected unification constraint using an efficient algorithm applicable only
to certain classes of terms. On the other hand, \unifrulename{Bind} is applied to
explore the whole search space for any given constraint. These rules are separated
in two to make \unifrulename{Bind} applicable only if \unifrulename{OracleSucc}
(or \unifrulename{OracleFail}) is not, so that possible solutions (or failures) are
detected early.

Our approach is to apply substitutions and $\alpha\beta\eta$-normalize terms
lazily. In this context, laziness means that the transitions
\unifrulename{Normalize$_{\alpha\eta}$}, \unifrulename{Normalize$_{\beta}$}, and
\unifrulename{Dereference} partially normalize and partially apply the constructed
substitution just enough to ensure that the heads are the ones we would get if
the substitution was fully applied and the term was fully normalized. 
Additionally, the transitions that modify the constructed substitution,
\unifrulename{OracleSucc} and \unifrulename{Bind}, do not apply that substitution to the
unification pairs directly, but only extend it with a new binding. To support
lazy dereferencing, these rules must maintain the
invariant that all substitutions are idempotent. The invariant is easily
preserved if the substitution $\varrho$ from the definition of
\unifrulename{OracleSucc} and \unifrulename{Bind} is itself idempotent and no variable
mapped by $\sigma$ occurs in $\varrho F$, for any variable $F$ mapped by
$\varrho$.

The \textsf{OracleSucc} and \textsf{OracleFail} transitions
invoke oracles, such as pattern unification,
to compute a CSU faster,
produce fewer redundant unifiers, and
discover nonunifiability earlier.
In some cases, addition of oracles lets the procedure terminate more often.

In the literature, oracles are usually stated under the assumption that their
input belongs to the appropriate fragment. To check whether a unification
constraint is inside the fragment, we need to fully apply the substitution and
$\beta$-normalize the constraint.
% To use an off-the-shelf oracle
% in our procedure, we need to wrap it inside an algorithm that applies the currently
% built substitution $\sigma$, normalizes terms and checks whether terms belong to the
% fragment the oracle handles, before invoking the oracle. Applying substitution and
% normalizing eagerly to invoke an oracle can inadvertently cancel the effects of
% lazy normalization built into our procedure.
To avoid these expensive operations and enable efficient oracle integration, 
oracles must be redesigned to lazily discover whether the terms
belong to their fragment.  
% Usually, the operations
% oracles perform can clearly be separated into those that do and those that do
% not depend on whether the terms belong to the fragment oracle manages. 
Most oracles contain a decomposition operation which requires only a partial application of
the substitution and only partial $\beta$-normalization. If one of the constraints resulting from decomposition is not in
the fragment, the original problem is not in the fragment. 
This allows us to detect that the problem is not in the fragment without fully applying
the substitution and $\beta$-normalizing.
% Thus, without checking if all subterms are in the fragment, the oracle
% can check if the term heads are equal after getting terms in hnf. This
% effectively delegates the fragment check to the
% other operations of the oracle. 

%  If some
% other transitions applies, it is the only applicable transition. Transitions
% \textsf{Bind} and \textsf{OracleSucc} may be applicable with different partial
% bindings or unifiers.

%
%Mention log on variables?
%
The core of the procedure lies in the \textsf{Bind} step, parameterized
by the mapping~$\mathcal{P}$ that determines which substitutions (\emph{bindings}) 
to create. The bindings are defined as follows:
%
% \begin{addmargin}[0.5em]{0.5em}
%\vspace{-0.2em}
\begin{description}[itemsep=1\jot]
    \item[JP-style projection for $F$] Let $F$ be a free variable of
    type $\alpha_1 \rightarrow \cdots \rightarrow \alpha_n \rightarrow \beta$, where
    some $\alpha_i$ is equal to $\beta$ and $n > 0$. Then the JP-style projection binding is
        \[F \mapsto \lambda \overline{x}_n.\,x_i\]
    \item[Huet-style projection for $F$] Let $F$ be a free variable of type $\alpha_1
    \rightarrow \cdots \rightarrow \alpha_n \rightarrow \beta$, where some $\alpha_i
    = \gamma_1 \rightarrow \cdots \rightarrow \gamma_m \rightarrow \beta$, $n > 0$ and $m \geq 0$. Huet-style projection is
    \[F \mapsto \lambda \overline{x}_n. \, x_i \, (F_1 \, \overline{x}_n) \, \ldots \, (F_m \, \overline{x}_n)\]
    where the fresh free variables $\overline{F}_m$ and bound variables $\overline{x}_n$ are of appropriate types.
    \item[Imitation of $a$ for $F$] 
    \looseness=-1 Let $F$ be a free variable of type $\alpha_1 \rightarrow
    \cdots \rightarrow \alpha_n \rightarrow \beta$ and $a$ be a free variable or a constant
    of type $\gamma_1 \rightarrow \cdots
    \rightarrow \gamma_m \rightarrow \beta$ where $n,m \geq 0$. The imitation binding is
    \[F \mapsto \lambda \overline{x}_n. \, a \, (F_1 \, \overline{x}_n) \ldots
    (F_m \, \overline{x}_n)\] where the fresh free variables $\overline{F}_m $ and
    bound variables $\overline{x}_n$ are of appropriate types.
    \item[Elimination for $F$] Let $F$ be a free variable of type $\alpha_1 \rightarrow
    \cdots \rightarrow \alpha_n \rightarrow \beta$, where $n >0$. In addition, let $1 \leq j_1 < \cdots < j_i \leq n$ and $i<n$. Elimination
    for the sequence $(j_k)_{k=1}^i$ is
    \[ F \mapsto \lambda \overline{x}_n. \, G \, x_{j_1} \, \ldots \, x_{j_i}\]
    where the fresh free variable $G$ as well as all $x_{j_k}$ are of appropriate type.
    We call fresh variables emerging from this binding in the role of $G$ 
    \emph{elimination variables}.
    \item[Identification for $F$ and $G$] Let $F$ and $G$ be different free variables. Furthermore, let
    the type of $F$ be $\alpha_1 \rightarrow \cdots \rightarrow \alpha_n
    \rightarrow \beta$ and the type of $G$ be $\gamma_1 \rightarrow \cdots
    \rightarrow \gamma_m \rightarrow \beta$, where $n,m\geq 0$. Then, the identification binding binds
    $F$ and $G$ with
    \begin{equation*}
        F \mapsto \lambda \overline{x}_n. \, H \, \overline{x}_n \, (F_1 \, \overline{x}_n)
        \ldots (F_m \, \overline{x}_n) \quad
        G \mapsto \lambda \overline{y}_m. \,
        H \, (G_1 \, \overline{y}_m) \ldots (G_n \, \overline{y}_m) \, \overline{y}_m
    \end{equation*}
    \looseness=-1
    where the fresh free variables $H,\overline{F}_m,\overline{G}_n$ and bound
    variables $\overline{x}_n$,$\overline{y}_m$ are of appropriate types. 
    Fresh variables from this binding with the role of $H$ are called \emph{identification variables}.
    \item[Iteration for $F$]
    \looseness=-1 
    Let $F$ be a free variable of the type $\alpha_1 \rightarrow \cdots
    \rightarrow \alpha_n \rightarrow \beta_1$ and let some $\alpha_i$ be the
    type $\gamma_1 \rightarrow \cdots \rightarrow \gamma_m \rightarrow
    \beta_2$, where $n>0$ and $m\geq 0$. Iteration for $F$ at $i$ is  
    \[ F \mapsto \lambda \overline{x}_n.\,H\,\overline{x}_n \, (\lambda
    \overline{y}.\,x_i \, (G_1 \, \overline{x}_n \, \overline{y}) \ldots (G_m \,
    \overline{x}_n \, \overline{y}) )\]
    The free variables $H$ and $G_1, \ldots,
    G_m$ are fresh, and $\overline{y}$ is an arbitrary-length sequence of bound
    variables of arbitrary types. All new variables are of
    appropriate type. Due to indeterminacy of $\overline{y}$, this step is
    infinitely branching. 
    

\end{description}

\noindent
The following mapping $\mathcal{P}_\mathsf{c}(\lambda \overline{x}.
\, s \unif \lambda \overline{x}. \, t)$ is used as the parameter $\mathcal{P}$ of the procedure:
\begin{itemize}
  \setlength\itemsep{1\jot}

    \item If the constraint is rigid-rigid, $\mathcal{P}_\mathsf{c}(\lambda \overline{x}.
    \, s \unif \lambda \overline{x}. \, t) = \varnothing$.

    \item If the constraint is flex-rigid,
    let 
    $\mathcal{P}_\mathsf{c}(\lambda \overline{x}. \, F \,\overline{s} \unif \lambda \overline{x}. \, a\, \overline{t})$ 
    be 
    \begin{itemize}
        \item an imitation of $a$ for $F$, if $a$ is a constant, and
        \item all Huet-style projections for $F$, if $F$ is not an identification variable.
    \end{itemize}
    
    \item If the constraint is flex-flex and the heads are different,
    let 
    $\mathcal{P}_\mathsf{c}(\lambda \overline{x}. \, F \,\overline{s} \unif \lambda \overline{x}. \, G\, \overline{t})$ 
    be 
    \begin{itemize}
        \item all identifications and iterations for both $F$ and $G$, and
        \item all JP-style projections for non-identification variables among $F$ and $G$.
    \end{itemize}
    
    \item If the constraint is flex-flex and the heads are identical,
    we distinguish two cases:
    \begin{itemize}
        \item if the head is an elimination variable, 
        $\mathcal{P}_\mathsf{c}(\lambda \overline{x}.\, s \unif \lambda \overline{x}.\, t ) = \varnothing$;
        \item otherwise, let 
        $\mathcal{P}_\mathsf{c}(\lambda \overline{x}. \, F \,\overline{s} \unif \lambda \overline{x}. \, F\, \overline{t})$ 
        be all iterations for $F$ at arguments of functional type and
        all eliminations for~$F$.
    \end{itemize}
\end{itemize}

\ourpara{Comparison with the JP Procedure}
%
% Our procedure is based on the JP procedure, which is the best known procedures that
% enumerates a CSU \cite{jp-76-unif}. Unlike other complete unification procedures
% \cite{sg-89-unif}, it does not try to blindly guess the solution of flex-flex
% unification constraints, but enumerates them systematically. 
%
\looseness=-1
The JP procedure enumerates unifiers by constructing a search tree with nodes of the form $(s \unif t, \sigma)$, where $s \unif t$ is the current unification
problem and $\sigma$ is the substitution built so far. 
The initial node consists of the input problem and the
identity substitution. Success nodes are nodes of the form $(s \unif s, \sigma)$.
The set of all substitutions contained in the success nodes form a CSU.

To determine the
child nodes of a node $(s \unif t, \sigma)$, the procedure  computes the common context $C$ of $s$ and $t$,
yielding term pairs $(s_1, t_1), \ldots, (s_n, t_n)$, called \emph{disagreement
pairs}, such that $s = C[s_1,\ldots,s_n]$ and $t = C[t_1,\ldots,t_n]$. It
chooses one of the disagreement pairs $(s_i, t_i)$.
Depending on the context $C$ and the chosen disagreement pair $(s_i, t_i)$,
it determines a set of bindings $\mathcal{P}_\mathsf{JP}(C,s_i, t_i)$.
For each of the bindings $\rho\in\mathcal{P}_\mathsf{JP}(C,s_i, t_i)$, it creates a child node
$(\nf{(\varrho{s})}{\beta\eta} \unif \nf{(\varrho{t})}{\beta\eta}, \varrho\sigma)$,
where $\nf{u}{\beta\eta}$ denotes a
$\beta\eta$-normal form of a term $u$. 

The set of bindings $\mathcal{P}_\mathsf{JP}(C,s_i, t_i)$
is based on the heads of $s_i$ and $t_i$, and the free variables occurring above
$s_i$ and $t_i$ in $C$. The set $\mathcal{P}_\mathsf{JP}(C,s_i, t_i)$ contains
\begin{itemize}
  \item all JP-style projections for free variables that are heads of $s_i$ or $t_i$;%
\footnote{In JP's formulation of projection, they explicitly mention that the projected argument must be of base type. 
In our presentation, this follows from $\beta$ being of base type by the convention introduced in Sect.~\ref{sec:pre:hol}.}
  
  \item an imitation of $a$ for $F$ if a free variable $F$ is the head of $s_i$ and a free
  variable or constant $a$ is the head of $t_i$ (or vice versa);
  \item all eliminations for free variables occurring above the
  chosen disagreement pair
  eliminating only the argument containing the disagreement pair;
  \item an identification for the heads of $s_i$ and $t_i$ if they are both free variables; and
  \item all iterations for the heads of $s_i$ and
  $t_i$ if they are free variables, and for all free variables occurring above the disagreement pair.%
\footnote{In JP's formulation of iteration, it is not immediately obvious whether they intend to require iteration of arguments of base type.
However, their Definition 2.4 \cite{jp-76-unif} shows that they do.}
\end{itemize}

Architecturally, the most noticeable difference between the JP procedure and ours is
the representation of the problem: The JP procedure works on a single constraint,
while our procedure maintains a multiset of constraints. At a first glance, this
is a merely presentational change. However, it has consequences for termination,
performance, and redundancy of the procedure.

\looseness=-1
Since the JP procedure never decomposes the common context of its only constraint, it allows
iteration or elimination to be applied at a free variable above the disagreement
pair, even if  bindings were already applied below that free variable. This can
lead to many different paths to the same unifier. In contrast, our procedure
makes the decision which binding to apply to a flex-flex pair with the same
head as soon as it is observed. Also, it explores the possibility of not
applying a binding and decomposing the pair. In either way, the flex-flex pair
is never revisited, which improves the performance and returns fewer redundant
unifiers. We show that this restriction prunes the search space without
influencing the completeness.

Our procedure makes the choice of child nodes based only on the heads of the
chosen unification constraint. In contrast, the JP procedure tracks all the
variables occurring in the common context. Thus, lazy normalization and lazy
variable substitution cannot be integrated in the JP procedure a straightforward
fashion. Moreover, as it does not feature a rule similar to
\unifrulename{Decompose}, it always retraverses the already unified part of the
problem, resulting in poor performance on deep terms.


\looseness=-1
% The JP procedure can be modified to solve the preunification problem by making
% it never choose flex-flex disagreement pairs, ignore the disagreement pairs
% below a free variable, and terminate with a preunifier when no flex-rigid pair
% remains. However, such a procedure would be less efficient than Huet's procedure
% because it would use the iteration binding instead of Huet-style projection to
% solve flex-rigid pairs. Our procedure applies Huet-style projections on
% flex-rigid pairs, which results in two important improvements over the JP
% procedure. First, our procedure terminates more often than the JP procedure
% because Huet-style projections cause only a finite branching, whereas iteration
% causes an infinite branching. Second, when our procedure is modified to solve
% the preunification problem by never selecting flex-flex pairs and stopping when
% only flex-flex pairs remain, it becomes an optimized variant of Huet's procedure
% that supports oracles as well as lazy substitution and $\beta$-reduction.
One of the main drawbacks of the JP procedure is that it features a highly
explosive, infinitely branching iteration rule. This rule is a more general
version of Huet-style projection. Its universality enables finding elements of
CSU for flex-flex pairs, for which Huet-style projection does not suffice.
However, the JP procedure applies iteration indiscriminately on both flex-flex and
flex-rigid pairs. We discovered that our procedure remains complete if iteration
is applied only on flex-flex pairs, and Huet-style projection only on
flex-rigid ones. This helps our procedure terminate more often than the JP
procedure. As a side-effect, the restriction of our procedure to
the preunification problem is a graceful generalization of Huet procedure, with
additional improvements such as oracles, lazy substitution, and lazy
$\beta$-reduction.


\looseness=-1
The bindings of our procedure contain further optimizations that are absent in
the JP procedure: The JP procedure applies eliminations for one parameter
at a time, yielding multiple paths to the same unifier. It applies imitations to
flex-flex pairs, which we found to be unnecessary. Similarly, we found out that
tracking which rules introduced which variables can avoid computing redundant
unifiers: It is not necessary to apply iterations and eliminations on
elimination variables, and projections on identification variables.

\ourpara{Examples}
We present some examples that demonstrate advantages of our procedure.
The displayed branches of the constructed trees are not necessarily exhaustive.
We abbreviate 
JP-style projection as \textsf{JP\,Proj},
imitation as \textsf{Imit},
identification as \textsf{Id},
\unifrulename{Decompose} as \textsf{Dc}, 
\unifrulename{Dereference} as \textsf{Dr},
$\textsf{Normalize}_\beta$ as $\textsf{N}_\beta$,
and
\textsf{Bind} of a binding $x$ as $\textsf{B(}x\textsf{)}$.
Transitions of the JP procedure are denoted by $\Longrightarrow$.
For the JP transitions we implicitly apply the generated bindings and fully
normalize terms, which significantly shortens JP derivations.

\begin{exa}
The JP procedure does not terminate on the problem $G \unif \cst{f} \, G$:
\[(G \unif \cst{f} \, G, \makeop{id}) 
\overset{\textsf{Imit}}{\jpunifarrow}
(\cst{f}\,G' \unif \cst{f}^2 \, G', \sigma_1) 
\overset{\textsf{Imit}}{\jpunifarrow}
(\cst{f}^2 \, G'' \unif \cst{f}^3 \, G'', \sigma_2) 
\overset{\textsf{Imit}}{\jpunifarrow}\cdots\]
where $\sigma_1 = \{G \mapsto  \lambda x. \, \cst{f} \, G' \}$
and $\sigma_2 = \{G' \mapsto  \lambda x. \, \cst{f} \, G'' \}\sigma_1$.
By including any oracle that supports the first-order occurs check, such as the
pattern oracle or the fixpoint oracle described in Section \ref{sec:unif:implementation},
our procedure gracefully generalizes first-order unification: 
\[(\{G \unif \cst{f} \, G\}, \makeop{id}) 
\overset{\textsf{OracleFail}}{\newunifarrow}
\bot
\]
\end{exa}

\begin{exa}
The following derivation illustrates the advantage of the \textsf{Decompose} rule.
\begin{align*}
&
(\{\cst{h}^{100} \, (F \, \cst{a}) \unif \cst{h}^{100} \, (G \,\cst{b})\}, \makeop{id}) 
\overset{\textsf{Dc}^{100}}{\newunifarrow}
(\{F \, \cst{a} \unif G \,\cst{b}\}, \makeop{id}) 
\overset{\textsf{B(Id)}}{\newunifarrow}
(\{F \, \cst{a} \unif G \,\cst{b}\}, \sigma_1) 
\\ &
\overset{\textsf{Dr}+\textsf{N}_\beta}{\newunifarrow}
(\{H \, \cst{a}\, (F' \, \cst{a}) \unif H \, (G'\,\cst{b}) \,\cst{b}\}, \sigma_1)
\overset{\textsf{Dc}}{\newunifarrow}
(\{\cst{a} \unif G'\,\cst{b}, F' \, \cst{a} \unif \cst{b}\}, \sigma_1) 
\\ &
\overset{\textsf{B(Imit)}}{\newunifarrow}
(\{\cst{a} \unif G'\,\cst{b}, F' \, \cst{a} \unif \cst{b}\}, \sigma_2) 
\overset{\textsf{Dr}+\textsf{N}_\beta}{\newunifarrow}
(\{\cst{a} \unif \cst{a}, F' \, \cst{a} \unif \cst{b}\}, \sigma_2) 
\overset{\textsf{Delete}}{\newunifarrow}
(\{F' \, \cst{a} \unif \cst{b}\}, \sigma_2) \\
&\overset{\textsf{B(Imit)}}{\newunifarrow}
(\{F' \, \cst{a} \unif \cst{b}\}, \sigma_3) 
\overset{\textsf{Dr}+\textsf{N}_\beta}{\newunifarrow}
(\{\cst{b} \unif \cst{b}\}, \sigma_3) 
\overset{\textsf{Delete}}{\newunifarrow}
(\varnothing, \sigma_3) 
\overset{\textsf{Succeed}}{\newunifarrow}
\sigma_3 
\end{align*}
where 
$\sigma_1 = \{
    F\mapsto \lambda x. \, H \, x\, (F' \, x), 
    G\mapsto \lambda y. \, H \, (G' \, y) \, y
    \}$;
$\sigma_2 = \{
  G' \mapsto \lambda x. \, \cst{a}
  \}\sigma_1$; and
$\sigma_3 = \{
  F' \mapsto \lambda x. \, \cst{b}
  \}\sigma_2$.
%
The JP procedure produces the same intermediate substitutions $\sigma_1$ to $\sigma_3$,
but since it does not decompose the terms, it retraverses the common context
$\cst{h}^{100}\,[\;]$ at every step to identify the contained disagreement pair:
\begin{align*}
    &
    (\cst{h}^{100} \, (F \, \cst{a}) \unif \cst{h}^{100} \, (G \,\cst{b}), \makeop{id}) 
    \overset{\textsf{Id}}{\jpunifarrow}
    (\cst{h}^{100} \, (H \, \cst{a}\, (F' \, \cst{a})) \unif \cst{h}^{100} \, (H \, (G' \,\cst{b}) \,\cst{b}), \sigma_1) 
    \\ &
    \overset{\textsf{Imit}}{\jpunifarrow}
    (\cst{h}^{100} \, (H \, \cst{a}\, (F' \, \cst{a})) \unif \cst{h}^{100} \, (H \, \cst{a} \,\cst{b}), \sigma_2) 
    \overset{\textsf{Imit}}{\jpunifarrow}
    (\cst{h}^{100} \, (H \, \cst{a}\, \cst{b}) \unif \cst{h}^{100} \, (H \, \cst{a} \,\cst{b}), \sigma_3) 
    \overset{\textsf{Succeed}}{\jpunifarrow}
    \sigma_3
\end{align*}

\end{exa}

\begin{exa}
  Even when no oracles are used, our procedure performs better than the JP procedure on small, simple
  problems. Consider the problem $F \, \cst{a} \unif \cst{a}$, which has a two
  element CSU: $\{ F \mapsto \lambda x. \, x, F \mapsto \lambda x. \, \cst{a}
  \}$. Our procedure terminates, finding both unifiers:
\begin{align*}
    &
    (\{ F \, \cst{a} \unif \cst{a} \}, \makeop{id})
    \overset{\textsf{B(JP Proj)}}{\newunifarrow}
    (\{ F \, \cst{a} \unif \cst{a} \},  \{ F \mapsto \lambda x. \, x \} )
    \overset{\textsf{Dr}+\textsf{N}_\beta}{\newunifarrow}
    (\{ \cst{a} \unif \cst{a} \}, \{  F \mapsto \lambda x. \, x \})
    \\ &
    \overset{\textsf{Delete}}{\newunifarrow}
    (\varnothing, \{  F \mapsto \lambda x. \, x \})
    \overset{\textsf{Succeed}}{\newunifarrow}
    \{  F \mapsto \lambda x. \, x \}
    \\[1\jot]
    &
    (\{ F \, \cst{a} \unif \cst{a} \}, \makeop{id})
    \overset{\textsf{B(Imit)}}{\newunifarrow}
    (\{ F \, \cst{a} \unif \cst{a} \},  \{ F \mapsto \lambda x. \, \cst{a} \} )
    \overset{\textsf{Dr}+\textsf{N}_\beta}{\newunifarrow}
    (\{ \cst{a} \unif \cst{a} \}, \{  F \mapsto \lambda x. \, \cst{a} \})
    \\ &
    \overset{\textsf{Delete}}{\newunifarrow}
    (\varnothing, \{  F \mapsto \lambda x. \, \cst{a} \})
    \overset{\textsf{Succeed}}{\newunifarrow}
    \{  F \mapsto \lambda x. \, \cst{a} \}
\end{align*}

The JP procedure finds those two unifiers as well, but it does not terminate as
it applies iterations to $F$.


\end{exa}

\begin{exa}
The search space restrictions also allow us to prune some redundant unifiers. Consider the problem $F \, (G\,\cst{a}) \unif F \,
\cst{b}$, where $\cst{a}$ and $\cst{b}$ are of base type. 
Our procedure produces only one failing branch and the following two successful branches:
%
\begin{align*}
&
(\{ F\,(G\,\cst{a}) \unif F \, \cst{b} \}, \makeop{id})
\overset{\textsf{Dc}}{\newunifarrow}
(\{ G\,\cst{a} \unif \cst{b} \}, \makeop{id})
\overset{\textsf{B(Imit)}}{\newunifarrow}
(\{ G\,\cst{a} \unif \cst{b} \}, \{ G \mapsto \lambda x.\, \cst{b} \})
\\ &
\overset{\textsf{Dr}+\textsf{N}_\beta}{\newunifarrow}
(\{ \cst{b} \unif \cst{b} \}, \{ G \mapsto \lambda x.\, \cst{b} \})
\overset{\textsf{Delete}}{\newunifarrow}
(\varnothing, \{ G \mapsto \lambda x.\, \cst{b} \})
\overset{\textsf{Succeed}}{\newunifarrow}
\{ G \mapsto \lambda x.\, \cst{b} \}
\\[1\jot]
&
(\{ F\,(G\,\cst{a}) \unif F \, \cst{b} \}, \makeop{id})
\overset{\textsf{B(Elim)}}{\newunifarrow}
(\{ F\,(G\,\cst{a}) \unif F \, \cst{b} \}, \{ F \mapsto \lambda x.\, F' \})
\\ &
\overset{\textsf{Dr}+\textsf{N}_\beta}{\newunifarrow}
(\{ F' \unif F' \}, \{ F \mapsto \lambda x.\, F' \})
\overset{\textsf{Delete}}{\newunifarrow}
(\varnothing, \{ F \mapsto \lambda x.\, F' \})
\overset{\textsf{Succeed}}{\newunifarrow}
\{ F \mapsto \lambda x.\, F' \}
\end{align*}
%
The JP procedure additionally produces the following redundant unifier:
%
\begin{align*}
    &
    ( F\,(G\,\cst{a}) \unif F \, \cst{b} , \makeop{id})
    \overset{\textsf{JP Proj}}{\jpunifarrow}
    ( F \, \cst{a} = F \, \cst{b} , \{ G \mapsto \lambda x. \, x \})
    \\ &
    \overset{\textsf{Elim}}{\jpunifarrow}
    (F' = F' , \{ G \mapsto \lambda x. \, x, F \mapsto \lambda x. \, F' \})
    \overset{\textsf{Succeed}}{\jpunifarrow}
    \{ G \mapsto \lambda x. \, x, F \mapsto \lambda x. \, F' \}
\end{align*}
Moreover, the JP procedure does not terminate because an infinite number of iterations is 
applicable at the root. Our procedure terminates in this case since we only apply iteration
binding for non base-type arguments, which $F$ does not have.
\end{exa}

\ourpara{Pragmatic Variant} We structured our procedure so that most of
the unification machinery is contained in the \unifrulename{Bind} step. Modifying
$\mathcal{P}$, we can sacrifice completeness and obtain a pragmatic variant of
the procedure that often performs better in practice. 
%The simplest such modification
%is to limit the number or kind of bindings applied to a single constraint. 
Our preliminary experiments showed that using mapping $\mathcal{P}_\mathsf{p}$ defined as follows is a reasonable compromise between completeness and
performance:
\begin{itemize}
    \setlength\itemsep{1\jot}
    \item If the constraint is rigid-rigid, $\mathcal{P}_\mathsf{p}(\lambda \overline{x}.
    \, s \unif \lambda \overline{x}. \, t) = \varnothing$.
    \item If the constraint is flex-rigid,
    let 
    $\mathcal{P}_\mathsf{p}(\lambda \overline{x}. \, F \,\overline{s} \unif \lambda \overline{x}. \, a\, \overline{t})$ 
    be 
    \begin{itemize}
        \item an imitation of $a$ for $F$, if $a$ is a constant, and
        \item all Huet-style projections for $F$ if $F$ is not an identification variable.
    \end{itemize}
    
    \item If the constraint is flex-flex and the heads are different,
    let 
    $\mathcal{P}_\mathsf{p}(\lambda \overline{x}. \, F \,\overline{s} \unif \lambda \overline{x}. \, G\, \overline{t})$ 
    be
    \begin{itemize}
      \item an identification binding for $F$ and $G$, and
      \item all Huet-style projections for $F$ if $F$ is not an identification variable
  \end{itemize}
   
  \item If the constraint is flex-flex and the heads are identical, we distinguish two cases:
      \begin{itemize}
      \item if the head is an elimination variable, 
      $\mathcal{P}_\mathsf{p}(\lambda \overline{x}. \, F \,\overline{s} \unif \lambda \overline{x}. \, F\, \overline{t}) = \varnothing$;
      \item otherwise, let $\mathcal{P}_\mathsf{p}(\lambda \overline{x}. \, F \,\overline{s} \unif \lambda \overline{x}. \, F\, \overline{t})$
            be the set of all eliminations bindings for $F$. 
  \end{itemize}
\end{itemize}

The pragmatic variant of our procedure removes all iteration bindings to enforce finite branching. 
Moreover, it
imposes limits 
on the number of bindings applied,
counting the applications of bindings locally, per
constraint. 
It is useful to distinguish the Huet-style projection cases where
$\alpha_i$ is a base type (called \emph{simple projection}), which always
reduces the problem size, and the cases where $\alpha_i$ is a functional type (called \emph{functional
projection}). We limit the number applications of the following bindings:
functional projections, eliminations, imitations and identifications.
In addition,
a limit on the total number of applied bindings can be set. 
An elimination binding that removes $k$ arguments counts as $k$
elimination steps.
Due to these limits, the pragmatic variant 
terminates.

To fail as soon as any of the limits
is reached, the pragmatic variant employs an additional oracle.
If this oracle determines that the limits are reached and the constraint is of the form
$\lambda \overline{x}.\, F \, \overline{s}_m \unif \lambda \overline{x}.\, G \, \overline{t}_n$, it returns a \emph{trivial unifier} --
a substitution $\{ F \mapsto \lambda \overline{x}_m.\, H, G \mapsto \lambda \overline{x}_n.\, H \}$, where $H$ is a fresh variable; if the limits are reached and
the constraint is flex-rigid, the oracle fails; if the limits are not reached, 
it reports that terms are outside its fragment. The trivial unifier prevents the procedure from failing on
easily unifiable flex-flex pairs.

Careful tuning of each limit optimizes the procedure for a specific class of problems. 
For problems originating from proof assistants, shallow unification depth usually suffices. However, hard
hand-crafted problems often need
deeper unification.




\section{Proof of Completeness}
\label{sec:unif:proof-of-completeness}

\section{A New Decidable Fragment}
\label{sec:unif:solid-oracle}

\section{An Extension of Fingerprint Indexing}
\label{sec:unif:indexing}

\section{Implementation}
\label{sec:unif:implementation}

\section{Evaluation}
\label{sec:unif:evaluation}

\section{Discussion and Related Work}

\section{Conclusion}

