\chapter{Making Higher-Order Superposition Work}
\setheader{Making Higher-Order Superposition Work}
\label{ch:ho-techniques}

\authors{
    Joint work with\\
    Alexander Bentkamp,
    Jasmin Blanchette,
    Simon Cruanes,
    Visa Nummelin, and Sophie Tourret
}

\blfootnote{In this work I was the main designer of all presented techniques,
with the exception of inference streams which were designed by Sophie Tourret
and Alexander Bentkamp. Alexander Bentkamp and Jasmin Blanchette also discussed many
of the techniques with me and suggested important updates. Visa Nummelin worked on
implementation of FOOL preprocessing. Simon Cruanes is the original developer
of Zipperposition and provided us with invaluable knowledge.
}


\begin{abstract}
    Superposition is among the most successful calculi for first-order logic. Its
    extension to higher-order logic introduces new challenges such as infinitely
    branching inference rules, new possibilities such as reasoning about
    Booleans, and the need to curb the explosion of specific higher-order rules. We
    describe techniques that address these issues and extensively evaluate their
    implementation in the Zipperposition theorem prover. Largely thanks to their use,
    Zipperposition won the higher-order division of the CASC competition in 2020.  
  \end{abstract}
\newpage

\section{Introduction}
\label{sec:ho-tech:intro}

% %In recent decades,
% Superposition-based first-order automatic theorem provers
% have emerged as useful reasoning tools. They dominate at the annual CASC
% \cite{gs-2016-casc} theorem prover competition, having always won the
% first-order theorem division. They are also used as backends to proof assistants
% \cite{ck-18-coqhammer,ku-15-holyhammer,pb-12-sh}, automatic
% higher-order theorem provers \cite{sb-21-leo3}, and software verifiers
% \cite{fp-13-why3}.

% The superposition calculus has only recently been extended
% to higher-order logic (more precisely, extensional simple type theory
% \cite{henkin-1950-completeness}), resulting in
% \emph{\lsup} \cite{bbtvw-21-sup-lam}, which we developed
% together with Waldmann, as well as \emph{combinatory superposition}
% \cite{br-20-full-sup-w-combs} by Bhayat and Reger. Although these two
% calculi do not support an interpreted Boolean type,
% they can be extended by ad hoc rules \cite{our-bool-paper} that support
% most of the Boolean reasoning necessary in practice.

% Both higher-order superposition calculi were designed to gracefully
% extend first-order reasoning. As most steps in higher-order
% proofs tend to be essentially first-order, extending the most successful first-order
% calculus to higher-order logic seemed worth trying.
% Our first attempt at testing this idea was in 2019:
% Zipperposition~1.5, based on \lsup, finished third
% in the higher-order theorem division of CASC-27 \cite{gs-19-casc27},
% 12~percentage points behind the winner, the tableau prover Satallax 3.4 \cite{cb-2013-satallax}.

The landscape of higher-order proving techniques based on extension of efficient
first-order ones has tremendously expanded in the late 2010s and early 2020s. As
mentioned is Sect.~\ref{sec:pre:ho-sup-calculi} we have implemented three
higher-order calculi---\lfsup{}, \lsup{}, and \osup{}---which extend first-order superposition in a graceful way.
Bhayat and Reger also gracefully extended superposition to higher-order logic using
\textsf{SKBCI} combinators \cite{br-20-full-sup-w-combs}. Significant progress has been
made on SMT front as well \cite{brotb-19-ho-smt}.

In 2019 we tested if the idea of gracefully extending first-order provers to
higher-order logic really improves state of the art for the first time. We implemented \lsup{} \cite{bbtvw-21-sup-lam} in
Zipperposition~1.5 with basic heuristics and rudimentary extensions of the
calculus to deal with Booleans. It finished third at that year's higher-order
division of CASC competition \cite{gs-19-casc27}, 12~percentage points behind the
winner, the tableau prover Satallax 3.4 \cite{cb-2013-satallax}.

Studying the competition results, we found that higher-order tableaux have some
advantages over higher-order superposition. To bridge the gap, we developed
techniques and heuristics that simulate tableaux in the context of saturation.
We implemented them in Zipperposition~2, which took part at the higher-order
division of CASC \cite{gs-21-cascj10} in 2020. This time, our prover won the
division, proving 84\% of the problems, a whole 20~percentage points ahead of
the runner-up, Satallax 3.4.

In this chapter, we describe the main techniques that explain this reversal of
fortunes. They cover most parts of a modern higher-order theorem prover, from
preprocessing to additional calculus rules to heuristics to backend integration.
Compared to the previous chapter, in which we discussed rules used to treat
Boolean terms, in this chapter we use a newer version of Zipperposition, based
on a newer calculus. Instead of \lsup{} augmented with ad hoc Boolean rules, we
work with {\osup} \cite{bbtv-21-full-ho-sup}, a principled extension of
superposition to full higher-order logic, including an interpreted Boolean type.

Many higher-order problems extensively use symbol definitions to simplify
their representation. We describe several ways to exploit the definitions,
%of which the most successful is
such as turning them into rewrite rules (Sect.~\ref{sec:ho-tech:preprocessing}).
%Interesting patterns can be observed in various higher-order problem encodings.
%We show how we can exploit these to simplify problems (Sect.~\ref{sec:ho-tech:preprocessing}).
%
By working on formulas rather than clauses, tableau techniques take a more
holistic view of a higher-order problem.
Through its support for delayed clausification and, more generally,
calculus-level formula manipulation, \osup{} enables us to
simulate most successful tableau techniques in a saturating prover
(Sect.~\ref{sec:ho-tech:formulas}). This calculus also supports \emph{Boolean selection
functions}, a mechanism that allows us to choose on which Boolean subterms
to perform inferences first.
We implemented some Boolean selection functions and
evaluated them (Sect.~\ref{sec:ho-tech:bool-select}).

The main drawback of both $\lambda$-superposition variants compared with
combinatory superposition is that they rely on rules that enumerate possibly
infinite sets of unifiers. We describe a mechanism that interleaves infinitely
branching inferences with the standard saturation process
(Sect.~\ref{sec:ho-tech:infinite-branching}). The prover retains the same
behavior as before on first-order problems, smoothly scaling with increasing
numbers of higher-order clauses.
%
We also propose some heuristics to curb the explosion induced by highly
prolific calculus rules (Sect.~\ref{sec:ho-tech:explosiveness}).

Using first-order backends to finish the proof is common practice in
higher-order reasoning. Since \osup{} coincides with standard
superposition on first-order clauses, invoking backends may
seem redundant; yet Zipperposition is nowhere as efficient as E
\cite{scv-19-e23} or Vampire \cite{lkav-13-vampire}, so invoking a more
efficient backend does make sense. We describe how to achieve a balance
between allowing native higher-order reasoning and
delegating reasoning to a backend (Sect.~\ref{sec:ho-tech:backends}).
%
Finally, we compare Zipperposition~2 with other provers on all monomorphic
higher-order TPTP benchmarks \cite{gs-17-tptp} to perform a more extensive
evaluation than at CASC (Sect.~\ref{sec:ho-tech:comparison}). Our evaluation
corroborates the competition results.
