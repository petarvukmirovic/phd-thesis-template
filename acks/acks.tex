\chapter*{Preface}
\addcontentsline{toc}{chapter}{Preface}
\setheader{Preface}

When I downloaded the \LaTeX{} template to write this thesis, the following
sentence was present as a placeholder:
\begin{quote}
    Without a doubt, the acknowledgments are the most widely and most eagerly
    read part of any thesis.
\end{quote}

I do not know if that is true in general, but I have to admit I am guilty of
spending quite some time reading acknowledgments. I find that the reason why we
spend so much time reading acknowledgments is that we want to peek behind the
rigorous, formal shape that a PhD thesis is required to take. We also want to
see what PhD candidates have to say about their personal experiences of
finishing a PhD.

After reading many preface or acknowledgments chapters I realized that there
is a discrepancy between what research shows candidate's experience of finishing
a PhD is and what is actually written in these parts of theses. An
article by Levecque et al. \cite{labhg-17-phdstudents} states that one in two
PhD students experiences psychological distress, while one in three is at risk
of a common psychiatric disorder. In the light of this fact, I want to write
this chapter not only to the people whose help and work indebted me, but also to
(future) PhD students that, after looking up a detail in this thesis,
voyeuristically read this chapter (like I did many times for other theses).

In my first year of PhD, I was overwhelmed with fear and doubts. Did I make the
right choice by starting a PhD? Am I performing up to expectations? How many
papers should I publish? Should I read more papers? Did I \emph{actually} learn
anything? I could go on an on like this.

The fact that I was surrounded by very successful postdocs and PhD students on
later years of their studies made me feel like I was the only one having this
problem. To the students reading this chapter, I would very much like to assure
you you are not. 

What helped me go out of the self-doubting loop was actually taking action. From
the simplest ones, like saying ``I don't understand'' out loud instead of
silently nodding when I have a trouble following scientific discussion, to the
ones with more impact on my career like actively looking for projects I could
join while I did not have much work on my plate. As the years went on I felt
that I could more openly talk about my insecurities and I found out that many
people in my situation have them as well.

What made my PhD journey more enjoyable are undoubtedly the people I worked
with. I want to thank the whole Theoretical Computer Science group at the VU
which was there with me from the early days of my master studies. They helped me
organize my master programme so that I follow more theory courses which most
likely lead  me to start an academic career. I also want to thank other postdocs
and PhDs that studied during my PhD or are still studying. Most notably, I would
like to thank Alexander Bentkamp, one of the most gifted mathematicians I know.
I coauthored most of my papers with him, each of which was a pleasure to work
on. I would also like to thank all of my collaborators and people that suggested
improvements to the articles I wrote.

I want to thank my daily supervisor, Jasmin Blanchette. He always says that
the main product of a PhD is not a thesis but a person you come out as after four
years. With his guidance, I learned to approach things more rigorously, slower,
and with more attention. He also helped me tremendously  with my writing and
presentation skills. Even though he tried very hard, he did not teach me to use 
articles though, and capitulated by adding them to all of the articles I wrote.

Lastly, I would like to thank Martijn. He was besides me for the last four years
and helped me overcome many of the doubts and issues I mentioned. He also always
patiently waits until my medication or second glass of Grüner Veltliner kick in
(not at the same time). I also want to thank Jelisaveta and Tara, which
sometimes had to deal with me when medication or wine did not kick in. All my other friends,
in Serbia or the Netherlands, thank you for your support.

 

\begin{flushright}
{\makeatletter\itshape
    Petar \\
    Amsterdam, April 2022
\makeatother}
\end{flushright}


