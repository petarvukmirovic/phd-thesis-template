\chapter{SAT-Inspired Eliminations for Superposition}
\setheader{SAT-Inspired Eliminations for Superposition}
\label{ch:satfol}

\includeversion{noqle}
\excludeversion{qle}
\newcommand{\paper}[0]{chapter}

\authors{
    Joint work with\\
    Jasmin Blanchette,
    and Marijn J.H. Heule
}

\begin{abstract}
Optimized SAT solvers not only preprocess the clause set, they also transform it
during solving as inprocessing. Some preprocessing techniques have been
generalized to first-order logic with equality. In this \paper, we port
inprocessing techniques to work with superposition, a leading first-order proof
calculus, and we strengthen known preprocessing techniques. Specifically, we
look into elimination of hidden literals, variables (predicates), and blocked
clauses. Our evaluation using the Zipperposition prover confirms that the new
techniques usefully supplement the existing superposition machinery.
\end{abstract}

\blfootnote{In this work I was the main designer behind all the presented techniques.
Marijn Heule did weekly supervision and provided his knowledge of SAT solving to 
guide the design of all techniques. Jasmin Blanchette found the exact conditions
under which superposition remains complete when predicate elimination rule is added
to the calculus. I also implemented and evaluated all the techniques. 
}

\newpage
