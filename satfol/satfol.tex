\chapter{SAT-Inspired Eliminations for Superposition}
\setheader{SAT-Inspired Eliminations for Superposition}
\label{ch:satfol}

\excludeversion{conf}
\includeversion{rep}
\includeversion{noqle}
\excludeversion{qle}
\newcommand{\paper}[0]{chapter}

\newcommand{\eqlit}[2]{\ensuremath{#1 \eq #2}}
\newcommand{\neqlit}[2]{\ensuremath{#1 \noteq #2}}
% \newcommand{\eqneqlit}[2]{\ensuremath{#1 \mathrel{\dot{\eq}} #2}}
\newcommand{\pospredlit}[1]{{#1}}
\renewcommand{\neglit}[1]{\neg{\kern.2ex #1}}
\newcommand{\negpredlit}[1]{\neglit{#1}}
\newcommand{\arbpredlit}[1]{(\neg)\kern.2ex\pospredlit{#1}}
\newcommand{\cor}{\ensuremath{\mathrel{\lor}}}
\newcolumntype{d}{D{.}{.}{0}}



\authors{
    Joint work with\\
    Jasmin Blanchette
    and Marijn J.H. Heule
}

\begin{abstract}
Optimized SAT solvers not only preprocess the clause set, they also transform it
during solving as inprocessing. Some preprocessing techniques have been
generalized to first-order logic with equality. In this \paper, we port
inprocessing techniques to work with superposition, a leading first-order proof
calculus, and we strengthen known preprocessing techniques. Specifically, we
look into elimination of hidden literals, variables (predicates), and blocked
clauses. Our evaluation using the Zipperposition prover confirms that the new
techniques usefully supplement the existing superposition machinery.
\end{abstract}

\blfootnote{In this work I was the main designer behind all the presented techniques.
Marijn Heule did weekly supervision and provided his knowledge of SAT solving to 
guide the design of all techniques. Jasmin Blanchette found the exact conditions
under which superposition remains complete when predicate elimination rule is added
to the calculus. I also implemented and evaluated all the techniques. 
}

\newpage

\section{Introduction}
\label{sec:satfol:introduction}

Automated reasoning tools have become much more powerful in the last few
decades thanks to procedures such as conflict-driven clause learning (CDCL)
\cite{MSLM09HBSAT} for propositional logic and superposition
\cite{bg-94-superposition} for first-order logic with equality. However,
the effectiveness of these procedures crucially depends on how the input
problem is represented as a clause set. The clause set
can be optimized beforehand (\emph{preprocessing}) or during the execution of
the procedure (\emph{inprocessing}). In this \paper, we lift several preprocessing and
inprocessing techniques from propositional logic to clausal first-order logic
and demonstrate their usefulness in a superposition prover.

For many years, SAT solvers have used inexpensive clause simplification
techniques such as \emph{hidden literal} and \emph{hidden tautology elimination}
\cite{hjb-2010-cl-elim,hjb-2011-big-simplification} and \emph{failed literal
detection} \cite[Sect.~1.6]{jwf-1995-fld}. We generalize these techniques to
first-order logic with equality
(Sect.~\ref{sec:satfol:hidden-literal-based-elimination}). Since the generalization
involves reasoning about infinite sets of literals, we propose restrictions to
make them usable.

\emph{Variable elimination}, based on Davis--Putnam resolution \cite{dp-60-dp}, has
been studied in the context of both propositional logic
\cite{sp-04-niver,cs-00-zres} and quantified Boolean formulas (QBFs)
\cite{ab-2004-re}. The basic idea is to resolve all clauses with negative
occurrences of a propositional variable (i.e., a nullary predicate symbol) against
clauses with positive occurrences and delete the parent clauses. E\'en and
Biere \cite{eb-2005-satpreprocess} refined the technique to identify a subset
of clauses that effectively define a variable and use it to further optimize the
clause set. This latter technique, \emph{variable elimination by substitution},
has been an important preprocessor component in many SAT solvers since its
introduction in 2004.

\begin{sloppypar}
    Specializing second-order quantifier elimination
    \cite{go-1992-so-pred-elim,hjo-1996-scan}, Khasidashvili and Korovin~\cite{kk-2016-pe-fol} adapted variable elimination to preprocess first-order
    problems, yielding a technique we call \emph{singular predicate elimination}. We
    extend their work along two axes (Sect.~\ref{sec:satfol:predicate-elimination}): We
    generalize E\'en and Biere's refinement to first-order logic, resulting in
    \emph{defined predicate elimination}, and explain how both types of predicate
    elimination can be used during the proof search as inprocessing.    
\end{sloppypar}

\begin{qle}
It is well known that clauses containing \emph{pure literals}---predicate
literals whose root symbol occurs only with one polarity in the problem---can
be eliminated. We propose a generalization of this technique, under the name
\emph{quasipure literal elimination}, to allow the elimination of some bipolar
literals as well (Sect.~\ref{sec:satfol:quasipure-literal-elimination}).
\end{qle}

The last technique we study is \emph{blocked clause elimination} (Sect.\
\ref{sec:satfol:satisfiability-by-clause-elimination}). It is used in both SAT
\cite{jbh-10-BCE} and QBF solvers \cite{bls-11-bloqqer}. Its generalization to first-order
logic has produced good results when used as a preprocessor, especially on
satisfiable problems \cite{ksstb-2017-blockedfol}. We explore more ways to use
blocked clause elimination on satisfiable problems, including using it to
establish equisatisfiability with an empty clause set or as an inprocessing
rule. Unfortunately, we find that its use as inprocessing can compromise the
refutational completeness of superposition.


All techniques are implemented in the Zipperposition prover
(Sect.~\ref{sec:satfol:implementation}), allowing us to ascertain their usefulness
(Sect.~\ref{sec:satfol:evaluation}). The best configuration solves \NumberOK{160}
additional problems on benchmarks consisting of all 13\,495 first-order TPTP theorems
\cite{gs-17-tptp}. The raw experimental data are publicly available.%
\footnote{\url{https://doi.org/10.5281/zenodo.4552499}}\confrep{
More details, including all the proofs, can be found in a
technical report \cite{our-report}.}{}

\section{Preliminaries}
\label{sec:satfol:preliminaries}

% \ourpara{Clausal First-Order Logic}%
Our setting is many-sorted (i.e., many-typed), first-order
logic~\cite{jg-1987-logic-textbook} with interpreted equality and a
distinguished type (or sort) $o$. We introduced this logic in
Sect.~\ref{sec:pre:fol} and we use the same notation introduced in this section.
We write $\cst{f}^i(s)$ for the $i$-fold application of an unary symbol
$\cst{f}$ (e.g., $\cst{f}^3(x) = \cst{f}(\cst{f}(\cst{f}(x)))$) and call nullary
function symbols \emph{constants}. 

We assume the same notion of substitution introduced in Sect.~\ref{sec:pre:fol}.
We extend the notation by writing a substitution $\{x_1 \mapsto t_1,
\ldots,\allowbreak x_n \mapsto t_n\}$ shortly as $\{ \tupleempty{x} \mapsto \tupleempty{t} \}$.
A substitution is a \emph{variable renaming} if it is a
bijection from a set of variables to a set of variables.


As we are working with clausal logic, we assume the clausal structure as defined
in Sect.~\ref{sec:pre:clauses}. Recall that predicate literals  are encoded
as (dis)equations with equality. To simplify the notation in this chapter we write
predicate literals in unencoded form: positive literal $\eqlit{s}{\itrue}$ is written
as $s$ and negative literal $\neqlit{s}{\top}$ is written as $\negpredlit{s}$.
Literals $\eqlit{s}{t}$ where neither $s$ nor $t$ have the type $o$ are called \emph{functional}.
Given a literal $L$, we overload notation and write $\neglit{L}$ to
denote its complement. Clauses are often defined
as sets of literals, but superposition needs multisets; with multisets,
an instance $\sigmacl{C}$ always has the same number of literals as $C$, a most
convenient property. Given a clause set $N$, $\binset{N}$ denotes the subset of
its binary clauses: $\binset{N} = \{ L_1 \cor L_2 \mid L_1 \cor L_2 \in N \}$.

We assume the natural extensions of domain, valuation, interpretation and model
(as defined by Fitting \cite{mf-1996-fol}) from unsorted to many-sorted logic.
We also assume the notion of normal models from Sect.~\ref{sec:pre:fol} (with
corresponding notation) and extend it with the notion of \emph{cannonical
models}: A canonical model $\ourmodel$ is a normal model such that for every
element $d$ in one of $\ourmodel$'s domains there exists a ground term $t$ such
that $\ourmodel$ interprets $t$ as $d$. Canonical models generalize Herbrand
models \cite{mf-1996-fol} to first-order logic with equality: If a clause set is satisfiable, then
it is satisfiable in a canonical model.

% \ourpara{Superposition Provers}
%
% Superposition \cite{bg-94-superposition} is a calculus for clausal
% first-order logic that extends ordered resolution
% \cite{bg-01-resolution} with equality reasoning. It is
% refutationally complete: Given a finite, unsatisfiable clause set, it will
% eventually derive the empty clause. It is parameterized by a \emph{selection
% function} that influences which of a clause's literals are eligible as the target
% of inferences. Moreover, it is compatible with the \emph{standard redundancy
% criterion}, which can be used to delete a clause $C$ while preserving completeness of the calculus.

% \looseness=-1
% The redundancy criterion relies on an order $\succ$ that
% compares terms, literals, or clauses.
% The order is used to determine whether clauses can be deleted.
% If $N$ is ground, $C$ can be deleted
% if it is entailed by $\prec$-smaller clauses in $N.$ This definition is lifted
% to nonground sets $N$. The criterion can be used to delete a
% clause that is \emph{subsumed} by another clause (e.g.,
% $\cst{\cst{p}(\cst{a})} \cor \cst{q}$ by $\cst{p}(x)$) or
% to \emph{simplify} a clause $C$ into $C'$, which amounts to adding $C'$ and
% then deleting $C$ as redundant with respect to $N \cup \{C'\}$.
% Subsumption and simplification are the main inprocessing mechanisms available
% to superposition provers. Some provers also implement clause splitting
% \cite{riazanov-voronkov-2001,fietzke-weidenbach-2009,av-2014-avatar}.

% Superposition provers saturate the input problem with respect to the calculus's
% inference rules using the \emph{given clause procedure}
% \cite{mcw-1997-otter,adf-1995-discount}. It partitions the proof state into a
% passive set $\mathcalx{P}$ and an active set $\mathcalx{A}$. All clauses start in
% $\mathcalx{P}$. At each iteration of the procedure's main loop, the prover
% chooses a clause $C$ from $\mathcalx{P}$, simplifies it, and moves it to
% $\mathcalx{A}$. Then all inferences between $C$ and active clauses are performed.
% The resulting clauses are again simplified and put in $\mathcalx{P}$.\confrep{}{
% The provers differ in which clauses are used for simplification: Otter-loop
% \cite{mcw-1997-otter} provers use both active and passive clauses whereas
% DISCOUNT-loop \cite{adf-1995-discount} provers use only active clauses.}

All of the techniques described in this chapter are applied in the context of
first-order superposition which is introduced in Sect.~\ref{sec:pre:sup}. We
reuse all the notions introduced in this section and the corresponding notation. 
