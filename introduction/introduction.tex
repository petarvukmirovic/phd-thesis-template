\chapter{Introduction}
\label{introduction}

\dropcap{H}alf sleeping, half playing Fruit Ninja on my smartphone at one of the
philosophy lectures in my high school I heard my professor exclaim with
excitement: “Mathematical truth is the highest and absolute form of truth!”.
Unbothered and uninterested, I continued playing Fruit Ninja.

Many years later I started studying various formal methods in computer science.
There I learned the value of formal, mathematical languages and finally
understood him. Not only does using a rigorous mathematical language avoid the
ambiguity of natural language, it allows us to use formal rules of reasoning to
make conclusions from initial arguments in a trustworthy manner. 

Though the interest for rigorous and formal reasoning dates back to Aristotle,
it intensified at the end of 19th and the beginning of 20th century
\cite{jf-01-modern-logic}. Researchers in that time were mostly interested in
finding a formal language that is expressive enough to describe complex
mathematical theories, yet intuitive and understandable enough to allow
reasonably simple formal reasoning systems. First-order logic seemed to satisfy
both requirements.  This formalism does not only allow one to model simple
logical relations such as ``if it rains, the road is muddy'' (or formally
$\cst{rains} \imp \cst{muddy}$), but also to quantify over objects as in ``all
people are mortal'' ($\forall x.\, \cst{human}(x) \imp \cst{mortal}(x)$).

The initial study of this formalism and the rules to reason about statements in
it was very prolific. In 1930 G\"odel \cite{kg-30-completeness-theorem} showed
that there is a system of inference rules such that for any true (valid) statement a
proof of validity can be constructed in this system. Around the same time, the
first computers were leaving the imagination of the engineers and began to
automate many complicated processes.  Naturally, the question as to  whether a
computer can decide if a first-order statement is valid (i.e., a theorem) arose.
Church \cite{ac-36-fol-undecidable} and Turing \cite{tm-37-undecidable} answered
this question negatively in 1936.





% \begin{abstract}
% Sample Abstract.
% \end{abstract}

% \blfootnote{This chapter is partly based on \faFileTextO~\emph{M. Beller. Toward an
%     Empirical Theory of Feedback-Driven Development, ICSE'18 (Student Research Competition)}~\cite{BellerSRC2018}.
% }


% \newpage

% \dropcap{T}his is a introductory page.

% \section{Background \& Context}
% In this thesis, you can reference pictures~\Cref{fig:devmodel} using Cleverref and circles \circled{5}.

% \begin{figure}[htb]
% 	\centering
% 	\includegraphics[width=0.65\columnwidth]{development_model_without_papers}
% 	\caption{The stages of the FDD model and their relationship to other
%           Software Engineering concepts.}
% 	\label{fig:devmodel}
% \end{figure}

% We also have lists:

% \begin{enumerate}
%   \item Static Analysis~\circled{3} examines program artifacts or
%     their source code without executing them~\cite{wichmann1995industrial}, while
%  \item Dynamic Analysis~\circled{4} relies on information gathered from their
%    execution~\cite{cornelissen2009systematic}.
% \end{enumerate}

% Or boxes:

% \begin{framed}
% This thesis is concerned with the empirical assessment of the state of the art of how developers
% drive software development with the help of feedback loops.
% \end{framed}

% Or code:
% \begin{lstlisting}[caption={\textsc{TrinityCore}},label={lst:e1}]
%  x += other.x;
%  y += other.y;
%  z += other.y;
% \end{lstlisting}


% I hope this helps you get started!
% Moritz
