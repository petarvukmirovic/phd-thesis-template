\chapter{Introduction}
\setheader{Introduction}
\label{ch:intro}

Half sleeping, half playing Fruit Ninja on my smartphone at one of the
philosophy lectures in my high school I heard my professor exclaim with
excitement: “Mathematical truth is the highest form of truth!”.
Unbothered and uninterested, I continued playing Fruit Ninja.

Many years later I started studying various formal methods in computer science.
There I learned the value of formal, mathematical language and finally
understood him. Not only does using a rigorous mathematical language avoid the
ambiguity of the natural one, it allows us to use the formal rules of reasoning
to make conclusions from initial arguments in a trustworthy manner. 

Though the interest for rigorous and formal reasoning dates back to Aristotle,
it intensified at the end of 19th and the beginning of 20th century
\cite{jf-01-modern-logic}. Researchers in that time were mostly interested in
finding a formal language that is expressive enough to describe complex
mathematical theories, yet intuitive and understandable enough to allow
reasonably simple formal reasoning systems. First-order logic seemed to satisfy
both requirements.  This formalism not only allows one to model simple
logical relations such as ``if it rains, the road is muddy'' (or formally
$\cst{rains} \iimplies \cst{muddy}$), but also to quantify over objects as in ``all
people are mortal'' ($\iforall x.\, \cst{human}(x) \iimplies \cst{mortal}(x)$).

\paragraph{Automatic Theorem Proving}The initial study of first-order logic and the rules to reason about statements in
it was very prolific. In 1930 G\"odel \cite{kg-30-completeness-theorem} showed
that there is a system of inference rules (calculus) such that for any valid
statement a proof of validity can be constructed in this system. Around the same
time, the first computers were leaving the imagination of the engineers and
began automating many complicated processes.  Naturally, the question as to
whether a computer can decide if a first-order statement is valid (i.e., a
theorem) arose. Church \cite{ac-36-fol-undecidable} and Turing
\cite{tm-37-undecidable} answered this question negatively in 1936.

Despite this negative result, the prospect of automatically proving theorems of
first-order logic remained too enticing. In 1960, Davis and Putnam
\cite{dp-60-dp} described a procedure for checking validity of a
first-order formula. As first-order logic is undecidable, this procedure
terminated only on valid formulas. Even though it was efficient enough to prove
only simple formulas, it was an impressive achievement for the time.

A bigger breakthrough happened in 1965 when Robinson introduced a calculus that
would shape the future of automatic reasoning --- the resolution calculus
\cite{ar-65-resolution}. Consisting of a single inference rule it was simple and
elegant. It was also \emph{complete}: each theorem could be proven using this
system. Furthermore, it was less explosive than Davis and Putnam's algorithm
since it provided better guidance for search space exploration.

Since its introduction, resolution was improved in many ways. Strategies and
heuristics~\cite{lw-65-sos} to curb the explosion of the search space were
introduced, as well as complete, but less explosive variants of the calculus
\cite{cc-73-resolution-book}. Despite this progress, reasoning about equality of
objects was still hard for these methods. This changed with introduction of
superposition~\cite{bg-94-superposition} in the early 1990s.

Superposition is a complete calculus for first-order logic loosely based on
resolution, featuring support for equality on the calculus level. This support
is further optimized by use of various heuristics built to help prune the
search space. Thanks to efficient implementation, this calculus was used to
prove long-standing open problem known as the Robbins conjecture
\cite{mccune-97-robbins}. Provers based on this calculus have been
winning the first-order division of the CASC theorem proving competition since its
inception in the 1990s \cite{ss-96-casc}.

Even though the successes of proving the Robbins conjecture showed that it is
possible, automatically proving hard mathematical problems is still out of reach
for theorem provers. However, there are other ways in which humans and computers can work together to construct formal
proofs.

\paragraph{Proof Assistants and Hammers} Proof assistants (also called interactive
theorem provers) are programs which allow users to describe a theory formally
and to write proofs of the statements about this theory using inference 
rules of the proof assistant. These programs can then check if the given proof
is correct. As their core is usually small and well-tested, proofs that pass the
checking phase are considered trustworthy.

Proof assistants were essential in proving very complex mathematical theorems in a
trustworthy manner. For example, the four-color theorem \cite{gg-2008-four-color}
and the Kepler conjecture \cite{th-2015-kepler} were proven inside proof assistants,
ending the need to trust hundreds of pages of mathematical proofs. Proof
assistants are also used to show correctness of complex software such as
compilers \cite{xl-09-compcert} or operating systems kernels \cite{gk-09-sel4}.
This verified software can be used in safety-critical scenarios such as nuclear
plants or autonomous vehicles.

The proofs need to be spelled out in minute detail to be accepted by a proof assistant.
Even though many proof assistants offer some sort of automation, seemingly simple
statements cannot be automatically proved. This is one of the reasons why large
verification projects can take years to finish.

Integrating proof automation directly into the proof assistant is not an easy task.
Most proof assistants are based on variants of higher-order logic, which is more
expressive than first-order logic, but harder to automate as in general it does not even allow complete
proof calculi. To provide some level of automation, proof assistant developers
use the efficient first-order provers as follows. First, the proof goal is
translated from higher-order to first-order logic. Then the first-order prover
is run on the translated problem. If the first-order prover finds the proof, the
proof is reconstructed in the proof assistant. Modern assistants such as Coq \cite{bc-04-coq}, HOL Light \cite{jh-09-hol-light}
and Isabelle \cite{lc-88-isabelle} implement the approach using the \emph{hammers} CoqHammer
\cite{ck-18-coqhammer}, HOL(y)Hammer \cite{ku-15-holyhammer} and Sledgehammer
\cite{bn-10-sh}, respectively. 

The approach proved useful: on some benchmarks, around 77\% of the proof goals can be
discharged automatically using Sledgehammer \cite{bgkku-16-larning-fact-selector}. However, there is
clearly some untapped potential in the approach as it communicates with
automatic theorem provers through a translation. To bridge this translation gap,
attempts were made to use higher-order automatic theorem provers instead of
first-order ones \cite{ns-13-leo2sh}. This proved unfruitful as state-of-the-art
higher-order theorem provers of the time were primarily designed for small and
tricky problems. On the other hand, problems coming from hammers consist of
mostly first-order formulas, are rather large, and mostly require simple
higher-order proof steps. An example of a problem that is seemingly simple, but whose translation to first-order logic
is out of reach for even the best of first-order provers is \cite{bbbntvw-2x-mechanical-mathematicians}:

$$ \big(\sum_{i=1}^{n} i^2 + 2i + 1\big) \ieq \big(\sum_{i=1}^{n} i^2\big) +
\big(\sum_{i=1}^{n} 2i\big) + \big(\sum_{i=1}^{n} 1\big) $$ \noindent One of the
reasons why this problem is hard is that higher-order proof assistants rely on
unnamed functions, called \emph{$\lambda$-abstractions}, to represent
variable-binding mathematical operators such as summations, integrals and
limits. More precisely, the left-hand side of the above equation is usually
represented as $\cst{sum}(1, n, \lambda i. \, i\cst{*}i + 2\cst{*}i + 1)$. When
translated to first-order logic, $\lambda$-abstractions, which are ubiquitous as
all-purpose binders in higher-order logic, are obfuscated to the point that all
but the simplest  reasoning about them are out of reach for first-order provers.
However, for higher-order provers the crux of the proof is matching the axiom
$\iforall i j f g.\, \cst{sum}(i, j, \lambda x. \, f \, x \, \cst{+} \, g\,x) \ieq
(\cst{sum}(i, j, \lambda x. \, f \, x ) \, \cst{+} \, \cst{sum}(i, j, \lambda x. \, g
\, x ) )$ against the goal. This operation is relatively simple and, in general, much more efficient
than first-order reasoning about translations.


Thus, from the standpoint of proof assistants the ideal
automatic theorem prover fulfills the following wishlist:
\begin{enumerate}
  \item\label{it:fo} Performs as the best first-order provers on first-order problems
  \item\label{it:size} Scales well with the size of the problem
  \item\label{it:ho} Supports higher-order logic and scales with the amount of higher-order axioms
\end{enumerate}
This brings us to the topic of this thesis: \emph{How to develop a prover fulfilling this wishlist?}

\paragraph{Our Approach} To improve the automatic reasoning for higher-order
logic, we decided not to start from a blank slate and build a new theorem prover.
Rather, we start from a position of strength---from the state-of-the-art
first-order superposition theorem prover E~\cite{scv-19-e23}---and extend it to higher-order logic. In this way, the points \ref{it:fo} and
\ref{it:size} from the above wishlist are already fulfilled, as E is very
efficient and deals well with large problems.

We do not extend E to higher-order logic in one atomic step. Instead, we extend
it by adding support for features of higher-order logic one by one,
which gradually increase its reasoning capabilities. With this gradual
approach we make sure that after every extension, E's
efficiency on first-order problems did not decrease. Thus, while trying
to fulfill the point \ref{it:ho} from the wishlist we make sure that points
\ref{it:fo} and \ref{it:size} are not affected.

We have identified three stops on the road to full higher-order logic. A
distinguished feature of higher-order logic is that it not only supports
quantification over objects, but also over functions that manipulate these
objects. Thus, higher-order logic allows us to make a statement ``all functions
with arguments $0$ and $1$ result in $2$" ($\iforall x. \,  x (\cst{0}, \cst{1}) \eq
\cst{2}$). Replacing $x$ with the addition function ($\cst{+}$) is enough to
disprove this statement. The first stop is to extend E to support
quantification over functions that are already present in the theory.

Higher-order logic also supports expressing functions that are not explicitly
present in the theory. Suppose we want to prove there is a
function that returns 4 when given 2 and 5 when given 3 ($\iexists x. x(\cst{2})
\eq \cst{4} \lland x (\cst{3}) \eq \cst{5} $). An example of such function is
the function that adds 2 to its argument. In higher-order logic such unnamed
functions can be represented using $\lambda$-abstractions as $\lambda y.\, y \,
\cst{+} \, \cst{2}$. This term suffices to prove the original statement. The
second stop is to support automatically synthesizing such functions.

In higher-order logic we can make statements about statements. More precisely,
it treats Boolean terms (formulas) as first-class citizens, enabling us to
model some concepts in a more natural manner. For example, we can naturally
represent statements such as ``what Jasmin says is true'' as $\iforall x. \,
\cst{says}(\cst{jasmin}, x) \iimplies x$. The last stop is to support Boolean terms
as first-class citizens.
\pagebreak[2]

\section{Contributions}

The main contribution of this thesis is the extension of two theorem provers,
E and Zipperposition \cite{sc-15-simon-phd}, to full higher-order logic. To
extend these provers, we faced many challenges. Some of them were of engineering
nature, some of them were algorithmic, while others concerned fine-tuning the
heuristics to optimize the performance. We discuss those challenges in more detail.

  
  First, we implemented the complete superposition calculus for $\lambda$-free
  higher-order logic \cite{bbcw-21-lfho} in E, obtaining a prover we call Ehoh.
  We had to modify the internal term representation of E to support $\lambda$-free
  higher-order terms, implement new unification and matching algorithms, extend
  term indexing data structures to work with higher-order terms, and improve the
  performance of heuristics on higher-order problems.
  
  Together with Alexander Bentkamp and other colleagues we developed
  a complete superposition calculus for the second phase described above
  \cite{bbtvw-21-sup-lam}. Due to the complexity of the calculus and many design
  decisions that had to be evaluated in a short amount of time we decided to
  implement the calculus in Zipperposition \cite{sc-15-simon-phd,sc-supind-17}. This prover is less efficient than
  E, but it is implemented in OCaml and it was designed to make prototyping
  ideas and experimenting with calculi much easier. In 2019 we entered the
  higher-order division of CASC theorem proving competition~\cite{gs-19-casc27} with Zipperposition. 
  It finished third, one percent point behind Leo-III
  \cite{sb-21-leo3}  and 12 percent points behind  Satallax
  \cite{cb-12-satallax}.
  
  Through this experience we identified one of the bottlenecks of
  Zipperposition: its unification algorithm. To remove this bottleneck we
  developed a complete higher-order unification algorithm that is more
  efficient and generates fewer redundant unifiers than the state of the art. 
  
  We also noticed that Zipperposition's support for first-class Booleans
  was lacking compared to competition. We have closely studied the problems on
  which Zipperposition fails but are easily solved by other provers, and created
  a set of incomplete rules that enhance Zipperposition's Boolean support.
  Zipperposition 2 implemented these rules together with the new unification
  algorithm. In 2020, Zipperposition 2 won the higher-order division
  of CASC \cite{gs-21-cascj10}. This time it was 20 percent points ahead of the second best
  prover, Satallax.
  
  Inspired by this success, we implemented the complete superposition
  calculus for higher-order logic with Booleans (i.e., the third phase), that we developed with Bentkamp and other
  colleagues \cite{bbtv-21-full-ho-sup}. Developing this calculus helped us understand in which
  ways our pragmatic extension was too weak to prove hard theorems of
  higher-order logic. We further fine-tuned the heuristics and implemented new ways
  to tame the search space explosion. In 2021 we released Zipperposition 2.1
  which implemented these new features. It again won
  CASC, outperforming Zipperposition 2. It was 16 percent points ahead of the
  runner-up, Vampire 4.5 \cite{lkav-13-vampire}. 
  
  Building on all the experience we gathered while working on Zipperposition,
  we further extended Ehoh to full higher-order logic, obtaining a prover we call
  \newehoh{}. We again modified the term representation, implemented new unification and
  matching algorithms, extended indexing data structures, and tweaked
  heuristics. Even though this extension has a larger scope than the original
  extension from E to Ehoh, the strong basis that we had built earlier made it
  manageable.
  \pagebreak[2]

  
  After spending so much time and energy on developing higher-order provers, we
  took a step back and worked on improving first-order provers. As they form the
  basis of higher-order reasoning in our approach, this should improve
  higher-order performance as well. In particular, we looked at efficient
  approaches to simplify problems in propositional logic, which is simpler than
  first-order logic and even decidable. We lifted some of these propositional
  approaches to first-order logic and implemented them in Zipperposition. 

\section{Thesis Structure and Publications}

This thesis is cumulative: chapters~\ref{ch:ehoh}--\ref{ch:satfol}  are
taken from previous publications (or drafts in case of
Chapter~\ref{ch:ehoh2}) where I was the main author. Use
of the publications in this thesis was authorized by the coauthors. 

Chapter~\ref{ch:pre} introduces the background material necessary for following this thesis,
while Chapter~\ref{ch:conclusion} gives a brief summary of the work and describes alleys for future work.
The other six chapters discuss previously described contributions in detail:

\vspace{0.5em}
\noindent\textbf{Chapter~\ref{ch:ehoh}} discusses the steps we have taken to extend first-order
  prover E to $\lambda$-free higher-order logic, obtaining Ehoh. It is mostly based on the journal publication
  \begin{enumerate}
    \item Petar Vukmirović, Jasmin Blanchette, Simon Cruanes, and Stephan Schulz.
    Extending a Brainiac Prover to Lambda-Free Higher-Order Logic.
    In \emph{International Journal on Software Tools for Technology Transfer}. Springer, 2021. Published online only.
  \end{enumerate}
\vspace{-0.3em}
while parts of the chapter were previously published in the conference publication
\vspace{-0.3em}
  \begin{enumerate}[resume]
    \item Petar Vukmirović, Jasmin Blanchette, Simon Cruanes, and Stephan Schulz.
    Extending a Brainiac Prover to Lambda-Free Higher-Order Logic.
    In Vojnar, T., Zhang, L. (eds.) \emph{TACAS 2019}, LNCS 11427, pp. 192--210, Springer, 2019.
  \end{enumerate}
\noindent\textbf{Chapter~\ref{ch:unif}} discusses the higher-order unification procedure we designed to be used
inside efficient theorem provers. It is mostly based on the journal publication
  \begin{enumerate}[resume]
    \item Petar Vukmirović, Alexander Bentkamp, and Visa Nummelin. Efficient Full Hi\-gher-Order Unification. 
    In \emph{Logical Methods in Computer Science} 17(4): 18:1--18:31, 2021.
  \end{enumerate}
\vspace{-0.3em}
while parts of the chapter were previously published in the conference
publication, which won the best paper by a junior researcher award:
\vspace{-0.3em}
\begin{enumerate}[resume]
  \item Petar Vukmirović, Alexander Bentkamp, and Visa Nummelin. Efficient Full Hi\-gher-Order Unification. 
  In Ariola, Z.M. (ed.) \emph{FSCD 2020} LIPIcs 167, pp. 5:1--5:17, Schloss Dagstuhl--Leibniz-Zentrum für Informatik, 2020. 
\end{enumerate}
\noindent\textbf{Chapter~\ref{ch:bools}} discusses the pragmatic extensions of
the complete calculus for superposition with $\lambda$-abstractions to support first-class
Booleans. It is based on the publication
    \begin{enumerate}[resume]
      \item Petar Vukmirović and Visa Nummelin. Boolean Reasoning in a Higher-Order
      Superposition Prover. In P. Fontaine, K. Korovin, I.S. Kotsireas, P.
      R\"ummer, S. Tourret (eds.) \emph{PAAR-2020}, CEUR Workshop Proceedings, vol.
      2752, pp. 148--166. CEUR-WS.org (2020)
    \end{enumerate}
\noindent\textbf{Chapter~\ref{ch:ho-techniques}} discusses the approach we took to
control the explosion of the search space inherent to higher-order superposition. It is mostly based on the journal publication
  \begin{enumerate}[resume]
    \item Petar Vukmirović, Alexander Bentkamp, Jasmin Blanchette, Simon Cruanes, Visa Nummelin, and Sophie Tourret.
    Making Higher-Order Superposition Work. In \emph{Journal of Automated Reasoning}. Springer, 2022.
  \end{enumerate}
\vspace{-0.3em}
  while parts of the chapter were previously published in the conference publication, which won the best paper by a student award:
\vspace{-0.3em}
\begin{enumerate}[resume]
  \item Petar Vukmirović, Alexander Bentkamp, Jasmin Blanchette, Simon Cruanes, Visa Nummelin, and Sophie Tourret.
  Making Higher-Order Superposition Work. In Platzer, A., Sutcliffe, G. (eds.) \emph{CADE-28},
  LNCS 12699, pp. 415--432, Springer, 2021.
\end{enumerate}
\noindent\textbf{Chapter~\ref{ch:ehoh2}} discusses the extension of Ehoh to
support full higher-order logic, resulting in \newehoh.
It is based on the draft 
\begin{enumerate}[resume]
  \item Petar Vukmirović, Jasmin Blanchette, and Stephan Schulz.  Extending a
  High-Per\-for\-mance Prover to Higher-Order Logic. 2022.
\end{enumerate}

\noindent\textbf{Chapter~\ref{ch:satfol}} discusses lifting some of the techniques used to
simplify formulas in propositional logic to first-order logic. It is based on the technical report version of the publication:
  \begin{enumerate}[resume]
    \item Petar Vukmirović, Jasmin Blanchette, and Marijn J.H. Heule. 
    SAT-Inspired Eliminations for Superposition. In Piskac, R., Whalen, M. (eds.) \emph{FMCAD 2021}, pp. 231--240, IEEE, 2021. 
  \end{enumerate}

As a PhD candidate, I coauthored the following articles and papers, that not included in this thesis:

\begin{enumerate}[resume]
  \item Martin Desharnais, Petar Vukmirović, Jasmin Blanchette, and Makarius Wenzel. Seventeen Provers Under the Hammer.
  Accepted at ITP 2022.
  % \item Alexander Bentkamp, Jasmin Blanchette, Sophie Tourret, and Petar Vukmirović. Superposition for Higher-Order Logic.
  % Draft article.
  \item Alexander Bentkamp, Jasmin Blanchette, Sophie Tourret, Petar Vukmirović, and Uwe Waldmann.  Superposition with Lambdas.
  \emph{Journal of Automated Reasoning} 65(7): 893--940, 2021. 
  \item Visa Nummelin, Alexander Bentkamp, Sophie Tourret, and Petar Vukmirović. Superposition with First-Class Booleans and Inprocessing Clausification.
  In Platzer, A., Sutcliffe, G. (eds.) \emph{CADE-28}, LNCS 12699, pp. 378--395, Springer, 2021.
  \item Alexander Bentkamp, Jasmin Blanchette, Sophie Tourret, and Petar Vukmirović. Superposition for Full Higher-Order Logic.
  In Platzer, A., Sutcliffe, G. (eds.) \emph{CADE-28}, LNCS 12699, pp. 396--412, Springer, 2021.
  \item Stephan Schulz, Simon Cruanes, and Petar Vukmirović. Faster, Higher, Stronger: E 2.3.
  In Fontaine, P. (ed.) \emph{CADE-27},  LNCS 11716, pp. 495--507, Springer, 2019.
  \item Alexander Bentkamp, Jasmin Blanchette, Sophie Tourret, Petar Vukmirović, and Uwe Waldmann.  Superposition with Lambdas.
  In Fontaine, P. (ed.) \emph{CADE-27}, LNCS 11716, pp. 55--73, Springer, 2019. 
\end{enumerate}

\section{Related Work}

The earliest attempts to automate higher-order logic can be traced back to
Robinson, who provided two approaches. The first one \cite{ar-69-hol} operates
directly on higher-order formulas. The second approach \cite{ar-70-hol} is based
on translating higher-order logic to first-order logic using combinators,
similar to the approach taken by hammers. Huet~\cite{gh-73-hol} and
Andrews~\cite{pa-71-type-theory} designed calculi that are directly applied to
higher-order formulas.

\looseness=-1
These attempts resulted in early higher-order theorem provers. Andrews and
colleagues developed TPS~\cite{abinpx-96-tps}: an automatic prover which allows
users to specify proof outlines, based on expansion proofs. Benzm\"uller and
colleagues developed LEO \cite{cbmk-98-leo}, a prover based on higher-order
resolution which introduced the cooperative paradigm. Provers implementing this
paradigm invoke first-order backends in regular intervals to finish the proof.
The latest iteration in the LEO family of provers is Leo-III \cite{sb-21-leo3}.
Satallax \cite{cb-12-satallax} is based on higher-order analytic tableaux. Such
representations of logical formulas are tree structures with at each node a
subformula of the original formula to be proved or refuted. Satallax improves on
the method by using a SAT solver to close tableau branches.
It won the higher-order division of CASC on eight occasions: in 2011 and every
year from 2013 to 2019.

Various researchers also experimented with the idea of extending a first-order
prover to higher-order logic. Beeson \cite{mb-04-lam-logic} implemented second-order
unification and added $\lambda$-terms to one of the best superposition
provers at the time, Otter~\cite{mcc-03-otter}. Cruanes extended Zipperposition
\cite{sc-15-simon-phd,sc-supind-17} with arithmetic, induction, and rudimentary support for
higher-order logic. Bhayat and Reger implemented a complete superposition
calculus  for higher-order logic, based on combinators  in Vampire
\cite{br-20-full-sup-w-combs}. Authors of SMT (satisfiability modulo theories) solvers followed a similar path:
they extended cvc4~\cite{cbetal-11-cvc4} and veriT~\cite{bodf-09-veriT} to
support higher-order logic \cite{brotb-19-ho-smt}.

Higher-order unification is one of the central procedures in a higher-order
prover. Jensen and Pietrzykowski introduced a complete procedure for enumerating
elements unifiers for higher-order terms \cite{jp-76-unif}. Huet noticed that
some calculi remain complete if solving a difficult subproblem of higher-order
unification (``flex-flex'' pairs) is delayed. He described a procedure to solve
this easier problem in a more efficient way \cite{gh-75-unification}. Dougherty
desgined a procedure to enumerate higher-order unifiers using higher-order
combinators~\cite{dd-93-comb-unif}. 

To make unification more efficient, Bhayat
and Reger implemented efficient term indexing in their extension of Vampire
\cite{br-20-full-sup-w-combs}. Libal and Steen designed efficient term indexing
based on substitution trees \cite{ls-16-indexing}.

There have been many studies on the performance of algorithms or success rate of
heuristics in first-order theorem proving. For example, Hoder and Voronkov
evaluated different unification algorithms \cite{hv-09-unifalgs}, while Reger and
Voronkov evaluated different parameters of a novel prover architecture
\cite{rsv-15-playing-with-avatar}. Other authors evaluated different proof
search heuristics for superposition provers \cite{gs-20-clausesel,
hrsv-16-selsel, sm-16-clausesel}. Evaluation of many prover parameters has been
done for higher-order provers as well
\cite{benzmueller-et-al-05-can-ho-fo-coop,fb-2016-internal-guidance-satallax,sb-15-beta,wskb-16-effective-norm}.
